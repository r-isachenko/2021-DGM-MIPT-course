\documentclass{beamer}
\usepackage[utf8]{inputenc}
\usepackage{graphicx, epsfig}
\usepackage{amsmath,mathrsfs,amsfonts,amssymb}
%\usepackage{subfig}
\usepackage{floatflt}
\usepackage{epic,ecltree}
\usepackage{mathtext}
\usepackage{fancybox}
\usepackage{fancyhdr}
\usepackage{multirow}
\usepackage{enumerate}
\usepackage{epstopdf}
\usepackage{multicol}
\usepackage{algorithm}
\usepackage[noend]{algorithmic}
\usepackage{tikz}
\usepackage{blindtext}
\usetheme{default}%{Singapore}%{Warsaw}%{Warsaw}%{Darmstadt}
\usecolortheme{default}
\setbeamerfont{title}{size=\Huge}
\setbeamertemplate{footline}[page number]{}


\makeatletter
\newcommand\HUGE{\@setfontsize\Huge{35}{40}}
\makeatother    

\setbeamerfont{title}{size=\HUGE}
\beamertemplatenavigationsymbolsempty

\input{../utils/newcommands}
\input{../utils/title}

\newcommand\myfootnote[1]{%
  \tikz[remember picture,overlay]
  \draw (current page.south west) +(1in + \oddsidemargin,0.5em)
  node[anchor=south west,inner sep=0pt]{\parbox{\textwidth}{%
      \rlap{\rule{10em}{0.4pt}}\raggedright\scriptsize \textit{#1}}};}

\newcommand\myfootnotewithlink[2]{%
  \tikz[remember picture,overlay]
  \draw (current page.south west) +(1in + \oddsidemargin,0.5em)
  node[anchor=south west,inner sep=0pt]{\parbox{\textwidth}{%
      \rlap{\rule{10em}{0.4pt}}\raggedright\scriptsize\href{#1}{\textit{#2}}}};}
\createdgmtitle{11}
%--------------------------------------------------------------------------------
\begin{document}
%--------------------------------------------------------------------------------
\begin{frame}[noframenumbering,plain]
%\thispagestyle{empty}
\titlepage
\end{frame}
%=======
\begin{frame}{Recap of previous lecture}
	\begin{block}{Standard GAN}
		\vspace{-0.6cm}
		\[
			\min_{G} \max_D V(G, D) = \min_{G} \max_D \left[ \bbE_{\pi(\bx)} \log D(\bx) + \bbE_{p(\bz)} \log (1 - D(G(\bz))) \right]
		\]
		\vspace{-0.7cm}
	\end{block}
	\begin{block}{Main problems}
		\begin{itemize}
			\item Vanishing gradients (non-saturating GAN does not suffer of it);
			\item Mode collapse (caused by behaviour of Jensen-Shannon divergence).
		\end{itemize}
	\end{block}
	\vspace{-0.1cm}
	\begin{block}{Informal theoretical results}
		Distribution of real images $\pi(\bx)$ and distribution of generated images $p(\bx | \btheta)$ are low-dimensional and have disjoint supports. In this case
		\vspace{-0.3cm}
		\[
			KL(\pi || p) = KL(p || \pi) = \infty, \quad JSD(\pi || p) = \log 2
		\]
		\end{block}
		\myfootnote{\href{https://arxiv.org/abs/1406.2661}{Goodfellow I. J. et al. Generative Adversarial Networks, 2014} \\
		\href{https://arxiv.org/abs/1701.04862}{Arjovsky M., Bottou L. Towards Principled Methods for Training Generative Adversarial Networks, 2017}}
\end{frame}
%=======
\begin{frame}{Recap of previous lecture}
		\begin{block}{Wasserstein distance}
			\vspace{-0.5cm}
			\[
				W(\pi, p) = \inf_{\gamma \in \Gamma(\pi, p)} \bbE_{(\bx, \by) \sim \gamma} \| \bx - \by \| =  \inf_{\gamma \in \Gamma(\pi, p)} \int \| \bx - \by \| \gamma (\bx, \by) d \bx d \by
			\]
			\vspace{-0.5cm}
			\begin{itemize}
				\item $\Gamma(\pi, p)$ -- the set of all joint distributions $\Gamma (\bx, \by)$ with marginals $\pi$ and $p$ ($\int \gamma(\bx, \by) d \bx = p(\by)$, $\int \gamma(\bx, \by) d \by = \pi(\bx)$)
				\item $\gamma(\bx, \by)$ -- transportation plan (the amount of "dirt" that should be transported from point $\bx$ to point $\by$).
				\item $\gamma(\bx, \by)$ -- the amount, $\|\bx - \by \|$-- the distance.
			\end{itemize}
		\end{block}
		\begin{block}{Theorem (Kantorovich-Rubinstein duality)}
			\vspace{-0.2cm}
			\[
				W(\pi || p) = \frac{1}{K} \max_{\| f \|_L \leq K} \left[ \bbE_{\pi(\bx)} f(\bx)  - \bbE_{p(\bx)} f(\bx)\right],
			\]
			where $\| f \|_L \leq K$ are $K-$Lipschitz continuous functions ($f: \cX \rightarrow \bbR$).
		\end{block}
		\myfootnotewithlink{https://arxiv.org/abs/1701.07875}{Arjovsky M., Chintala S., Bottou L. Wasserstein GAN, 2017}
\end{frame}
%=======
\begin{frame}{Recap of previous lecture}
		\begin{block}{WGAN objective}
			\vspace{-0.5cm}
			\[
				\min_{G} W(\pi || p) = \min_{G} \max_{\bphi \in \boldsymbol{\Phi}} \left[ \bbE_{\pi(\bx)} f(\bx, \bphi)  - \bbE_{p(\bz)} f(G(\bz), \bphi )\right].
			\]
			\vspace{-0.5cm}
		\end{block}
		\begin{itemize}
			\item Function~$f$ in WGAN is usually called $\textit{critic}$.
			\item If parameters $\bphi$ lie in a compact set $\boldsymbol{\Phi} \in [-0.01, 0.01]^d$ then $f(\bx, \bphi)$ will be $K$-Lipschitz continuous function. 
		\end{itemize}
		\begin{block}{Gradient penalty}
			\vspace{-0.7cm}
			\[
				W(\pi || p) = \underbrace{\bbE_{\pi(\bx)} f(\bx)  - \bbE_{p(\bx)} f(\bx)}_{\text{original critic loss}} + \lambda \underbrace{\bbE_{U[0, 1]} \left[ \left( \| \nabla_{\hat{\bx}} f(\hat{\bx}) \|_2 - 1 \right) ^ 2\right]}_{\text{gradient penalty}}.
			\]
			\vspace{-0.7cm}
		\end{block}
		Samples $\hat{\bx}_t = t \bx + (1 - t) \by$ with $t \in [0, 1]$ are uniformly sampled along straight lines between pairs of points: $\bx$ from the data distribution $\pi(\bx)$ and $\by$ from the generator distribution $p(\bx | \btheta)$.
		\myfootnote{
		\href{https://arxiv.org/abs/1701.07875}{Arjovsky M., Chintala S., Bottou L. Wasserstein GAN, 2017} \\
		\href{https://arxiv.org/abs/1704.00028}{Gulrajani I. et al. Improved Training of Wasserstein GANs, 2017} }
\end{frame}
%=======
\begin{frame}{Spectral Normalization GAN}
	\begin{block}{Definition}
		$\| \bA \|_2$ is a \textit{spectral norm} of matrix $\bA$:
		\[
			\| \bA \|_2 = \max_{\bh \neq 0} \frac{\|\bA \bh\|_2}{\|\bh\|_2} = \max_{\|\bh\|_2 \leq 1} \| \bA \bh \|_2 = \lambda_{\text{max}}(\bA^T \bA),
		\]
		where $\lambda_{\text{max}}(\bA^T \bA)$ is the largest eigenvalue value of $\bA^T \bA$.
	\end{block}
	\begin{block}{Statement 1}
		if $g$ is a K-Lipschitz function then 
		\[
			\| \mathbf{g} \|_L \leq K = \sup_\bx \| \nabla \mathbf{g}(\bx)) \|_2.
		\]
		\vspace{-0.7cm}
	\end{block}
	\begin{block}{Statement 2}
		Lipschitz norm of superposition is bounded above by product of Lipschitz norms
		\vspace{-0.2cm}
		\[
			\| \mathbf{g}_1 \circ \mathbf{g}_2 \|_L \leq \| \mathbf{g}_1 \|_L \cdot \| \mathbf{g}_2\|_L
		\]
	\end{block}
	\myfootnotewithlink{https://arxiv.org/abs/1802.05957}{Miyato T. et al. Spectral Normalization for Generative Adversarial Networks, 2018}
\end{frame}
%=======
\begin{frame}{Spectral Normalization GAN}
	Let consider the critic $f(\bx, \bphi)$ of the following form:
	\[
		f(\bx, \bphi) = \bW_{K+1} \sigma_K (\bW_K \sigma_{K-1}(\dots \sigma_1(\bW_1 \bx) \dots)).
	\]
	This feedforward network is a superposition of simple functions.
	\begin{itemize}
		\item $\sigma_k$ is a pointwise nonlinearities. We assume that $\| \sigma_k \|_L = 1$ (it holds for ReLU).
		\item $\mathbf{g}(\bx) = \bW \bx$ is a linear transformation ($\nabla \mathbf{g}(\bx) = \bW$).
		\[
			\| \mathbf{g} \|_L \leq \sup_\bx \| \nabla \mathbf{g}(\bx)\|_2 = \| \bW \|_2.
		\]
	\end{itemize}
	\vspace{-0.4cm}
	\begin{block}{Critic spectral norm}
		\vspace{-0.4cm}
		\[
			\| f \|_L \leq \| \bW_{K+1}\| \cdot \prod_{k=1}^K  \| \sigma_k \|_L \cdot \| \bW_k \|_2 = \prod_{k=1}^{K+1} \|\bW_k\|_2.
		\]
		\vspace{-0.2cm}
	\end{block}
	If we replace the weights in the critic $f(\bx, \bphi)$ by $\bW^{SN}_k = \bW_k / \|\bW_k\|_2$, we will get $\| f\|_L \leq 1.$ \\
	
	\myfootnotewithlink{https://arxiv.org/abs/1802.05957}{Miyato T. et al. Spectral Normalization for Generative Adversarial Networks, 2018}
\end{frame}
%=======
\begin{frame}{Spectral Normalization GAN}
	How to compute $ \| \bW \|_2 = \lambda_{\text{max}}(\bW^T \bW)$? \\
	 If we apply SVD to compute the $\| \bW \|_2$ at each iteration, the algorithm becomes intractable.
	 
	 \begin{block}{Power iteration method}
	 	\begin{itemize}
	 		\item $\bu_0$ -- random vector.
	 		\item for $k = 0, \dots, n - 1$: ($n$ is a large enough number of steps)
	 		\[
	 			\bv_{k+1} = \frac{\bW^T \bu_{k}}{\| \bW^T \bu_{k} \|}, \quad \bu_{k+1} = \frac{\bW \bv_{k+1}}{\| \bW \bv_{k+1} \|}.
	 		\]
	 		\item approximate the spectral norm
	 		\[
	 			\| \bW \|_2 = \lambda_{\text{max}}(\bW^T \bW) \approx \bu_{n}^T \bW \bv_{n}.
	 		\]
	 	\end{itemize}
	 \end{block}

	\myfootnotewithlink{https://arxiv.org/abs/1802.05957}{Miyato T. et al. Spectral Normalization for Generative Adversarial Networks, 2018}
\end{frame}
%=======
\begin{frame}{Spectral Normalization GAN}
	\begin{figure}
		\centering
		\includegraphics[width=0.85\linewidth]{figs/sngan_pseudocode}
	\end{figure}
	\begin{figure}
		\centering
		\includegraphics[width=0.85\linewidth]{figs/sngan_fids}
	\end{figure}

	\myfootnotewithlink{https://arxiv.org/abs/1802.05957}{Miyato T. et al. Spectral Normalization for Generative Adversarial Networks, 2018}
\end{frame}
%=======
\begin{frame}{Divergences}
	\begin{itemize}
		\item Forward KL divergence in maximum likelihood estimation.
		\item Reverse KL in variational inference.
		\item JS divergence in standard GAN.
		\item Wasserstein distance in WGAN.
	\end{itemize}
	\begin{block}{What is a divergence?}
		Let $\cS$ be the set of all possible probability distributions. Then $D: \cS \times \cS \rightarrow \bbR$ is a divergence if 
		\begin{itemize}
			\item $D(\pi || p) \geq 0$ for all $\pi, p \in \cS$;
			\item $D(\pi || p) = 0$ if and only if $\pi \equiv p$.
		\end{itemize}
	\end{block}
	\begin{block}{General divergence minimization task}
		\vspace{-0.3cm}
		\[
			\min_p D(\pi || p)
		\]
		\vspace{-0.7cm}
	\end{block}
	\begin{block}{Chalenge}
		We do not know the real distribution $\pi(\bx)$!
	\end{block}
\end{frame}
%=======
\begin{frame}{f-divergence family}
	
	\begin{block}{f-divergence}
		\vspace{-0.3cm}
		\[
		D_f(\pi || p) = \bbE_{p(\bx)}  f\left( \frac{\pi(\bx)}{p(\bx)} \right)  = \int p(\bx) f\left( \frac{\pi(\bx)}{p(\bx)} \right) d \bx.
		\]
		Here $f: \bbR_+ \rightarrow \bbR$ is a convex, lower semicontinuous function satisfying $f(1) = 0$.
	\end{block}
	\begin{figure}
		\centering
		\includegraphics[width=\linewidth]{figs/f_divs}
	\end{figure}
	\myfootnotewithlink{https://arxiv.org/abs/1606.00709}{Nowozin S., Cseke B., Tomioka R. f-GAN: Training Generative Neural Samplers using Variational Divergence Minimization, 2016}
\end{frame}
%=======
\begin{frame}{f-divergence family}
	\vspace{-0.2cm}
	\begin{block}{Fenchel conjugate}
		\vspace{-0.7cm}
		\[
		f^*(t) = \sup_{u \in \text{dom}_f} \left( ut - f(u) \right), \quad f(u) = \sup_{t \in \text{dom}_{f^*}} \left( ut - f^*(t) \right)
		\]
		\vspace{-0.5cm}
	\end{block}
	\textbf{Important property:} $ f^{**} = f$ for convex $f$.
	\begin{block}{f-divergence}
		\vspace{-0.8cm}
		\begin{multline*}
			D_f(\pi || p) = \bbE_{p(\bx)}  f\left( \frac{\pi(\bx)}{p(\bx)} \right)  = \int p(\bx) f\left( \frac{\pi(\bx)}{p(\bx)} \right) d \bx = \\ = \int p(\bx) \sup_{t \in \text{dom}_{f^*}} \left( \frac{\pi(\bx)}{p(\bx)} t - f^*(t) \right) d \bx = \\ 
			= \int \sup_{t \in \text{dom}_{f^*}} \left( \pi(\bx)t - p(\bx) f^*(t) \right) d \bx .
		\end{multline*}
		\vspace{-0.6cm}
	\end{block}
	Here we seek value of $t$, which gives us maximum value of $ \pi(\bx)t - p(\bx) f^*(t)$, for each data point $\bx$.
	\myfootnotewithlink{https://arxiv.org/abs/1606.00709}{Nowozin S., Cseke B., Tomioka R. f-GAN: Training Generative Neural Samplers using Variational Divergence Minimization, 2016}
\end{frame}
%=======
\begin{frame}{f-divergence family}
	\vspace{-0.4cm}
	\begin{block}{f-divergence}
		\vspace{-0.3cm}
		\[
		D_f(\pi || p) = \bbE_{p(\bx)}  f\left( \frac{\pi(\bx)}{p(\bx)} \right)  = \int p(\bx) f\left( \frac{\pi(\bx)}{p(\bx)} \right) d \bx.
		\]
		\vspace{-0.4cm}
	\end{block}
	\begin{block}{Variational f-divergence estimation}
		\vspace{-0.8cm}
		\begin{multline*}
			D_f(\pi || p)  = \int \sup_{t \in \text{dom}_{f^*}} \left( \pi(\bx)t - p(\bx) f^*(t) \right) d \bx \geq \\
			 \geq \sup_{T \in \cT} \int \left( \pi(\bx)T(\bx) - p(\bx) f^*(T(\bx)) \right) d \bx = \\
			 = \sup_{T \in \cT} \left[\bbE_{\pi}T(\bx) -  \bbE_p f^*(T(\bx)) \right]
		\end{multline*}
	\vspace{-0.6cm}
	\end{block}
	This is a lower bound because of Jensen-Shannon inequality and restricted class of functions $\cT: \cX \rightarrow \bbR$.
	
	\myfootnotewithlink{https://arxiv.org/abs/1606.00709}{Nowozin S., Cseke B., Tomioka R. f-GAN: Training Generative Neural Samplers using Variational Divergence Minimization, 2016}
\end{frame}
%=======
\begin{frame}{f-divergence family}
	\begin{block}{Variational divergence estimation}
		\[
			D_f(\pi || p) \geq \sup_{T \in \cT} \left[\bbE_{\pi}T(\bx) -  \bbE_p f^*(T(\bx)) \right]
		\]
		The lower bound is tight for $T^*(\bx) = f'\left( \frac{\pi(\bx)}{p(\bx)} \right)$.
	\end{block}
	\begin{block}{Example (JSD)}
		\begin{itemize}
			\item Let define function $f$ and its conjugate $f^*$
			\[ 
				f(u) = u \log u - (u + 1) \log (u + 1), \quad f^*(t) = - \log (1 - e^t).
			\]
			\item Let reparametrize $T(\bx) = \log D(\bx)$.
		\end{itemize}
		\vspace{-0.7cm}
	\end{block}
	\[
		\min_{G} \max_D V(G, D) = \min_{G} \max_D \left[ \bbE_{\pi(\bx)} \log D(\bx) + \bbE_{p(\bz)} \log (1 - D(G(\bz))) \right]
	\]

	\myfootnotewithlink{https://arxiv.org/abs/1606.00709}{Nowozin S., Cseke B., Tomioka R. f-GAN: Training Generative Neural Samplers using Variational Divergence Minimization, 2016}
\end{frame}
%=======
\begin{frame}{f-divergence family}
	\begin{block}{Variational divergence estimation}
		\[
			D_f(\pi || p) \geq \sup_{T \in \cT} \left[\bbE_{\pi}T(\bx) -  \bbE_p f^*(T(\bx)) \right]
		\]
	\end{block}
	\textbf{Note:} To evaluate lower bound we only need samples from $\pi(\bx)$ and $p(\bx)$. Hence, we could fit implicit generative model.
	\begin{figure}
		\centering
		\includegraphics[width=1.0\linewidth]{figs/f_div_results}
	\end{figure}

	\myfootnotewithlink{https://arxiv.org/abs/1606.00709}{Nowozin S., Cseke B., Tomioka R. f-GAN: Training Generative Neural Samplers using Variational Divergence Minimization, 2016}
\end{frame}
%=======
\begin{frame}{Evaluation of likelihood-free models}
	How to evaluate generative models?
	\begin{block}{Likelihood-based models}
		\begin{itemize}
			\item Split data to train/val/test.
			\item Fit model on the train part.
			\item Tune hyperparameters on the validation part.
			\item Evaluate generalization by reporting likelihoods on the test set.
		\end{itemize}
	\end{block}
	\begin{block}{Not all models have tractable likelihoods}
		\begin{itemize}
			\item VAE: compare ELBO values.
			\item GAN: ???
		\end{itemize}
	\end{block}
\end{frame}
%=======
\begin{frame}{Evaluation of likelihood-free models}
	Let take some pretrained image classification model to get the conditional label distribution $p(y | \bx)$ (e.g. ImageNet classifier).
	\begin{block}{What do we want from samples?}
		\begin{itemize}
			\item \textbf{Sharpness}
			\begin{figure}
				\centering
				\includegraphics[width=0.9\linewidth]{figs/sharpness}
			\end{figure}
			The conditional distribution $p(y | \bx)$ should have low entropy (each image $\bx$ should have distinctly recognizable object).
			\item \textbf{Diversity}
			\begin{figure}
				\centering
				\includegraphics[width=0.9\linewidth]{figs/diversity}
			\end{figure}
			The marginal distribution $p(y) = \int p(y | \bx) p(\bx) d \bx$ should have high entropy (there should be as many classes generated as possible).
		\end{itemize}
	\end{block}
	\myfootnotewithlink{https://deepgenerativemodels.github.io}{image credit: https://deepgenerativemodels.github.io}
\end{frame}
%=======
\begin{frame}{Evaluation of likelihood-free models}
	\begin{block}{What do we want from samples?}
		\begin{itemize}
			\item \textbf{Sharpness.}
			The conditional distribution $p(y | \bx)$ should have low entropy (each image $\bx$ should have distinctly recognizable object).
			\item \textbf{Diversity.}
			The marginal distribution $p(y) = \int p(y | \bx) p(\bx) d \bx$ should have high entropy (there should be as many classes generated as possible).
		\end{itemize}
	\end{block}
	\begin{figure}
		\centering
		\includegraphics[width=1.0\linewidth]{figs/is_toy}
	\end{figure}
	\myfootnotewithlink{https://medium.com/octavian-ai/a-simple-explanation-of-the-inception-score-372dff6a8c7a}{image credit: https://medium.com/octavian-ai/a-simple-explanation-of-the-inception-score-372dff6a8c7a}
\end{frame}
%=======
\begin{frame}{Evaluation of likelihood-free models}
		\begin{block}{What do we want from samples?}
		\begin{itemize}
			\item Sharpness $\Rightarrow$ low $H(y | \bx) = - \sum_{y} \int_{\bx} p(y, \bx) \log p(y | \bx) d\bx$.
			\item Diversity $\Rightarrow$ high $H(y)  = - \sum_{y} p(y) \log p(y)$.
		\end{itemize}
	\end{block}
	\begin{block}{Inception Score}
		\vspace{-0.3cm}
		\footnotesize
		\begin{align*}
			IS &= \exp(H(y) - H(y | \bx)) \\ 
			&= \exp \left( - \sum_{y} p(y) \log p(y) + \sum_{y} \int_{\bx} p(y, \bx) \log p(y | \bx) d\bx\right) \\
			&= \exp \left( \sum_{y} \int_{\bx} p(y, \bx) \log \frac{p(y | \bx)}{p(y)} d\bx\right) \\ 
			&= \exp \left( \bbE_{\bx} \sum_{y} p(y | \bx) \log \frac{p(y | \bx)}{p(y)} \right) = \exp \left( \bbE_{\bx} KL(p(y | \bx) || p(y)) \right)
		\end{align*}
	\end{block}
	\myfootnotewithlink{https://arxiv.org/abs/1606.03498}{Salimans T. et al. Improved Techniques for Training GANs, 2016}
\end{frame}
%=======
\begin{frame}{Evaluation of likelihood-free models}
	\begin{block}{Theorem (informal)}
		If $\pi(\bx)$ and $p(\bx | \btheta)$ has moment generation functions then
		\vspace{-0.1cm}
		\[
			\pi(\bx) = p(\bx | \btheta) \, \Leftrightarrow \, \bbE_{\pi} \bx^k = \bbE_{p} \bx^k, \quad \forall k \geq 1.
		\]
		\vspace{-0.7cm}
	\end{block}
	This is intractable to calculate all moments.
	\begin{block}{Frechet Inception Distance}
		\vspace{-0.3cm}
		\[
			FID (\pi, p) = \| \mathbf{m}_{\pi} - \mathbf{m}_{p}\|_2^2 + \text{Tr} \left( \bSigma_{\pi} + \bSigma_p - 2 \sqrt{\bSigma_{\pi} \bSigma_p} \right)
		\]
		\vspace{-0.5cm}
	\end{block}
	\begin{itemize}
		\item Representations are outputs of intermediate layer from pretrained classification model.
		\item $\mathbf{m}_{\pi}$, $\bSigma_{\pi} $ are mean vector and covariance matrix of feature representations for real samples from $\pi(\bx)$
		\item $\mathbf{m}_{p}$, $\bSigma_p$ are mean vector and covariance matrix of feature representations for generated samples from $p(\bx | \btheta)$.
	\end{itemize} 

	\myfootnotewithlink{https://arxiv.org/abs/1706.08500}{Heusel M. et al. GANs Trained by a Two Time-Scale Update Rule Converge to a Local Nash Equilibrium, 2017}
\end{frame}
%=======
\begin{frame}{Evaluation of likelihood-free models}
	\begin{figure}
		\centering
		\includegraphics[width=0.9\linewidth]{figs/fid_results}
	\end{figure}
	
	\myfootnotewithlink{https://arxiv.org/abs/1706.08500}{Heusel M. et al. GANs Trained by a Two Time-Scale Update Rule Converge to a Local Nash Equilibrium, 2017}
\end{frame}
%=======
\begin{frame}{Limitations}
	\vspace{-0.5cm}
	\begin{block}{Inception Score}
		\vspace{-0.5cm}
		\[
			IS =  \exp \left( \bbE_{\bx} KL(p(y | \bx) || p(y)) \right)
		\]
		\vspace{-0.7cm}
	\end{block}
	\begin{itemize}
		\item If generator produces images with a different set of labels from the classifier training set, IS will be low.
		\item If generator produces one image per class, the IS will be perfect (there is no measure of intra-class diversity).
	\end{itemize}
	\begin{block}{Frechet Inception Distance}
		\vspace{-0.4cm}
		\[
			FID = \| \mathbf{m}_{\pi} - \mathbf{m}_{p}\|_2^2 + \text{Tr} \left( \bSigma_{\pi} + \bSigma_p - 2 \sqrt{\bSigma_{\pi} \bSigma_p} \right)
		\]
		\vspace{-0.7cm}
	\end{block}
	\begin{itemize}
		\item Needs a large sample size for evaluation.
		\item Calculation of FID is slow.
		\item Estimates only two sample moments.
	\end{itemize}
	Both scores depend on the pretrained classifier $p(y | \bx)$.

	\myfootnote{\href{https://arxiv.org/abs/1801.01973}{Barratt S., Sharma R. A Note on the Inception Score, 2018} \\
	\href{https://arxiv.org/abs/1706.08500}{Heusel M. et al. GANs Trained by a Two Time-Scale Update Rule Converge to a Local Nash Equilibrium, 2017}}
\end{frame}
%=======
\begin{frame}{Summary}
	\begin{itemize}
		\item Spectral normalization is a weight normalization technique to enforce Lipshitzness, which is helpful for generator and discriminator.
		\vfill
		\item f-divergence family is a unified framework for divergence minimization, which uses variational approximation. Standard GAN is a special case of it.
		\vfill
		\item Inception Score and Frechet Inception Distance are the common metrics for GAN evaluation, but both of them have drawbacks.
	\end{itemize}
\end{frame}
\end{document} 