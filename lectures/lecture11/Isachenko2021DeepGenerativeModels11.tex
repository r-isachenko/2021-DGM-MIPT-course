\documentclass{beamer}
\usepackage[utf8]{inputenc}
\usepackage{graphicx, epsfig}
\usepackage{amsmath,mathrsfs,amsfonts,amssymb}
%\usepackage{subfig}
\usepackage{floatflt}
\usepackage{epic,ecltree}
\usepackage{mathtext}
\usepackage{fancybox}
\usepackage{fancyhdr}
\usepackage{multirow}
\usepackage{enumerate}
\usepackage{epstopdf}
\usepackage{multicol}
\usepackage{algorithm}
\usepackage[noend]{algorithmic}
\usepackage{tikz}
\usepackage{blindtext}
\usetheme{default}%{Singapore}%{Warsaw}%{Warsaw}%{Darmstadt}
\usecolortheme{default}
\setbeamerfont{title}{size=\Huge}
\setbeamertemplate{footline}[page number]{}


\makeatletter
\newcommand\HUGE{\@setfontsize\Huge{35}{40}}
\makeatother    

\setbeamerfont{title}{size=\HUGE}
\beamertemplatenavigationsymbolsempty

\input{../utils/newcommands}
\input{../utils/title}

\newcommand\myfootnote[1]{%
  \tikz[remember picture,overlay]
  \draw (current page.south west) +(1in + \oddsidemargin,0.5em)
  node[anchor=south west,inner sep=0pt]{\parbox{\textwidth}{%
      \rlap{\rule{10em}{0.4pt}}\raggedright\scriptsize \textit{#1}}};}

\newcommand\myfootnotewithlink[2]{%
  \tikz[remember picture,overlay]
  \draw (current page.south west) +(1in + \oddsidemargin,0.5em)
  node[anchor=south west,inner sep=0pt]{\parbox{\textwidth}{%
      \rlap{\rule{10em}{0.4pt}}\raggedright\scriptsize\href{#1}{\textit{#2}}}};}
\createdgmtitle{11}
%--------------------------------------------------------------------------------
\begin{document}
%--------------------------------------------------------------------------------
\begin{frame}[noframenumbering,plain]
%\thispagestyle{empty}
\titlepage
\end{frame}
%=======
\begin{frame}{Recap of previous lecture}
	\begin{block}{Vanilla GAN}
		\vspace{-0.5cm}
		\[
			\min_{G} \max_D V(G, D) = \min_{G} \max_D \left[ \bbE_{\pi(\bx)} \log D(\bx) + \bbE_{p(\bz)} \log (1 - D(G(\bz))) \right]
		\]
		\vspace{-0.7cm}
	\end{block}
	\begin{block}{Main problems}
		\begin{itemize}
			\item Vanishing gradients (non-saturating GAN does not suffer of it);
			\item Mode collapse (caused by behaviour of Jensen-Shannon divergence).
		\end{itemize}
	\end{block}
	\vspace{-0.1cm}
	\begin{block}{Informal theoretical results}
		Distribution of real images $\pi(\bx)$ and distribution of generated images $p(\bx | \btheta)$ are low-dimensional and have disjoint supports. In this case
		\vspace{-0.3cm}
		\[
			KL(\pi || p) = KL(p || \pi) = \infty, \quad JSD(\pi || p) = \log 2
		\]
		\end{block}
		\myfootnote{\href{https://arxiv.org/abs/1406.2661}{Goodfellow I. J. et al. Generative Adversarial Networks, 2014} \\
		\href{https://arxiv.org/abs/1701.04862}{Arjovsky M., Bottou L. Towards Principled Methods for Training Generative Adversarial Networks, 2017}}
\end{frame}
%=======
\begin{frame}{Recap of previous lecture}
		\begin{block}{Wasserstein distance}
			\vspace{-0.5cm}
			\[
				W(\pi, p) = \inf_{\gamma \in \Gamma(\pi, p)} \bbE_{(\bx, \by) \sim \gamma} \| \bx - \by \| =  \inf_{\gamma \in \Gamma(\pi, p)} \int \| \bx - \by \| \gamma (\bx, \by) d \bx d \by
			\]
			\vspace{-0.5cm}
			\begin{itemize}
				\item $\Gamma(\pi, p)$ -- the set of all joint distributions $\Gamma (\bx, \by)$ with marginals $\pi$ and $p$ ($\int \gamma(\bx, \by) d \bx = p(\by)$, $\int \gamma(\bx, \by) d \by = \pi(\bx)$)
				\item $\gamma(\bx, \by)$ -- transportation plan (the amount of "dirt" that should be transported from point $\bx$ to point $\by$).
				\item $\gamma(\bx, \by)$ -- the amount, $\|\bx - \by \|$-- the distance.
			\end{itemize}
		\end{block}
		\begin{block}{Kantorovich-Rubinstein duality}
			\vspace{-0.2cm}
			\[
				W(\pi || p) = \frac{1}{K} \max_{\| f \|_L \leq K} \left[ \bbE_{\pi(\bx)} f(\bx)  - \bbE_{p(\bx)} f(\bx)\right],
			\]
			where $\| f \|_L \leq K$ are $K-$Lipschitz continuous functions ($f: \cX \rightarrow \bbR$).
		\end{block}
		\myfootnotewithlink{https://arxiv.org/abs/1701.07875}{Arjovsky M., Chintala S., Bottou L. Wasserstein GAN, 2017}
\end{frame}
%=======
\begin{frame}{Recap of previous lecture}
		\begin{block}{Vanilla GAN objective}
			\vspace{-0.2cm}
			\[
				\min_{G} \max_D \bbE_{\pi(\bx)} \log D(\bx) + \bbE_{p(\bz)} \log (1 - D(G(\bz)))
			\]
			\vspace{-0.2cm}
		\end{block}
		\begin{block}{WGAN objective}
			\vspace{-0.6cm}
			\[
			\min_{G} W(\pi || p) = \min_{G} \max_{\bphi \in \boldsymbol{\Phi}} \left[ \bbE_{\pi(\bx)} f(\bx, \bphi)  - \bbE_{p(\bz)} f(G(\bz), \bphi )\right].
			\]
		\end{block}
		\begin{itemize}
			\item Discriminator $D$ is similar to the function $f$, but not the same (it is not a classifier anymore). In the WGAN model, function~$f$ is usually called $\textit{critic}$.
			\item "Weight clipping is a clearly terrible way to enforce a Lipschitz constraint". If the clipping parameter is large, it is hard to train the critic till optimality. If the clipping parameter is too small, it could lead to vanishing gradients.
		\end{itemize}
		\myfootnotewithlink{https://arxiv.org/abs/1701.07875}{Arjovsky M., Chintala S., Bottou L. Wasserstein GAN, 2017}
		
\end{frame}
%=======
\begin{frame}{Recap of previous lecture}
	\begin{block}{Kantorovich-Rubinstein duality}
		\vspace{-0.2cm}
		\[
			W(\pi || p) = \frac{1}{K} \max_{\| f \|_L \leq K} \left[ \bbE_{\pi(\bx)} f(\bx)  - \bbE_{p(\bx)} f(\bx)\right],
		\]
		where $\| f \|_L \leq K$ are $K-$Lipschitz continuous functions ($f: \cX \rightarrow \bbR$).
	\end{block}
	\begin{block}{Gradient penalty}
		\vspace{-0.3cm}
		\[
			W(\pi || p) = \underbrace{\bbE_{\pi(\bx)} f(\bx)  - \bbE_{p(\bx)} f(\bx)}_{\text{original critic loss}} + \lambda \underbrace{\bbE_{U[0, 1]} \left[ \left( \| \nabla_{\hat{\bx}} f(\hat{\bx}) \|_2 - 1 \right) ^ 2\right]}_{\text{gradient penalty}}.
		\]
		Samples $\hat{\bx}_t = t \bx + (1 - t) \by$ with $t \in [0, 1]$ are uniformly sampled along straight lines between pairs of points: $\bx$ from the data distribution $\pi(\bx)$ and $\by$ from the generator distribution $p(\bx | \btheta)$.
	\end{block}
	\myfootnote{
	\href{https://arxiv.org/abs/1701.07875}{Arjovsky M., Chintala S., Bottou L. Wasserstein GAN, 2017} \\
	\href{https://arxiv.org/abs/1704.00028}{Gulrajani I. et al. Improved Training of Wasserstein GANs, 2017} }
\end{frame}
%=======
\begin{frame}{Spectral Normalization GAN}
	How else could we enforce Lipschitzness?
	\begin{block}{Fact 1}
		Let denote by $\sigma(\bA)$ a spectral norm of matrix $\bA$.
		\[
			\sigma(\bA) = \max_{\bh \neq 0} \frac{\|\bA \bh\|_2}{\|\bh\|_2} = \max_{\|\bh\|_2 \leq 1} \| \bA \bh \|_2 = \lambda_{\text{max}}(\bA),
		\]
		where $\lambda_{\text{max}}(\bA)$ is the largest singular value of $\bA$. \\
		By definition, Lipschitz norm is 
		\[
			\| \mathbf{g} \|_L = \sup_\bx \sigma( \nabla \mathbf{g}(\bx))
		\]
	\end{block}
	\vspace{-0.5cm}
	\begin{block}{Fact 2}
		Lipschitz norm of superposition is bounded above by product of Lipschitz norms
		\vspace{-0.2cm}
		\[
			\| \mathbf{g}_1 \circ \mathbf{g}_2 \|_L \leq \| \mathbf{g}_1 \|_L \cdot \| \mathbf{g}_2\|_L
		\]
	\end{block}
	\myfootnotewithlink{https://arxiv.org/abs/1802.05957}{Miyato T. et al. Spectral Normalization for Generative Adversarial Networks, 2018}
\end{frame}
%=======
\begin{frame}{Spectral Normalization GAN}
	Let consider the critic $f(\bx, \bphi)$ of the following form:
	\[
		f(\bx, \bphi) = \bW_{K+1} a_K (\bW_K a_{K-1}(\dots a_1(\bW_1 \bx) \dots)).
	\]
	This feedforward network is a superposition of simple functions.
	\begin{itemize}
		\item $a_k$ is a pointwise nonlinearities. We assume that $\| a_k \|_L = 1$ (it holds for ReLU).
		\item $\mathbf{g}(\bx) = \bW \bx$ is a linear transformation ($\nabla \mathbf{g}(\bx) = \bW$).
		\[
			\| \mathbf{g} \|_L = \sup_\bx \sigma( \nabla \mathbf{g}(\bx)) = \sigma(\bW).
		\]
	\end{itemize}
	\vspace{-0.3cm}
	\begin{block}{Critic spectral norm}
		\vspace{-0.3cm}
		\[
			\| f \|_L \leq \| \bW_{K+1}\| \cdot \prod_{k=1}^K  \| a_k \|_L \cdot \| \bW_k \| = \prod_{k=1}^{K+1} \sigma(\bW_k).
		\]
		\vspace{-0.2cm}
	\end{block}
	If we replace the weights in the critic $f(\bx, \bphi)$ by $\bW^{SN}_k = \bW_k / \sigma(\bW_k)$, we will get $\| f\|_L \leq 1.$ \\
	
	\myfootnotewithlink{https://arxiv.org/abs/1802.05957}{Miyato T. et al. Spectral Normalization for Generative Adversarial Networks, 2018}
\end{frame}
%=======
\begin{frame}{Spectral Normalization GAN}
	How to compute $\sigma(\bW)$? \\
	 If we apply singular value decomposition to compute the $\sigma(\bW)$ at each round of the algorithm, the algorithm becomes intractable.
	 
	 \begin{block}{Power iteration}
	 	\begin{itemize}
	 		\item $\bu_0$ -- random vector.
	 		\item for $k = 0, \dots, n - 1$: ($n$ is a large enough number of steps)
	 		\[
	 			\bv_{k+1} = \frac{\bW^T \bu_{k}}{\| \bW^T \bu_{k} \|}, \quad \bu_{k+1} = \frac{\bW \bv_{k+1}}{\| \bW \bv_{k+1} \|}.
	 		\]
	 		\item approximate the spectral norm
	 		\[
	 			\sigma(\bW) \approx \bu_{n}^T \bW \bv_{n}.
	 		\]
	 	\end{itemize}
	 \end{block}

	\myfootnotewithlink{https://arxiv.org/abs/1802.05957}{Miyato T. et al. Spectral Normalization for Generative Adversarial Networks, 2018}
\end{frame}
%=======
\begin{frame}{Spectral Normalization GAN}
	\begin{figure}
		\centering
		\includegraphics[width=0.85\linewidth]{figs/sngan_pseudocode}
	\end{figure}
	\begin{figure}
		\centering
		\includegraphics[width=0.85\linewidth]{figs/sngan_fids}
	\end{figure}

	\myfootnotewithlink{https://arxiv.org/abs/1802.05957}{Miyato T. et al. Spectral Normalization for Generative Adversarial Networks, 2018}
\end{frame}
%=======
\begin{frame}{Divergences}
	\begin{block}{What do we have?}
		\begin{itemize}
			\item Forward KL divergence in maximum likelihood estimation
			\item Reverse KL in variational inference
			\item JS divergence in  vanilla gan
			\item Wasserstein distance in WGAN
		\end{itemize}
	\end{block}
	\begin{block}{What is a divergence?}
		Let $\cS$ be the set of all possible probability distributions. Then $D: \cS \times \cS \rightarrow \bbR$ is a divergence if 
		\begin{itemize}
			\item $D(\pi || p) \geq 0$ for all $\pi, p \in \cS$;
			\item $D(\pi || p) = 0$ if and only if $\pi \equiv p$.
		\end{itemize}
	\end{block}
	\begin{block}{General divergence minimization task}
		\vspace{-0.3cm}
		\[
			\min_p D(\pi || p)
		\]
		\vspace{-0.5cm}
	\end{block}
\end{frame}
%=======
\begin{frame}{f-divergence family}
	
	\begin{block}{f-divergence}
		\vspace{-0.3cm}
		\[
		D_f(\pi || p) = \bbE_{p(\bx)}  f\left( \frac{\pi(\bx)}{p(\bx)} \right)  = \int p(\bx) f\left( \frac{\pi(\bx)}{p(\bx)} \right) d \bx.
		\]
		Here $f: \bbR_+ \rightarrow \bbR$ is a convex, lower semicontinuous function satisfying $f(1) = 0$.
	\end{block}
	\begin{figure}
		\centering
		\includegraphics[width=\linewidth]{figs/f_divs}
	\end{figure}
	\myfootnotewithlink{https://arxiv.org/abs/1606.00709}{Nowozin S., Cseke B., Tomioka R. f-GAN: Training Generative Neural Samplers using Variational Divergence Minimization, 2016}
\end{frame}
%=======
\begin{frame}{f-divergence family}
	\vspace{-0.2cm}
	\begin{block}{Fenchel conjugate}
		\vspace{-0.7cm}
		\[
		f^*(t) = \sup_{u \in \text{dom}_f} \left( ut - f(u) \right), \quad f(u) = \sup_{t \in \text{dom}_{f^*}} \left( ut - f^*(t) \right)
		\]
		\vspace{-0.5cm}
	\end{block}
	\textbf{Important property:} $ f^{**} = f$ for convex $f$.
	\begin{block}{f-divergence}
		\vspace{-0.8cm}
		\begin{multline*}
			D_f(\pi || p) = \bbE_{p(\bx)}  f\left( \frac{\pi(\bx)}{p(\bx)} \right)  = \int p(\bx) f\left( \frac{\pi(\bx)}{p(\bx)} \right) d \bx = \\ = \int p(\bx) \sup_{t \in \text{dom}_{f^*}} \left( \frac{\pi(\bx)}{p(\bx)} t - f^*(t) \right) d \bx = \\ 
			= \int \sup_{t \in \text{dom}_{f^*}} \left( \pi(\bx)t - p(\bx) f^*(t) \right) d \bx .
		\end{multline*}
		\vspace{-0.6cm}
	\end{block}
	Here we seek value of $t$, which gives us maximum value of $ \pi(\bx)t - p(\bx) f^*(t)$, for each data point $\bx$.
	\myfootnotewithlink{https://arxiv.org/abs/1606.00709}{Nowozin S., Cseke B., Tomioka R. f-GAN: Training Generative Neural Samplers using Variational Divergence Minimization, 2016}
\end{frame}
%=======
\begin{frame}{f-divergence family}
	\vspace{-0.4cm}
	\begin{block}{f-divergence}
		\vspace{-0.3cm}
		\[
		D_f(\pi || p) = \bbE_{p(\bx)}  f\left( \frac{\pi(\bx)}{p(\bx)} \right)  = \int p(\bx) f\left( \frac{\pi(\bx)}{p(\bx)} \right) d \bx.
		\]
		\vspace{-0.4cm}
	\end{block}
	\begin{block}{Variational f-divergence estimation}
		\vspace{-0.8cm}
		\begin{multline*}
			D_f(\pi || p)  = \int \sup_{t \in \text{dom}_{f^*}} \left( \pi(\bx)t - p(\bx) f^*(t) \right) d \bx \geq \\
			 \geq \sup_{T \in \cT} \int \left( \pi(\bx)T(\bx) - p(\bx) f^*(T(\bx)) \right) d \bx = \\
			 = \sup_{T \in \cT} \left[\bbE_{\pi}T(\bx) -  \bbE_p f^*(T(\bx)) \right]
		\end{multline*}
	\vspace{-0.6cm}
	\end{block}
	This is a lower bound because of Jensen-Shannon inequality and restricted class of functions $\cT: \cX \rightarrow \bbR$.
	
	\textbf{Note:} To evaluate lower bound we only need samples from $\pi(\bx)$ and $p(\bx)$. Hence, we could fit implicit generative model.
	\myfootnotewithlink{https://arxiv.org/abs/1606.00709}{Nowozin S., Cseke B., Tomioka R. f-GAN: Training Generative Neural Samplers using Variational Divergence Minimization, 2016}
\end{frame}
%=======
\begin{frame}{f-divergence family}
	\begin{block}{Variational divergence estimation}
		\[
			D_f(\pi || p) \geq \sup_{T \in \cT} \left[\bbE_{\pi}T(\bx) -  \bbE_p f^*(T(\bx)) \right]
		\]
		The lower bound is tight for $T^*(\bx) = f'\left( \frac{\pi(\bx)}{p(\bx)} \right)$.
	\end{block}
	\begin{figure}
		\centering
		\includegraphics[width=1.0\linewidth]{figs/f_div_results}
	\end{figure}

	\myfootnotewithlink{https://arxiv.org/abs/1606.00709}{Nowozin S., Cseke B., Tomioka R. f-GAN: Training Generative Neural Samplers using Variational Divergence Minimization, 2016}
\end{frame}
%=======
\begin{frame}{Evaluation of likelihood-free models}
	How to evaluate generative models?
	\begin{block}{Likelihood-based models}
		\begin{itemize}
			\item Split data to train/val/test.
			\item Fit model on the train part.
			\item Tune hyperparameters on the validation part.
			\item Evaluate generalization by reporting likelihoods on the test set.
		\end{itemize}
	\end{block}
	\begin{block}{Not all models have tractable likelihoods}
		\begin{itemize}
			\item VAE: compare ELBO values.
			\item GAN: ???
		\end{itemize}
	\end{block}
\end{frame}
%=======
\begin{frame}{Evaluation of likelihood-free models}
	Let take some pretrained image classification model to get the conditional label distribution $p(y | \bx)$ (e.g. ImageNet classifier).
	\begin{block}{What do we want from samples?}
		\begin{itemize}
			\item \textbf{Sharpness}
			\begin{figure}
				\centering
				\includegraphics[width=0.9\linewidth]{figs/sharpness}
			\end{figure}
			The conditional distribution $p(y | \bx)$ should have low entropy (each image $\bx$ should have distinctly recognizable object).
			\item \textbf{Diversity}
			\begin{figure}
				\centering
				\includegraphics[width=0.9\linewidth]{figs/diversity}
			\end{figure}
			The marginal distribution $p(y) = \int p(y | \bx) p(\bx) d \bx$ should have high entropy (there should be as many classes generated as possible).
		\end{itemize}
	\end{block}
	\myfootnotewithlink{https://deepgenerativemodels.github.io}{image credit: https://deepgenerativemodels.github.io}
\end{frame}
%=======
\begin{frame}{Evaluation of likelihood-free models}
	\begin{block}{What do we want from samples?}
		\begin{itemize}
			\item \textbf{Sharpness.}
			The conditional distribution $p(y | \bx)$ should have low entropy (each image $\bx$ should have distinctly recognizable object).
			\item \textbf{Diversity.}
			The marginal distribution $p(y) = \int p(y | \bx) p(\bx) d \bx$ should have high entropy (there should be as many classes generated as possible).
		\end{itemize}
	\end{block}
	\begin{figure}
		\centering
		\includegraphics[width=1.0\linewidth]{figs/is_toy}
	\end{figure}
	\myfootnotewithlink{https://medium.com/octavian-ai/a-simple-explanation-of-the-inception-score-372dff6a8c7a}{image credit: https://medium.com/octavian-ai/a-simple-explanation-of-the-inception-score-372dff6a8c7a}
\end{frame}
%=======
\begin{frame}{Evaluation of likelihood-free models}
		\begin{block}{What do we want from samples?}
		\begin{itemize}
			\item Sharpness $\Rightarrow$ low $H(y | \bx) = - \sum_{y} \int_{\bx} p(y, \bx) \log p(y | \bx) d\bx$.
			\item Diversity $\Rightarrow$ high $H(y)  = - \sum_{y} p(y) \log p(y)$.
		\end{itemize}
	\end{block}
	\begin{block}{Inception Score}
		\vspace{-0.3cm}
		\footnotesize
		\begin{align*}
			IS &= \exp(H(y) - H(y | \bx)) \\ 
			&= \exp \left( - \sum_{y} p(y) \log p(y) + \sum_{y} \int_{\bx} p(y, \bx) \log p(y | \bx) d\bx\right) \\
			&= \exp \left( \sum_{y} \int_{\bx} p(y, \bx) \log \frac{p(y | \bx)}{p(y)} d\bx\right) \\ 
			&= \exp \left( \bbE_{\bx} \sum_{y} p(y | \bx) \log \frac{p(y | \bx)}{p(y)} \right) = \exp \left( \bbE_{\bx} KL(p(y | \bx) || p(y)) \right)
		\end{align*}
	\end{block}
	\myfootnotewithlink{https://arxiv.org/abs/1606.03498}{Salimans T. et al. Improved Techniques for Training GANs, 2016}
\end{frame}
%=======
\begin{frame}{Evaluation of likelihood-free models}
	\begin{block}{Inception Score}
		\vspace{-0.1cm}
		\[
			IS =  \exp \left( \bbE_{\bx} KL(p(y | \bx) || p(y)) \right)
		\]
		\vspace{-0.1cm}
	\end{block}
	\begin{block}{IS limitations}
		\begin{itemize}
			\item Inception score depends on the quality of the pretrained classifier $p(y | \bx)$.
			\item If generator produces images with a different set of labels from the classifier training set, IS will be low.
			\item If the generator produces one image per class, the IS will be perfect (there is no measure of intra-class diversity).
			\item IS only require samples from the generator and do not take into account the desired data distribution $\pi(\bx)$ directly (only implicitly via a classifier).
		\end{itemize}
	\end{block}
	\myfootnotewithlink{https://arxiv.org/abs/1801.01973}{Barratt S., Sharma R. A Note on the Inception Score, 2018}
\end{frame}
%=======
\begin{frame}{Evaluation of likelihood-free models}
	\begin{block}{Theorem (informal)}
		If $\pi(\bx)$ and $p(\bx | \btheta)$ has moment generation functions then
		\vspace{-0.1cm}
		\[
			\pi(\bx) = p(\bx | \btheta) \, \Leftrightarrow \, \bbE_{\pi} \bx^k = \bbE_{p} \bx^k, \quad \forall k \geq 1.
		\]
		\vspace{-0.7cm}
	\end{block}
	This is intractable to calculate all moments.
	\begin{block}{Frechet Inception Distance}
		\vspace{-0.3cm}
		\[
			D^2 (\pi, p) = \| \mathbf{m}_{\pi} - \mathbf{m}_{p}\|_2^2 + \text{Tr} \left( \bC_{\pi} + \bC_p - 2 \sqrt{\bC_{\pi} \bC_p} \right)
		\]
		\vspace{-0.5cm}
	\end{block}
	\begin{itemize}
		\item Representations are outputs of intermediate layer from pretrained classification model.
		\item $\mathbf{m}_{\pi}$, $\bC_{\pi} $ are mean vector and covariance matrix of feature representations for real samples from $\pi(\bx)$
		\item $\mathbf{m}_{p}$, $\bC_p$ are mean vector and covariance matrix of feature representations for generated samples from $p(\bx | \btheta)$.
	\end{itemize} 

	\myfootnotewithlink{https://arxiv.org/abs/1706.08500}{Heusel M. et al. GANs Trained by a Two Time-Scale Update Rule Converge to a Local Nash Equilibrium, 2017}
\end{frame}
%=======
\begin{frame}{Evaluation of likelihood-free models}
	\begin{figure}
		\centering
		\includegraphics[width=0.9\linewidth]{figs/fid_results}
	\end{figure}
	
	\myfootnotewithlink{https://arxiv.org/abs/1706.08500}{Heusel M. et al. GANs Trained by a Two Time-Scale Update Rule Converge to a Local Nash Equilibrium, 2017}
\end{frame}
%=======
\begin{frame}{Evaluation of likelihood-free models}
	\begin{block}{Frechet Inception Distance}
		\vspace{-0.1cm}
		\[
		D^2 (\pi, p) = \| \mathbf{m}_{\pi} - \mathbf{m}_{p}\|_2^2 + \text{Tr} \left( \bC_{\pi} + \bC_p - 2 \sqrt{\bC_{\pi} \bC_p} \right)
		\]
	\end{block}
	\begin{block}{FID limitations}
		\begin{itemize}
			\item FID depends on the pretrained classification model.
			\item FID needs a large samples  size for evaluation.
			\item Calculation of FID is slow.
			\item FID estimates only two sample moments.
		\end{itemize}
	\end{block}

	\myfootnotewithlink{https://arxiv.org/abs/1706.08500}{Heusel M. et al. GANs Trained by a Two Time-Scale Update Rule Converge to a Local Nash Equilibrium, 2017}
\end{frame}
%=======
\begin{frame}{Precision-Recall for Generative Models}
	\begin{block}{What do we want from samples}
		\begin{itemize}
			\item \textbf{Sharpness}: generated samples should be of high quality.
			\item \textbf{Diversity}: their variation should match that observed in the training set.
		\end{itemize}
	\end{block}
	\vspace{-0.5cm}
	\begin{figure}
		\includegraphics[width=0.95\linewidth]{figs/pr_curve}
	\end{figure}
	\vspace{-0.3cm}
	\begin{itemize}
		\item \textbf{Precision} denotes the fraction of generated images that are realistic.
		\item \textbf{Recall} measures the fraction of the training data manifold covered by the generator.
	\end{itemize}
	\myfootnotewithlink{https://arxiv.org/abs/1904.06991}{Kynkäänniemi T. et al. Improved precision and recall metric for assessing generative models, 2019}
\end{frame}
%=======
\begin{frame}{Precision-Recall for Generative Models}
	\begin{itemize}
		\item $\cS_{\pi} = \{\bx_i\}_{i=1}^{n} \sim \pi(\bx)$ -- real samples;
		\item $\cS_{p} = \{\bx_i\}_{i=1}^{n} \sim p(\bx | \btheta)$ -- generated samples.
	\end{itemize}
	Embed samples using pretrained classifier network (as previously):
	\[
		\cG_{\pi} = \{\mathbf{g}_i\}_{i=1}^n, \quad \cG_{p} = \{\mathbf{g}_i\}_{i=1}^n.
	\]
	Define binary function:
	\[
		f(\mathbf{g}, \cG) = 
		\begin{cases}
			1, \text{if exists } \mathbf{g}' \in \cG: \| \mathbf{g}  - \mathbf{g}'\|_2 \leq \| \mathbf{g} - \text{NN}_k(\mathbf{g}, \cG)\|_2; \\
			0, \text{otherwise.}
		\end{cases}
	\]
	\[
		\text{Precision} (\cG_{\pi}, \cG_{p}) = \frac{1}{n} \sum_{\mathbf{g} \in \cG_{p}} f(\mathbf{g}, \cG_{\pi}); \quad \text{Recall} (\cG_{\pi}, \cG_{p}) = \frac{1}{n} \sum_{\mathbf{g} \in \cG_{\pi}} f(\mathbf{g}, \cG_{p}).
	\]
	\vspace{-0.4cm}
	\begin{figure}
		\includegraphics[width=0.7\linewidth]{figs/pr_k_nearest}
	\end{figure}
	\myfootnotewithlink{https://arxiv.org/abs/1904.06991}{Kynkäänniemi T. et al. Improved precision and recall metric for assessing generative models, 2019}
\end{frame}
%=======
\begin{frame}{Precision-Recall for Generative Models}
	\begin{figure}
		\includegraphics[width=\linewidth]{figs/pr_vs_fid}
	\end{figure}
	\begin{figure}
		\includegraphics[width=0.75\linewidth]{figs/pr_truncation}
	\end{figure}
	\myfootnotewithlink{https://arxiv.org/abs/1904.06991}{Kynkäänniemi T. et al. Improved precision and recall metric for assessing generative models, 2019}
\end{frame}
%=======
\begin{frame}{Summary}
	\begin{itemize}
		\item Spectral normalization is a weight normalization technique to enforce Lipshitzness, which is helpful for generator and discriminator.
		\vfill
		\item f-divergence family is a unified framework for divergence minimization, which uses variational approximation.
		\vfill
		\item Inception Score and Frechet Inception Distance are the common metrics for GAN evaluation, but both of them have drawbacks.
	\end{itemize}
\end{frame}
\end{document} 