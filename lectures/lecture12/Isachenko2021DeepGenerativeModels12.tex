\documentclass{beamer}
\usepackage[utf8]{inputenc}
\usepackage{graphicx, epsfig}
\usepackage{amsmath,mathrsfs,amsfonts,amssymb}
%\usepackage{subfig}
\usepackage{floatflt}
\usepackage{epic,ecltree}
\usepackage{mathtext}
\usepackage{fancybox}
\usepackage{fancyhdr}
\usepackage{multirow}
\usepackage{enumerate}
\usepackage{epstopdf}
\usepackage{multicol}
\usepackage{algorithm}
\usepackage[noend]{algorithmic}
\usepackage{tikz}
\usepackage{blindtext}
\usetheme{default}%{Singapore}%{Warsaw}%{Warsaw}%{Darmstadt}
\usecolortheme{default}
\setbeamerfont{title}{size=\Huge}
\setbeamertemplate{footline}[page number]{}


\makeatletter
\newcommand\HUGE{\@setfontsize\Huge{35}{40}}
\makeatother    

\setbeamerfont{title}{size=\HUGE}
\beamertemplatenavigationsymbolsempty

\input{../utils/newcommands}
\input{../utils/title}

\newcommand\myfootnote[1]{%
  \tikz[remember picture,overlay]
  \draw (current page.south west) +(1in + \oddsidemargin,0.5em)
  node[anchor=south west,inner sep=0pt]{\parbox{\textwidth}{%
      \rlap{\rule{10em}{0.4pt}}\raggedright\scriptsize \textit{#1}}};}

\newcommand\myfootnotewithlink[2]{%
  \tikz[remember picture,overlay]
  \draw (current page.south west) +(1in + \oddsidemargin,0.5em)
  node[anchor=south west,inner sep=0pt]{\parbox{\textwidth}{%
      \rlap{\rule{10em}{0.4pt}}\raggedright\scriptsize\href{#1}{\textit{#2}}}};}
\createdgmtitle{12}
%--------------------------------------------------------------------------------
\begin{document}
%--------------------------------------------------------------------------------
\begin{frame}[noframenumbering,plain]
%\thispagestyle{empty}
\titlepage
\end{frame}
%=======
\begin{frame}{Recap of previous lecture}
	\vspace{-0.3cm}
	\[
		f(\bx, \bphi) = \bW_{K+1} \sigma_K (\bW_K \sigma_{K-1}(\dots \sigma_1(\bW_1 \bx) \dots)).
	\]
	\vspace{-0.3cm}
	\begin{itemize}
		\item $\sigma_k$ is a pointwise nonlinearities. We assume that $\| \sigma_k \|_L = 1$ (it holds for ReLU).
		\item $\mathbf{g}(\bx) = \bW \bx$ is a linear transformation ($\nabla \mathbf{g}(\bx) = \bW$).
		\[
			\| \mathbf{g} \|_L = \sup_\bx \| \nabla \mathbf{g}(\bx) \|_2 = \|\bW\|_2.
		\]
	\end{itemize}
	\vspace{-0.3cm}
	\begin{block}{Critic spectral norm}
		\vspace{-0.3cm}
		\[
			\| f \|_L \leq \| \bW_{K+1}\|_2 \cdot \prod_{k=1}^K  \| \sigma_k \|_L \cdot \| \bW_k \|_2 = \prod_{k=1}^{K+1} \|\bW_k\|_2.
		\]
		\vspace{-0.2cm}
	\end{block}
	\begin{block}{Spectral Normalization GAN}
	If we replace the weights in the critic $f(\bx, \bphi)$ by $\bW^{SN}_k = \bW_k / \|\bW_k\|_2$, we will get $\| f\|_L \leq 1.$ \\
	\end{block}
	 Power iteration approximates the value of $\|\bW\|_2$.
	
	\myfootnotewithlink{https://arxiv.org/abs/1802.05957}{Miyato T. et al. Spectral Normalization for Generative Adversarial Networks, 2018}

\end{frame}
%=======
\begin{frame}{Recap of previous lecture}
	\vspace{-0.3cm}
	\begin{block}{f-divergence minimization}
		\vspace{-0.3cm}
		\[
			D_f(\pi || p) = \bbE_{p(\bx)}  f\left( \frac{\pi(\bx)}{p(\bx)} \right) \rightarrow \min_p.
		\]
		Here $f: \bbR_+ \rightarrow \bbR$ is a convex, lower semicontinuous function satisfying $f(1) = 0$.
	\end{block}
	\begin{block}{Variational divergence estimation}
		\vspace{-0.3cm}
		\[
			D_f(\pi || p) \geq \sup_{T \in \cT} \left[\bbE_{\pi}T(\bx) -  \bbE_p f^*(T(\bx)) \right],
		\]
		\vspace{-0.7cm}
	\end{block}
	\begin{block}{Fenchel conjugate}
		\vspace{-0.7cm}
		\[
		f^*(t) = \sup_{u \in \text{dom}_f} \left( ut - f(u) \right), \quad f(u) = \sup_{t \in \text{dom}_{f^*}} \left( ut - f^*(t) \right)
		\]
		\vspace{-0.5cm}
	\end{block}
	\textbf{Note:} To evaluate lower bound we only need samples from $\pi(\bx)$ and $p(\bx)$. Hence, we could fit implicit generative model.
	\myfootnotewithlink{https://arxiv.org/abs/1606.00709}{Nowozin S., Cseke B., Tomioka R. f-GAN: Training Generative Neural Samplers using Variational Divergence Minimization, 2016}
\end{frame}
%=======
\begin{frame}{Recap of previous lecture}
	Let take some pretrained image classification model to get the conditional label distribution $p(y | \bx)$ (e.g. ImageNet classifier).
	\begin{block}{Evaluation of likelihood-free models}
		\begin{itemize}
			\item Sharpness $\Rightarrow$ low $H(y | \bx) = - \sum_{y} \int_{\bx} p(y, \bx) \log p(y | \bx) d\bx$.
			\item Diversity $\Rightarrow$ high $H(y)  = - \sum_{y} p(y) \log p(y)$.
		\end{itemize}
	\end{block}
	\vspace{-0.3cm}
	\begin{block}{Inception Score}
		\vspace{-0.2cm}
		\[
			IS = \exp(H(y) - H(y | \bx)) = \exp \left( \bbE_{\bx} KL(p(y | \bx) || p(y)) \right)
		\]
		\vspace{-0.5cm}
	\end{block}
	\begin{block}{Frechet Inception Distance}
		\vspace{-0.3cm}
		\[
			D^2 (\pi, p) = \| \mathbf{m}_{\pi} - \mathbf{m}_{p}\|_2^2 + \text{Tr} \left( \bSigma_{\pi} + \bSigma_p - 2 \sqrt{\bSigma_{\pi} \bSigma_p} \right).
		\]
		\vspace{-0.5cm}
	\end{block}
	FID is related to moment matching.
	
	\myfootnote{\href{https://arxiv.org/abs/1606.03498}{Salimans T. et al. Improved Techniques for Training GANs, 2016} \\
	\href{https://arxiv.org/abs/1706.08500}{Heusel M. et al. GANs Trained by a Two Time-Scale Update Rule Converge to a Local Nash Equilibrium, 2017} }
\end{frame}
%=======
\begin{frame}{Precision-Recall for Generative Models}
	\begin{block}{What do we want from samples}
		\begin{itemize}
			\item \textbf{Sharpness}: generated samples should be of high quality.
			\item \textbf{Diversity}: their variation should match that observed in the training set.
		\end{itemize}
	\end{block}
	\vspace{-0.5cm}
	\begin{figure}
		\includegraphics[width=0.95\linewidth]{figs/pr_curve}
	\end{figure}
	\vspace{-0.3cm}
	\begin{itemize}
		\item \textbf{Precision} denotes the fraction of generated images that are realistic.
		\item \textbf{Recall} measures the fraction of the training data manifold covered by the generator.
	\end{itemize}
	\myfootnotewithlink{https://arxiv.org/abs/1904.06991}{Kynkäänniemi T. et al. Improved precision and recall metric for assessing generative models, 2019}
\end{frame}
%=======
\begin{frame}{Precision-Recall for Generative Models}
	\begin{itemize}
		\item $\cS_{\pi} = \{\bx_i\}_{i=1}^{n} \sim \pi(\bx)$ -- real samples;
		\item $\cS_{p} = \{\bx_i\}_{i=1}^{n} \sim p(\bx | \btheta)$ -- generated samples.
	\end{itemize}
	Embed samples using pretrained classifier network (as previously):
	\[
		\cG_{\pi} = \{\mathbf{g}_i\}_{i=1}^n, \quad \cG_{p} = \{\mathbf{g}_i\}_{i=1}^n.
	\]
	Define binary function:
	\[
		f(\mathbf{g}, \cG) = 
		\begin{cases}
			1, \text{if exists } \mathbf{g}' \in \cG: \| \mathbf{g}  - \mathbf{g}'\|_2 \leq \| \mathbf{g}' - \text{NN}_k(\mathbf{g}', \cG)\|_2; \\
			0, \text{otherwise.}
		\end{cases}
	\]
	\[
		\text{Precision} (\cG_{\pi}, \cG_{p}) = \frac{1}{n} \sum_{\mathbf{g} \in \cG_{p}} f(\mathbf{g}, \cG_{\pi}); \quad \text{Recall} (\cG_{\pi}, \cG_{p}) = \frac{1}{n} \sum_{\mathbf{g} \in \cG_{\pi}} f(\mathbf{g}, \cG_{p}).
	\]
	\vspace{-0.4cm}
	\begin{figure}
		\includegraphics[width=0.7\linewidth]{figs/pr_k_nearest}
	\end{figure}
	\myfootnotewithlink{https://arxiv.org/abs/1904.06991}{Kynkäänniemi T. et al. Improved precision and recall metric for assessing generative models, 2019}
\end{frame}
%=======
\begin{frame}{Precision-Recall for Generative Models}
	\begin{figure}
		\includegraphics[width=\linewidth]{figs/pr_vs_fid}
	\end{figure}
	\begin{figure}
		\includegraphics[width=0.75\linewidth]{figs/pr_truncation}
	\end{figure}
	\myfootnotewithlink{https://arxiv.org/abs/1904.06991}{Kynkäänniemi T. et al. Improved precision and recall metric for assessing generative models, 2019}
\end{frame}
%=======
\begin{frame}{Evolution of GANs}
	\begin{figure}
		\centering
		\includegraphics[width=\linewidth]{figs/gan_evolution}
	\end{figure}
	\begin{itemize}
		\item \textbf{Standard GAN} \href{https://arxiv.org/abs/1406.2661}{https://arxiv.org/abs/1406.2661}
		\item \textbf{DCGAN} \href{https://arxiv.org/abs/1511.06434}{https://arxiv.org/abs/1511.06434}
		\item \textbf{CoGAN} \href{https://arxiv.org/abs/1606.07536}{https://arxiv.org/abs/1606.07536}
		\item \textbf{ProGAN} \href{https://arxiv.org/abs/1710.10196}{https://arxiv.org/abs/1710.10196} 
		\item \textbf{StyleGAN} \href{https://arxiv.org/abs/1812.04948}{https://arxiv.org/abs/1812.04948}
	\end{itemize}
\end{frame}
%=======
\begin{frame}{Self-Attention GAN}
	Convolutional layers process the information in a local neighborhood $\Rightarrow$ inefficient for modeling long-range dependencies in images.
	\vspace{-0.3cm}
	\begin{figure}
		\centering
		\includegraphics[width=0.9\linewidth]{figs/self-attention}
	\end{figure}
	\[
		\mathbf{f}(\bx) = \bW_f \bx, \quad \mathbf{g}(\bx) = \bW_g\bx, \quad \mathbf{h} (\bx) = \bW_h\bx, \quad \mathbf{v}(\bx) = \bW_v\bx
	\]
	\[
		s_{ij} = \mathbf{f}(\bx_i)^T \mathbf{g}(\bx_j), \quad a_{ij} = \frac{\exp{s_{ij}}}{\sum_{i=1}^N \exp{s_{ij}}}, \quad \textbf{o}_j = \textbf{v}\left( \sum_{i=1}^N a_{ij} \mathbf{h}(\bx_i) \right)
	\]
	\myfootnotewithlink{https://arxiv.org/abs/1805.08318}{Zhang H. et al. Self-Attention Generative Adversarial Networks, 2018}
\end{frame}
%=======
\begin{frame}{Self-Attention GAN}	
	\begin{block}{Convolution vs Attention}
		\vspace{-0.3cm}
		\begin{figure}
			\centering
			\includegraphics[width=0.7\linewidth]{figs/conv-vs-sa}
		\end{figure}
	\end{block}
	\vspace{-0.5cm}
	\begin{block}{Visualization of attention maps}
		\vspace{-0.3cm}
		\begin{figure}
			\centering
			\includegraphics[width=\linewidth]{figs/sa_maps}
		\end{figure}
	\end{block}

	\myfootnote{\href{https://lilianweng.github.io/lil-log/2018/06/24/attention-attention.html}{image credit: https://lilianweng.github.io/lil-log/2018/06/24/attention-attention.html}	\\ 
	\href{https://arxiv.org/abs/1805.08318}{Zhang H. et al. Self-Attention Generative Adversarial Networks, 2018}}
\end{frame}
%=======
\begin{frame}{BigGAN}
	\begin{block}{Batch-size is matter}
		\begin{figure}
			\centering
			\includegraphics[width=\linewidth]{figs/biggan_results}
		\end{figure}
	\end{block}
	\begin{block}{Samples (512x512)}
		\begin{figure}
			\centering
			\includegraphics[width=\linewidth]{figs/biggan_samples}
		\end{figure}
	\end{block}
	\vspace{-0.4cm}
	\myfootnotewithlink{https://arxiv.org/abs/1809.11096}{Brock A., Donahue J., Simonyan K. Large Scale GAN Training for High Fidelity Natural Image Synthesis, 2018}
\end{frame}
%=======
\begin{frame}{Progressive Growing GAN}
	\begin{block}{Problems with HR image generation}
		\begin{itemize}
			\item Disjoint manifolds $\Rightarrow$ gradient problem.
			\item Small minibatch $\Rightarrow$ training instability.
		\end{itemize}
	\end{block}
	\vspace{-0.2cm}
	\begin{block}{Samples (1024x1024)}
		\vspace{-0.2cm}
		\begin{figure}
			\includegraphics[width=0.9\linewidth]{figs/pggan_samples}
		\end{figure}
	\end{block}
	\myfootnotewithlink{https://arxiv.org/abs/1710.10196}{Karras T. et al. Progressive Growing of GANs for Improved Quality, Stability, and Variation, 2017}
s\end{frame}
\begin{frame}{Progressive Growing GAN}
	Grow both the generator and discriminator progressively, new layers will introduce higher-resolution details as the training progresses. 
	\begin{itemize}
		\item Train GAN which generate 4x4 images (2 convs for G and D).
		\item Add upsampling layers to G, downsampling layers to D.
		\item Train GAN which generate 8x8 images.
		\item etc.
	\end{itemize}
	\vspace{-0.2cm}
	\begin{figure}
		\includegraphics[width=0.75\linewidth]{figs/pggan_arch}
	\end{figure}

	\myfootnotewithlink{https://arxiv.org/abs/1710.10196}{Karras T. et al. Progressive Growing of GANs for Improved Quality, Stability, and Variation, 2017}
\end{frame}
%=======
\begin{frame}{StyleGAN}
	\begin{itemize}
		\item Generating of HR images is hard.
		\item Progressive growing greatly simplifies the task.
		\item The ability to control specific features of the generated image is very limited.
	\end{itemize}
	\begin{block}{Face image features}
		\begin{itemize}
			\item Coarse (pose, general hair style, face shape). Resolution $4^2 - 8^2$.
			\item Middle (finer facial features, hair style, eyes open/closed). Resolution $16^2 - 32^2$.
			\item Fine (color scheme (eye, hair and skin) and micro features). Resolution $64^2 - 1024^2$.
		\end{itemize}
	\end{block}
	\myfootnotewithlink{https://arxiv.org/abs/1812.04948}{Karras T., Laine S., Aila T. A Style-Based Generator Architecture for Generative Adversarial Networks, 2018}
\end{frame}
%=======
\begin{frame}{StyleGAN}
	\begin{block}{Mapping Network}
		\begin{itemize}
			\item Generator input is likely to be \textbf{disentangled}.  Each component of input vector $\bz$ should be responsible for one generative factor.
			\item Mapping network $f: \cZ \rightarrow \cW$ is used to reduce correlations between components of~$\bz$.
		\end{itemize}
		\begin{minipage}[t]{0.6\columnwidth}
			\begin{figure}
				\centering
				\includegraphics[width=0.98\linewidth]{figs/stylegan_mapping}
			\end{figure}
		\end{minipage}%
		\begin{minipage}[t]{0.38\columnwidth}
			\begin{figure}
				\centering
				\includegraphics[width=1.0\linewidth]{figs/stylegan_curved}
			\end{figure}
		\end{minipage}
	\vspace{0.3cm}
	\end{block}

	\myfootnotewithlink{https://arxiv.org/abs/1812.04948}{Karras T., Laine S., Aila T. A Style-Based Generator Architecture for Generative Adversarial Networks, 2018}
\end{frame}
%=======
\begin{frame}{Truncation trick}
	\begin{block}{BigGAN: truncated normal sampling}
		\vspace{-0.3cm}
		\[
			p(\bz | b) = \cN(\bz | 0, 1) / \int_{-\infty}^b \cN(\bz | 0, 1) d\bz
		\]
		Components of $\bz \sim \cN(0, \bI)$ which fall outside a predefined range are resampled.
	\end{block}
	
	\begin{block}{StyleGAN}
		\vspace{-0.2cm}
		\[
			\bw' = \hat{\bw} + \psi \cdot (\bw - \hat{\bw}), \quad \hat{\bw} = \bbE_{\bz} p(f(\bz))
		\]
		\vspace{-0.2cm}
		\begin{itemize}
			\item Constant $\psi$ is a tradeoff between diversity and fidelity. 
			\item $\psi=0.7$ is used for most of the results.
			\item Truncation is done only at the low-resolution layers.
		\end{itemize}
	\end{block}

	\myfootnotewithlink{https://arxiv.org/abs/1812.04948}{Karras T., Laine S., Aila T. A Style-Based Generator Architecture for Generative Adversarial Networks, 2018}
\end{frame}
%=======
\begin{frame}{StyleGAN}
	\begin{block}{Truncation trick}
		\begin{figure}
			\centering
			\includegraphics[width=0.85\linewidth]{figs/stylegan_truncation}
		\end{figure}
		\vspace{-0.4cm}
	\end{block}
	\begin{block}{Samples (1024x1024)}
		\begin{figure}
			\centering
			\includegraphics[width=0.8\linewidth]{figs/stylegan_samples}
		\end{figure}
	\end{block}

	\myfootnotewithlink{https://arxiv.org/abs/1812.04948}{Karras T., Laine S., Aila T. A Style-Based Generator Architecture for Generative Adversarial Networks, 2018}
\end{frame}
%=======
\begin{frame}{VAE recap}
	\vspace{-0.3cm}
	\begin{figure}[h]
		\centering
		\includegraphics[width=\linewidth]{figs/vae-gaussian.png}
	\end{figure}
	\vspace{-0.5cm}
	\begin{itemize}
		\item Encoder $q(\bz | \bx, \bphi) = \cN(\bz | \bmu_{\bphi}(\bx), \bsigma_{\bphi}(\bx))$.
		\item Variational posterior $q(\bz | \bx, \bphi)$ originally approximates the true posterior $p(\bz | \bx, \btheta)$.
		\item Which methods are you already familiar with to make the posterior is more flexible?
	\end{itemize}
	\myfootnotewithlink{https://lilianweng.github.io/lil-log/2018/08/12/from-autoencoder-to-beta-vae.html}{image credit: https://lilianweng.github.io/lil-log/2018/08/12/from-autoencoder-to-beta-vae.html}
\end{frame}
%=======
\begin{frame}{Adversarial Variational Bayes}
	\begin{block}{ELBO objective}
		\vspace{-0.5cm}
		\[
			 \mathcal{L} (\bphi, \btheta)  = \mathbb{E}_{q(\bz | \bx, \bphi)} \left[\log p(\bx | \bz, \btheta) + \log p(\bz) - \log q(\bz| \bx, \bphi) \right] \rightarrow \max_{\bphi, \btheta}.
		\]	
		\vspace{-0.5cm}
	\end{block}
	What is the problem to make the variational posterior model an implicit model?
	\begin{itemize}
	\item The first term is reconstruction loss that needs only samples from $q(\bz | \bx, \bphi)$ to evaluate.
	\item Reparametrization trick allows to get gradients of reconstruction loss
		\vspace{-0.4cm}
		\begin{multline*}
			\nabla_{\bphi}\int q(\bz|\bx, \bphi) f(\bz) d\bz = \nabla_{\bphi}\int r(\bepsilon)  f(\bz) d\bepsilon \\ = \int r(\bepsilon) \nabla_{\bphi} f(g(\bx, \bepsilon, \bphi)) d\bepsilon \approx \nabla_{\bphi} f(g(\bx, \bepsilon^*, \bphi)),
		\end{multline*}
		\vspace{-0.6cm} \\
		where $\bepsilon^* \sim r(\bepsilon), \quad \bz = g(\bx, \bepsilon, \bphi), \quad \bz \sim q(\bz | \bx, \bphi)$.
	\end{itemize}
	\myfootnotewithlink{https://arxiv.org/abs/1701.04722}{Mescheder L., Nowozin S., Geiger A. Adversarial variational bayes: Unifying variational autoencoders and generative adversarial networks, 2017}
\end{frame}
%=======
\begin{frame}{Adversarial Variational Bayes}
	\begin{block}{ELBO objective}
		\vspace{-0.5cm}
		\[
			 \mathcal{L} (\bphi, \btheta)  = \mathbb{E}_{q(\bz | \bx, \bphi)} \left[\log p(\bx | \bz, \btheta) + \log p(\bz) - \log q(\bz| \bx, \bphi) \right] \rightarrow \max_{\bphi, \btheta}.
		\]	
		\vspace{-0.5cm}
	\end{block}
	What is the problem to make the variational posterior model an implicit model?
	\begin{itemize}
	\item The third term requires the explicit the value of $q(\bz | \bx, \bphi)$.
	\item We could join second and third terms:
		\vspace{-0.2cm}
		\[
			\mathbb{E}_{q(\bz | \bx, \bphi)} \log \frac{p(\bz)}{q(\bz| \bx, \bphi)} = \mathbb{E}_{q(\bz | \bx, \bphi)} \log \frac{p(\bz) \pi(\bx)}{q(\bz| \bx, \bphi) \pi (\bx)}.
		\]
		\vspace{-0.5cm}
	\item We have to estimate density ratio 
		\vspace{-0.2cm}
		\[
			r(\bx, \bz) = \frac{q_1(\bx, \bz)}{q_2(\bx, \bz)} = \frac{p(\bz) \pi(\bx)}{q(\bz| \bx, \bphi) \pi (\bx)}.
		\] 
	\end{itemize}
	\myfootnotewithlink{https://arxiv.org/abs/1701.04722}{Mescheder L., Nowozin S., Geiger A. Adversarial variational bayes: Unifying variational autoencoders and generative adversarial networks, 2017}
\end{frame}
%=======
\begin{frame}{Density ratio trick}
	Consider two distributions $q_1(\bx)$, $q_2(\bx)$ and probabilistic model
	\[
		p(\bx | y) = \begin{cases}
			q_1(\bx), \text{ if } y = 1, \\
			q_2(\bx), \text{ if } y = 0,
		\end{cases}
		\quad 
		y \sim \text{Bern}(0.5).
	\]
	\vspace{-0.3cm}
	\begin{block}{Density ratio}
		\vspace{-0.7cm}
		{\small
		\begin{multline*}
			\frac{q_1(\bx)}{q_2(\bx)} = \frac{p(\bx | y = 1)}{p(\bx | y = 0)} = \frac{p(y = 1 | \bx) p(\bx)}{p(y=1)} \bigg/ \frac{p(y = 0 | \bx) p(\bx)}{p(y=0)} = \\
			= \frac{p(y = 1 | \bx)}{p(y = 0 | \bx)} = \frac{p(y = 1 | \bx)}{1 - p(y = 1 | \bx)} = \frac{D(\bx)}{1 - D(\bx)}
		\end{multline*}
		}
		Here $D(\bx)$ is a discriminator model the output of which is a probability that $\bx$ is a sample
		from $q_1(\bx)$ rather than from $q_2(\bx)$.
	\end{block}
	\begin{block}{Adversarial Variational Bayes}
		\vspace{-0.6cm}
		\[
			\max_D \left[ \bbE_{\pi(\bx)} \bbE_{q(\bz | \bx, \bphi)} \log D(\bx, \bz) + \bbE_{\pi(\bx)} \bbE_{p(\bz)} \log (1 - D(\bx, \bz)) \right]
		\]
	\end{block}
	\myfootnotewithlink{https://arxiv.org/abs/1701.04722}{Mescheder L., Nowozin S., Geiger A. Adversarial variational bayes: Unifying variational autoencoders and generative adversarial networks, 2017}
\end{frame}
%=======
\begin{frame}{Adversarial Variational Bayes}
	\begin{block}{ELBO objective}
		\vspace{-0.5cm}
		\[
			 \mathcal{L} (\bphi, \btheta)  = \mathbb{E}_{q(\bz | \bx, \bphi)} \left[\log p(\bx | \bz, \btheta) + \log \frac{p(\bz)}{q(\bz| \bx, \bphi)} \right] \rightarrow \max_{\bphi, \btheta}.
		\]	
		\vspace{-0.5cm}
	\end{block}
	\begin{figure}
		\includegraphics[width=0.7\linewidth]{figs/avb_scheme}
	\end{figure}
	\myfootnotewithlink{https://arxiv.org/abs/1701.04722}{Mescheder L., Nowozin S., Geiger A. Adversarial variational bayes: Unifying variational autoencoders and generative adversarial networks, 2017}
\end{frame}
%=======
\begin{frame}{Summary}
	\begin{itemize}
		\item Precision-recall allows to select model with compromise with sample quality and sample diversity.
		\vfill
		\item Self-Attention GAN allows to make huge receptive field and reduce convolution inductive bias.
		\vfill
		\item BigGAN shows that large batch size increase model quality gradually.
		\item Progressive growing for GAN learning allows to make training more stable.
		\vfill
		\item StyleGAN introduces mapping network to get more disentangled latent representation.
		\vfill
		\item Adversarial Variational Bayes uses density ratio trick to get more powerful variational posterior.
	\end{itemize}
\end{frame}
%=======
\end{document} 