\documentclass{beamer}
\usepackage[utf8]{inputenc}
\usepackage{graphicx, epsfig}
\usepackage{amsmath,mathrsfs,amsfonts,amssymb}
%\usepackage{subfig}
\usepackage{floatflt}
\usepackage{epic,ecltree}
\usepackage{mathtext}
\usepackage{fancybox}
\usepackage{fancyhdr}
\usepackage{multirow}
\usepackage{enumerate}
\usepackage{epstopdf}
\usepackage{multicol}
\usepackage{algorithm}
\usepackage[noend]{algorithmic}
\usepackage{tikz}
\usepackage{blindtext}
\usetheme{default}%{Singapore}%{Warsaw}%{Warsaw}%{Darmstadt}
\usecolortheme{default}
\setbeamerfont{title}{size=\Huge}
\setbeamertemplate{footline}[page number]{}


\makeatletter
\newcommand\HUGE{\@setfontsize\Huge{35}{40}}
\makeatother    

\setbeamerfont{title}{size=\HUGE}
\beamertemplatenavigationsymbolsempty

\input{../utils/newcommands}
\input{../utils/title}

\newcommand\myfootnote[1]{%
  \tikz[remember picture,overlay]
  \draw (current page.south west) +(1in + \oddsidemargin,0.5em)
  node[anchor=south west,inner sep=0pt]{\parbox{\textwidth}{%
      \rlap{\rule{10em}{0.4pt}}\raggedright\scriptsize \textit{#1}}};}

\newcommand\myfootnotewithlink[2]{%
  \tikz[remember picture,overlay]
  \draw (current page.south west) +(1in + \oddsidemargin,0.5em)
  node[anchor=south west,inner sep=0pt]{\parbox{\textwidth}{%
      \rlap{\rule{10em}{0.4pt}}\raggedright\scriptsize\href{#1}{\textit{#2}}}};}
\createdgmtitle{2}
%--------------------------------------------------------------------------------
\begin{document}
%--------------------------------------------------------------------------------
\begin{frame}[noframenumbering,plain]
%\thispagestyle{empty}
\titlepage
\end{frame}
%=======
\begin{frame}{Char RNN}
	Model tries to predict the next token (single letter) from previous context.
	\begin{minipage}[t]{0.55\columnwidth}
		\begin{figure}
			\centering
			\includegraphics[width=1.0\linewidth]{figs/char_rnn.png}
		\end{figure}
	\end{minipage}%
	\begin{minipage}[t]{0.44\columnwidth}
		\begin{figure}
			\centering
			\includegraphics[width=1.0\linewidth]{figs/char_rnn_output.png}
		\end{figure}
	\end{minipage}
\myfootnotewithlink{http://karpathy.github.io/2015/05/21/rnn-effectiveness/}{image credit: http://karpathy.github.io/2015/05/21/rnn-effectiveness}
\end{frame}
%=======
\begin{frame}{Autoregressive models}
		\begin{itemize}
			\item Convolutions could be used for autoregressive models, but they have to be \textbf{causal}. \\
			\item Try to find and understand the difference between Conv A/B.
		    \begin{figure}
		        \centering
		        \includegraphics[width=0.7\linewidth]{figs/sequential_CNN}
		    \end{figure}
		    \item Could learn long-range dependecies.
		    \item Do not suffer from gradient issues.
		    \item Easy to estimate probability for given input, but hard generation of new samples (the sequential process).
	   	\end{itemize}
	    \myfootnotewithlink{https://jmtomczak.github.io/blog/2/2\_ARM.html}{image credit: https://jmtomczak.github.io/blog/2/2\_ARM.html}
\end{frame}
%=======
\begin{frame}{MADE}
	\begin{itemize}
		\item Vanila autoencoder is not a generative model. Why?
		\item Let mask the weight matrices to make the model generative: $\bW_M = \bW \cdot \bM$.
		\begin{figure}
		    \centering
		    \includegraphics[width=0.7\linewidth]{figs/made}
		\end{figure}
		\item The question is how to create matrices $\bM$ which produce the autoregressive property?
	\end{itemize}
	\myfootnotewithlink{https://arxiv.org/abs/1502.03509}{Germain M. et al. Made: Masked autoencoder for distribution estimation, 2015}
\end{frame}
%=======
\begin{frame}{MADE}
		\begin{minipage}[t]{0.65\columnwidth}
		    \vspace{-0.5cm}
			\begin{block}{Masks generation}
				\begin{itemize}
					\item Define the ordering of input elements from 1 to $m$.
					\item Assign the random number $k$ from 1 to $m - 1$ to each hidden unit. The number gives the
					maximum number of input units to which the unit can be connected.
					\item Connect each hidden unit with number $k$ with the previous layer units which has the number is \textbf{less or equal} than~$k$.
					\item Connect each output unit with number $k$ with the previous layer units which has the number is \textbf{less} than $k$.
				\end{itemize}
			\end{block}
		\end{minipage}%
		\begin{minipage}[t]{0.33\columnwidth}
			\vspace{2cm}
			\begin{figure}
				\centering
				\includegraphics[width=1.0\linewidth]{figs/made2}
			\end{figure}
		\end{minipage}
	\myfootnotewithlink{https://arxiv.org/abs/1502.03509}{Germain M. et al. Made: Masked autoencoder for distribution estimation, 2015}
\end{frame}
%=======
\begin{frame}{MADE}
	\begin{block}{Possible variations}
		\begin{itemize}
			\item Order agnostic training (missing values in partially observed input vectors can be imputed efficiently);
			\item Connectivity-agnostic training (cheap ensembling).
		\end{itemize}
	\end{block}
	\vspace{-0.3cm}
	\begin{minipage}[t]{0.59\columnwidth}
		\begin{figure}
			  \includegraphics[width=\linewidth]{figs/made_nmasks}
		\end{figure}
	\end{minipage}%
	\begin{minipage}[t]{0.41\columnwidth}
		\begin{figure}
	  		\includegraphics[width=\linewidth]{figs/made_results}
		\end{figure}
	\end{minipage}
	\myfootnotewithlink{https://arxiv.org/abs/1502.03509}{Germain M. et al. Made: Masked autoencoder for distribution estimation, 2015}
\end{frame}
%=======%=======
\begin{frame}{WaveNet}
	\begin{block}{Goal}
		Efficient generation of raw audio waveforms with natural sounds.
	\end{block}
	\begin{figure}
	  \centering
	  \includegraphics[width=0.9\linewidth]{figs/wavenet_ex.png}
	\end{figure}
	\begin{block}{Solution}
		Autoregressive model
		\vspace{-0.3cm}
		\[
		    p(\bx| \btheta) = \prod_{t=1}^T p(x_t|\bx_{1:t-1}, \btheta).
		\]
		\vspace{-0.3cm}
	\end{block}
	\begin{itemize}
		\item Each conditional $p(x_t|\bx_{1:t-1}, \btheta)$ models the distribution for the timestamp $t$.
		\item The model uses \textbf{causal} dilated convolutions.
	\end{itemize}
	\myfootnotewithlink{https://arxiv.org/abs/1609.03499}{Oord A. et al. Wavenet: A generative model for raw audio, 2016}
\end{frame}
%=======
\begin{frame}{WaveNet}
	\begin{figure}
	    \centering
	    \includegraphics[width=0.9\linewidth]{figs/wavenet1.png}
	\end{figure}
	
	\begin{figure}
	    \centering
	    \includegraphics[width=0.9\linewidth]{figs/wavenet2.png}
	\end{figure}
	\myfootnotewithlink{https://arxiv.org/abs/1609.03499}{Oord A. et al. Wavenet: A generative model for raw audio, 2016}
\end{frame}
%=======
\begin{frame}{PixelCNN}
	\begin{block}{Goal}
		Model a distribution of natural images.
	\end{block}
	\begin{block}{Solution}
		Autoregressive model on 2D pixels
		\[
		    p(\bx | \btheta) = \prod_{i=1}^{\text{width} \times \text{height}} p(x_i|\bx_{1:i-1}, \btheta).
		\]
		\begin{itemize}
			\item We need to introduce the ordering of image pixels.
		    \item The convolution should be \textbf{masked} to make them causal.
		    \item The image has RGB channels, these dependencies could be addressed.
		\end{itemize}
	\end{block}
	\myfootnotewithlink{https://arxiv.org/abs/1601.06759}{Oord A., Kalchbrenner N., Kavukcuoglu K. Pixel recurrent neural networks, 2016}
\end{frame}
%=======
\begin{frame}{PixelCNN}
	\begin{minipage}[t]{0.5\columnwidth}
		\begin{block}{Raster ordering}
			\begin{figure}
				\centering
		        \includegraphics[width=0.7\linewidth]{figs/pixelcnn1.png}
			\end{figure}
		\end{block}
		\vspace{-0.5cm}
		\begin{block}{Masked convolution kernel}
			\begin{figure}
				\centering
		        \includegraphics[width=0.35\linewidth]{figs/pixelcnn_0_1.png}
			\end{figure}
		\end{block}
	\end{minipage}%
	\begin{minipage}[t]{0.5\columnwidth}
		\begin{block}{Dependencies between pixels}
			\begin{figure}
				\centering
		        \includegraphics[width=0.5\linewidth]{figs/pixelcnn_0_2.png}
			\end{figure}
			\vspace{-0.3cm}
			\begin{figure}
				\centering
		        \includegraphics[width=0.65\linewidth]{figs/pixelcnn2.png}
			\end{figure}
		\end{block}
	\end{minipage}
	\myfootnotewithlink{https://arxiv.org/abs/1601.06759}{Oord A., Kalchbrenner N., Kavukcuoglu K. Pixel recurrent neural networks, 2016}
\end{frame}
%=======
\begin{frame}{PixelCNN}
	\begin{block}{CIFAR-10 generated samples}
		\begin{figure}
			\centering
	  		\includegraphics[width=0.7\linewidth]{figs/pixelcnn_results}
		\end{figure}
	\end{block}
	\vspace{-0.2cm}
	\begin{block}{CIFAR-10 perfomance}
		\begin{figure}
			\centering
	  		\includegraphics[width=0.45\linewidth]{figs/pixelcnn_results2}
		\end{figure}
	\end{block}
	\myfootnotewithlink{https://arxiv.org/abs/1601.06759}{Oord A., Kalchbrenner N., Kavukcuoglu K. Pixel recurrent neural networks, 2016}
\end{frame}
%=======
\begin{frame}{PixelCNN++}
	\begin{block}{CIFAR-10 pixel values distribution}
		\begin{figure}
			\includegraphics[width=0.3\linewidth]{figs/pixelcnn++_pixels_distr}
		\end{figure}
	\end{block}
	\vspace{-0.5cm}
	\begin{itemize}
		\item Standard PixelCNN outputs softmax probabilities for values $\{0, 255\}$ (256 outputs feature maps). 
		\item Categorical distribution do not know anything about numerical relationships (220 is close to 221 and far from 15).
		\item If pixel value is not presented in the training dataset , it won't be predicted.
		\item (Look at the edges of the distributions: they have higher probability mass).
	\end{itemize}
	\myfootnotewithlink{https://arxiv.org/abs/1701.05517}{Salimans T. et al. PixelCNN++: Improving the PixelCNN with Discretized Logistic Mixture Likelihood and Other Modifications, 2017}
\end{frame}
%=======
\begin{frame}{PixelCNN++}
		\begin{minipage}[t]{0.55\columnwidth}
		\begin{block}{Mixture of logistic distributions}
			\begin{align*}
				p(\nu | \mu, s) &= \frac{\exp^{-(\nu - \mu) / s}}{s (1 + \exp^{-(\nu - \mu) / s})^2}; \\
				p(\nu | \bmu, \bs, \bpi) &= \sum_{i=1}^K \pi_k p(\nu | \mu_k, s_k);
			\end{align*}
		\end{block}
		\end{minipage}%
		\begin{minipage}[t]{0.45\columnwidth}
				\begin{figure}
					\includegraphics[width=\linewidth]{figs/picelcnn++logistic}
				\end{figure}
		\end{minipage}
	To adopt probability calculation to discrete values:
	\[
		P(x | \bmu, \bs, \bpi) = P(x + 0.5 | \bmu, \bs, \bpi) - P(x - 0.5 | \bmu, \bs, \bpi)
	\]
	For the edge case of 0, replace $x - 0.5$ by $-\infty$, and for 255 replace $x + 0.5$ by $+\infty$.
	\myfootnotewithlink{https://arxiv.org/abs/1701.05517}{Salimans T. et al. PixelCNN++: Improving the PixelCNN with Discretized Logistic Mixture Likelihood and Other Modifications, 2017}
\end{frame}
%=======
\begin{frame}{PixelCNN++}
	\begin{block}{CIFAR-10 generated samples}
		\begin{figure}
			\centering
	  		\includegraphics[width=0.7\linewidth]{figs/pixelcnn++samples}
		\end{figure}
	\end{block}
	\vspace{-0.2cm}
	\begin{block}{CIFAR-10 perfomance}
		\vspace{-0.3cm}
		\begin{figure}
			\centering
	  		\includegraphics[width=0.5\linewidth]{figs/pixelcnn++results}
		\end{figure}
	\end{block}
	\myfootnotewithlink{https://arxiv.org/abs/1701.05517}{Salimans T. et al. PixelCNN++: Improving the PixelCNN with Discretized Logistic Mixture Likelihood and Other Modifications, 2017}
\end{frame}
%=======
\begin{frame}{Recap of previous lecture}
	We are given i.i.d. samples $\{\bx_i\}_{i=1}^n \in X$ (e.g. $X = \bbR^m$) from unknown distribution $\pi(\bx)$.
	\begin{block}{Goal}
		We would like to learn a distribution $\pi(\bx)$ for 
		\begin{itemize}
			\item evaluating $\pi(\bx)$ for new samples (how likely to get object $\bx$?);
			\item sampling from $\pi(\bx)$ (to get new objects $\bx \sim \pi(\bx)$).
		\end{itemize}
	\end{block}
	\begin{block}{Challenge}
		Data is complex and high-dimensional.
	\end{block}
	\begin{block}{MLE problem}
	Fix probabilistic model $p(\bx | \btheta)$~-- a set of parameterized distributions, such that $p(\bx | \btheta) \approx \pi(\bx)$.
	
		\vspace{-0.3cm}
		\[
		\btheta^* = \argmax_{\btheta} p(\bX | \btheta) = \argmax_{\btheta} \prod_{i=1}^n p(\bx_i | \btheta) = \argmax_{\btheta} \sum_{i=1}^n \log p(\bx_i | \btheta).
		\]
		\vspace{-0.1cm}
	\end{block}
\end{frame}
\begin{frame}{Recap of previous lecture}
	\begin{block}{Likelihood as product of conditionals}
		Let $\bx = (x_1, \dots, x_m)$, $\bx_{1:i} = (x_1, \dots, x_i)$. Then 
		\[
		p(\bx | \btheta) = \prod_{i=1}^m p(x_i | \bx_{1:i - 1}, \btheta); \quad 
		\log p(\bx | \btheta) = \sum_{i=1}^m \log p(x_i | \bx_{1:i - 1}, \btheta).
		\]
	\end{block}
	\vspace{-0.3cm}
	\begin{block}{MLE problem for autoregressive model}
		\vspace{-0.5cm}
		\[
		\btheta^* = \argmax_{\btheta} p(\bX | \btheta) = \argmax_{\btheta} \sum_{i=1}^n \sum_{j=1}^m \log p(x_{ij} | \bx_{i, 1:j - 1}\btheta).
		\]
		\vspace{-0.5cm}
	\end{block}
	\begin{block}{Sampling}
		\vspace{-0.5cm}
		\[
			\hat{x}_1 \sim p(x_1 | \btheta), \quad \hat{x}_2 \sim p(x_2 | \hat{x}_1, \btheta), \dots, \quad \hat{x}_m \sim p(x_m | \hat{\bx}_{1:m-1}, \btheta)
		\]
		New generated object is $\hat{\bx} = (\hat{x}_1, \hat{x}_2, \dots, \hat{x}_m)$.
	\end{block}
\end{frame}
%=======
\begin{frame}{Generative models zoo}
    \begin{figure}
        \centering
        \includegraphics[width=1.0\linewidth]{figs/generative_models_zoo.pdf}
        \label{fig:generative_models_zoo}
    \end{figure}
\end{frame}
%=======
\begin{frame}{Bayesian framework}
	\begin{block}{Bayes theorem}
		\[
			p(\bt | \bx) = \frac{p(\bx | \bt) p(\bt)}{p(\bx)} = \frac{p(\bx | \bt) p(\bt)}{\int p(\bx | \bt) p(\bt) d \bt} 
		\]
		\begin{itemize}
			\item $\bx$ -- observed variables, $\bt$ -- unobserved variables (latent variables/parameters);
			\item $p(\bx | \bt)$ -- likelihood;
			\item $p(\bx) = \int p(\bx | \bt) p(\bt) d \bt$ -- evidence;
			\item $p(\bt)$ -- prior distribution, $p(\bt | \bx)$ -- posterior distribution.
		\end{itemize}
	\end{block}
	\begin{block}{Meaning}
		We have unobserved variables $\bt$ and some prior knowledge about them $p(\bt)$. Then, the data $\bx$ has been observed. 
		Posterior distribution $p(\bt | \bx)$ summarizes the knoweldge after the obbservations.
	\end{block}
\end{frame}
%=======
\begin{frame}{Bayesian framework}
	Let consider the case, where the unobserved variables $\bt$ is our model parameters $\btheta$.
	\begin{itemize}
		\item $\bX = \{\bx_i\}_{i=1}^n$ -- observed samples;
		\item $p(\btheta)$ -- prior parameters distribution (we treat model parameters $\btheta$ as random variables).
	\end{itemize}
	\begin{block}{Posterior distribution}
		\[
			p(\btheta | \bX) = \frac{p(\bX | \btheta) p(\btheta)}{p(\bX)} = \frac{p(\bX | \btheta) p(\btheta)}{\int p(\bX | \btheta) p(\btheta) d \btheta} 
		\]
		\vspace{-0.2cm}
	\end{block}
	\begin{block}{Bayesian inference}
		\vspace{-0.2cm}
		\[
			p(\bx | \bX) = \int p(\bx | \btheta) p(\btheta | \bX) d \btheta
		\]
	\end{block}
 	Note the difference from
	 	\[
	 		p(\bx) = \int p(\bx | \btheta) p(\btheta) d \btheta.
	 	\]
\end{frame}
%=======
\begin{frame}{Bayesian framework}
	\begin{block}{Posterior distribution}
		\[
		p(\btheta | \bX) = \frac{p(\bX | \btheta) p(\btheta)}{p(\bX)} = \frac{p(\bX | \btheta) p(\btheta)}{\int p(\bX | \btheta) p(\btheta) d \btheta} 
		\]
		\vspace{-0.2cm}
	\end{block}
	\begin{block}{Bayesian inference}
		\vspace{-0.2cm}
		\[
		p(\bx | \bX) = \int p(\bx | \btheta) p(\btheta | \bX) d \btheta
		\]
	\end{block}
	If evidence $p(\bX)$ is intractable (due to multidimensional integration), we can't get posterior distribution and perform the precise inference.
    \begin{block}{Maximum a posteriori (MAP) estimation}
    \vspace{-0.2cm}
    \[
        \btheta^* = \argmax_{\btheta} p(\btheta | \bX) = \argmax_{\btheta} \bigl(\log p(\bX | \btheta) + \log p(\btheta) \bigr)
    \]
    \end{block}
\end{frame}
%=======
\begin{frame}{Bayesian framework}
	\begin{block}{MAP estimation}
		\vspace{-0.2cm}
		\[
		\btheta^* = \argmax_{\btheta} p(\btheta | \bX) = \argmax_{\btheta} \bigl(\log p(\bX | \btheta) + \log p(\btheta) \bigr)
		\]
	\end{block}
	Estimated $\btheta^*$ is a deterministic variable, but we could treat it as a random variable with density $p(\btheta | \bX) = \delta(\btheta - \btheta^*)$.
	\begin{block}{Dirac delta function}
		\[
			\delta(x) = 
			\begin{cases}
				+\infty, \quad x = 0; \\
				0, \quad x \neq 0;
			\end{cases} \quad 
			\int f(x) \delta (x - y) dx = f(y).
		\]
	\end{block}
	\begin{block}{MAP inference}
		\[
			p(\bx | \bX) = \int p(\bx| \btheta) p(\btheta | \bX ) d \btheta \approx p(\bx | \btheta^*).
		\]
	\end{block}
\end{frame}
%=======
\begin{frame}{Latent variable models (LVM)}
    \begin{block}{MLE problem}
    \vspace{-0.5cm}
    \[
        \btheta^* = \argmax_{\btheta} p(\bX | \btheta) = \argmax_{\btheta} \prod_{i=1}^n p(\bx_i | \btheta) = \argmax_{\btheta} \sum_{i=1}^n \log p(\bx_i | \btheta).
    \]
    \vspace{-0.5cm}
    \end{block}
    \begin{block}{Challenge}
    $p(\bx | \btheta)$ could be intractable.
    \end{block}
    \begin{block}{Extend probabilistic model}
    Introduce latent variable $\bz$ for each sample $\bx$
    \[
        p(\bx, \bz | \btheta) = p(\bx | \bz, \btheta) p(\bz); \quad 
        \log p(\bx, \bz | \btheta) = \log p(\bx | \bz, \btheta) + \log p(\bz).
    \]
    \[
        p(\bx | \btheta) = \int p(\bx, \bz | \btheta) d\bz = \int p(\bx | \bz, \btheta) p(\bz) d\bz.
    \]
    \end{block}
	\vspace{-0.3cm}
	\begin{block}{Motivation}
		The distributions $p(\bx | \bz, \btheta)$ and $p(\bz)$ could be quite simple.
	\end{block}
\end{frame}
%=======
\begin{frame}{Latent variable models (LVM)}
    \[
    \log p(\bx | \btheta) = \log \int p(\bx | \bz, \btheta) p(\bz) d\bz \rightarrow \max_{\btheta}
    \]
    \vspace{-0.6cm}
    \begin{block}{Examples}
    \begin{minipage}[t]{0.45\columnwidth}
		\textit{Mixture of gaussians} \\
		\vspace{-0.5cm}
		\begin{figure}
			\centering
			\includegraphics[width=0.75\linewidth]{figs/mixture_of_gaussians.png}
		\end{figure}
		\vspace{-0.5cm}
	    \begin{itemize}
	        \item $p(\bx | \bz, \btheta) = \mathcal{N}(\bx | \boldsymbol{\mu}_\bz, \boldsymbol{\Sigma}_\bz)$
	        \item $p(\bz) = \text{Categorical}(\bz | \boldsymbol{\pi})$
	    \end{itemize}
	\end{minipage}%
	\begin{minipage}[t]{0.53\columnwidth}
		\textit{PCA model} \\
		\vspace{-0.5cm}
		\begin{figure}
			\centering
			\includegraphics[width=.7\linewidth]{figs/pca.png}
		\end{figure}
		\vspace{-0.5cm}
		\begin{itemize}
	        \item $p(\bx | \bz, \btheta) = \mathcal{N}(\bx | \mathbf{W} \bz + \boldsymbol{\mu}, \boldsymbol{\Sigma}_\bz)$
	        \item $p(\bz) = \mathcal{N}(\bz | 0, \mathbf{I})$
	    \end{itemize}
	\end{minipage}
	\end{block}
\myfootnote{Bishop\,C. Pattern Recognition and Machine Learning, 2006}
\end{frame}
%=======
\begin{frame}{Latent variable models (LVM)}
    \[
    \log p(\bx | \btheta) = \log \int p(\bx | \bz, \btheta) p(\bz) d\bz \rightarrow \max_{\btheta}
    \]
	\textbf{PCA goal:} Project original data $\bX$ onto a low dimensional latent space while maximizing the variance of the projected data. 
	\begin{figure}
		\centering
		\includegraphics[width=.7\linewidth]{figs/bayesian_pca.png}
	\end{figure}
	\vspace{-0.5cm}
	\begin{itemize}
        \item $p(\bx | \bz, \btheta) = \mathcal{N}(\bx | \mathbf{W} \bz + \boldsymbol{\mu}, \boldsymbol{\Sigma}_\bz)$
        \item $p(\bz) = \mathcal{N}(\bz | 0, \mathbf{I})$
    \end{itemize}
    
    \myfootnote{Bishop\,C. Pattern Recognition and Machine Learning, 2006}
    
\end{frame}
%=======
\begin{frame}{Incomplete likelihood}
        \begin{block}{MLE}
            \vspace{-0.7cm}
            \begin{multline*}
                \vspace{-0.5cm}
                \btheta^* = \argmax_{\btheta} p(\bX, \bZ | \btheta) = \argmax_{\btheta} \prod_{i=1}^n p(\bx_i, \bz_i | \btheta) = \\ = \argmax_{\btheta} \sum_{i=1}^n \log p(\bx_i, \bz_i | \btheta).
            \end{multline*}
            \vspace{-0.5cm}
        \end{block}
	Since $\bZ$ is unknown, maximize \textbf{incomplete likelihood}.
    \begin{block}{MILE problem}
        \vspace{-0.7cm}
    	\begin{multline*}
        	\btheta^* = \argmax_{\btheta} \log p(\bX| \btheta) = \argmax_{\btheta} \sum_{i=1}^n \log p(\bx_i | \btheta) = \\ =  \argmax_{\btheta}  \sum_{i=1}^n \log \int p(\bx_i, \bz_i | \btheta) d \bz_i = \\ = \argmax_{\btheta} \log \sum_{i=1}^n \int p(\bx_i| \bz_i, \btheta) p(\bz_i) d\bz_i.
    	\end{multline*}
	\end{block}
	
\end{frame}
%=======
\begin{frame}{Variational lower bound}
	\begin{multline*}
		\log p(\bx| \btheta) 
		= \log \frac{p(\bx, \bz| \btheta)}{p(\bz|\bx, \btheta)} = \\ 
		= \int q(\bz) \log \frac{p(\bx, \bz| \btheta)}{p(\bz|\bx, \btheta)}d\bz
		= \int q(\bz) \log \frac{p(\bx, \bz| \btheta) q(\bz)}{p(\bz|\bx, \btheta) q(\bz)} d\bz = \\
		= \int q(\bz) \log \frac{p(\bx, \bz | \btheta)}{q(\bz)}d\bz + \int q(\bz) \log \frac{q(\bz)}{p(\bz|\bx, \btheta)}d\bz = \\ 
		= \mathcal{L} (q, \btheta) + KL(q(\bz) || p(\bz|\bx, \btheta)) \geq \mathcal{L} (q, \btheta).
	\end{multline*}
	\vspace{-0.5cm}
	\begin{block}{Kullback-Leibler divergence}
	    \begin{itemize}
	    	\item $KL(q || p) = \int q(\bz) \log \frac{q(\bz)}{p(\bz)} d \bz$;
	        \item $KL(q || p) \geq 0$;
	        \item $KL(q || p) = 0 \Leftrightarrow q \equiv p$.
	    \end{itemize}
	\end{block}
\end{frame}
%=======
\begin{frame}{Variational lower bound}
\[
    \log p(\bx| \btheta) = \mathcal{L} (q, \btheta) + KL(q(\bz) || p(\bz|\bx, \btheta)) \geq \mathcal{L} (q, \btheta).
\]
\begin{block}{Evidence Lower Bound (ELBO)}
\begin{align*}
    \mathcal{L} (q, \btheta) &= \int q(\bz) \log \frac{p(\bx, \bz | \btheta)}{q(\bz)}d\bz = \\ 
    &= \int q(\bz) \log p(\bx | \bz, \btheta) d\bz + \int q(\bz) \log \frac{p(\bz)}{q(\bz)}d\bz \\ 
    &= \mathbb{E}_{q} \log p(\bx | \bz, \btheta) - KL (q(\bz) || p(\bz))
\end{align*}
\end{block}
Instead of maximizing incomplete likelihood, maximize ELBO (equivalently minimize KL)
\[
    \max_{\btheta} p(\bx | \btheta) \quad \rightarrow \quad \max_{q, \btheta} \mathcal{L} (q, \btheta) \equiv \min_{q, \btheta} KL(q(\bz) || p(\bz|\bx, \btheta)).
\]
    
\end{frame}
%=======
\begin{frame}{EM-algorithm}
	\[
		\mathcal{L} (q, \btheta)  = \int q(\bz) \log p(\bx | \bz, \btheta) d\bz + \int q(\bz) \log \frac{p(\bz)}{q(\bz)}d\bz.
	\]
	\begin{block}{Block-coordinate optimization}
	\begin{itemize}
		\item Initialize $\btheta^*$;
		\item E-step
		\[
			q(\bz) = \argmax_q \mathcal{L} (q, \btheta^*) = \argmin_q KL(q || p) =
			 p(\bz| \bx, \btheta^*);
		\]
		\item M-step
		\[
			\btheta^* = \argmax_{\btheta} \mathcal{L} (q, \btheta);
		\]
		\item Repeat E-step and M-step until convergence.
	\end{itemize}
	\end{block}
\end{frame}
%=======
\begin{frame}{Ugly pic}
\end{frame}
%=======
\begin{frame}{Amortized variational inference}
    \begin{block}{E-step}
    \vspace{-0.3cm}
    \[
		q(\bz) = \argmax_q \mathcal{L} (q, \btheta^*) = \argmin_q KL(q || p) =
		 p(\bz| \bx, \btheta^*).
	\]
	\begin{itemize}
		\item $p(\bz| \bx, \btheta^*)$ could be \textbf{intractable};
		\item $q(\bz)$ is different for each object $\bx$.
	\end{itemize}
    \end{block}
	\begin{block}{Idea}
	Restrict a family of all possible distributions $q(\bz)$ to a particular parametric class $q(\bz|\bx, \bphi)$ conditioned on samples $\bx$ with parameters $\bphi$.
	\end{block}
	
	\textbf{Variational Bayes}
	\begin{itemize}
		\item E-step
		\[
		\bphi_k = \bphi_{k-1} + \left.\eta \nabla_{\bphi} \mathcal{L}(\bphi, \btheta_{k-1})\right|_{\bphi=\bphi_{k-1}}
		\]
		\item M-step
		\[
		\btheta_k = \btheta_{k-1} + \left.\eta \nabla_{\btheta} \mathcal{L}(\bphi_k, \btheta)\right|_{\btheta=\btheta_{k-1}}
		\]
	\end{itemize}
\end{frame}
%=======
\begin{frame}{Variational EM-algorithm}

	\begin{block}{ELBO}
		\vspace{-0.1cm}
		\[
		\log p(\bx| \btheta) = \mathcal{L} (q, \btheta) + KL(q(\bz) || p(\bz|\bx, \btheta)) \geq \mathcal{L} (q, \btheta).
		\]
	\end{block}
	\begin{itemize}
		\item E-step
		\[
		\bphi_k = \bphi_{k-1} + \left.\eta \nabla_{\bphi} \mathcal{L}(\bphi, \btheta_{k-1})\right|_{\bphi=\bphi_{k-1}},
		\]
		where $\bphi$~-- parameters of variational distribution $q(\bz | \bx, \bphi)$.
		\item M-step
		\[
		\btheta_k = \btheta_{k-1} + \left.\eta \nabla_{\btheta} \mathcal{L}(\bphi_k, \btheta)\right|_{\btheta=\btheta_{k-1}},
		\]
		where $\btheta$~-- parameters of the generation function $p(\bx | \bz, \btheta)$.
	\end{itemize}
	Now all we have to do is to obtain two gradients $\nabla_{\bphi} \mathcal{L}(\bphi, \btheta)$, $\nabla_{\btheta} \mathcal{L}(\bphi, \btheta)$.  \\
	\textbf{Difficulty:} number of samples $n$.
\end{frame}
%=======
\begin{frame}{Summary}
	\begin{itemize}
		\item Bayesian inference is a generalization of most common machine learning tasks. It allows to construct MLE, MAP and bayesian inference, to compare models complexity and many-many more cool stuff.
		\vfill
		\item LVM introduce latent representation of observed samples to make model more interpretable.
		\vfill
		\item LVM maximizes variational evidence lower bound to find MLE of model parameters.
		\vfill
		\item ELBO maximization is performed by the general variational EM algorithm.
	\end{itemize}
\end{frame}
\end{document} 