\documentclass{beamer}
\usepackage[utf8]{inputenc}
\usepackage{graphicx, epsfig}
\usepackage{amsmath,mathrsfs,amsfonts,amssymb}
%\usepackage{subfig}
\usepackage{floatflt}
\usepackage{epic,ecltree}
\usepackage{mathtext}
\usepackage{fancybox}
\usepackage{fancyhdr}
\usepackage{multirow}
\usepackage{enumerate}
\usepackage{epstopdf}
\usepackage{multicol}
\usepackage{algorithm}
\usepackage[noend]{algorithmic}
\usepackage{tikz}
\usepackage{blindtext}
\usetheme{default}%{Singapore}%{Warsaw}%{Warsaw}%{Darmstadt}
\usecolortheme{default}
\setbeamerfont{title}{size=\Huge}
\setbeamertemplate{footline}[page number]{}


\makeatletter
\newcommand\HUGE{\@setfontsize\Huge{35}{40}}
\makeatother    

\setbeamerfont{title}{size=\HUGE}
\beamertemplatenavigationsymbolsempty

\input{../utils/newcommands}
\input{../utils/title}

\newcommand\myfootnote[1]{%
  \tikz[remember picture,overlay]
  \draw (current page.south west) +(1in + \oddsidemargin,0.5em)
  node[anchor=south west,inner sep=0pt]{\parbox{\textwidth}{%
      \rlap{\rule{10em}{0.4pt}}\raggedright\scriptsize \textit{#1}}};}

\newcommand\myfootnotewithlink[2]{%
  \tikz[remember picture,overlay]
  \draw (current page.south west) +(1in + \oddsidemargin,0.5em)
  node[anchor=south west,inner sep=0pt]{\parbox{\textwidth}{%
      \rlap{\rule{10em}{0.4pt}}\raggedright\scriptsize\href{#1}{\textit{#2}}}};}
\createdgmtitle{3}
%--------------------------------------------------------------------------------
\begin{document}
%--------------------------------------------------------------------------------
\begin{frame}[noframenumbering,plain]
%\thispagestyle{empty}
\titlepage
\end{frame}
%=======
\begin{frame}{Recap of previous lecture}
	\begin{block}{MLE problem for autoregressive model}
		\vspace{-0.7cm}
		\[
		\btheta^* = \argmax_{\btheta} p(\bX | \btheta) = \argmax_{\btheta} \sum_{i=1}^n \sum_{j=1}^m \log p(x_{ij} | \bx_{i, 1:j - 1}\btheta).
		\]
		\vspace{-0.7cm}
	\end{block}
	\begin{block}{Sampling}
		\vspace{-0.5cm}
		\[
			\hat{x}_1 \sim p(x_1 | \btheta), \quad \hat{x}_2 \sim p(x_2 | \hat{x}_1, \btheta), \dots, \quad \hat{x}_m \sim p(x_m | \hat{\bx}_{1:m-1}, \btheta)
		\]
		New generated object is $\hat{\bx} = (\hat{x}_1, \hat{x}_2, \dots, \hat{x}_m)$.
	\end{block}
	Masking helps to make neural network autoregressive.
	\begin{itemize}
		\item \textbf{MADE} - masked autoencoder (MLP).
		\item \textbf{WaveNet} - masked 1D convolutions.
		\item \textbf{PixelCNN} - masked 2D convolutions.
	\end{itemize}
	\textbf{PixelCNN++} uses discretized mixture of logistic distribution to make the output distribution more natural.
\end{frame}
%=======
\begin{frame}{Recap of previous lecture}
	\begin{block}{Posterior distribution}
		\[
		p(\btheta | \bX) = \frac{p(\bX | \btheta) p(\btheta)}{p(\bX)} = \frac{p(\bX | \btheta) p(\btheta)}{\int p(\bX | \btheta) p(\btheta) d \btheta} 
		\]
		\vspace{-0.2cm}
	\end{block}
	\begin{block}{Bayesian inference}
		\vspace{-0.2cm}
		\[
		p(\bx | \bX) = \int p(\bx | \btheta) p(\btheta | \bX) d \btheta
		\]
		\vspace{-0.2cm}
	\end{block}
	\begin{block}{Maximum a posteriori (MAP) estimation}
		\vspace{-0.2cm}
		\[
		\btheta^* = \argmax_{\btheta} p(\btheta | \bX) = \argmax_{\btheta} \bigl(\log p(\bX | \btheta) + \log p(\btheta) \bigr)
		\]
		\vspace{-0.2cm}
	\end{block}
	\begin{block}{MAP inference}
		\[
		p(\bx | \bX) = \int p(\bx| \btheta) p(\btheta | \bX ) d \btheta \approx p(\bx | \btheta^*).
		\]
	\end{block}
\end{frame}
%=======
\begin{frame}{Latent variable models (LVM)}
	\begin{block}{MLE problem}
		\vspace{-0.5cm}
		\[
		\btheta^* = \argmax_{\btheta} p(\bX | \btheta) = \argmax_{\btheta} \prod_{i=1}^n p(\bx_i | \btheta) = \argmax_{\btheta} \sum_{i=1}^n \log p(\bx_i | \btheta).
		\]
		\vspace{-0.5cm}
	\end{block}
	The distribution $p(\bx | \btheta)$ could be very complex and intractable (as well as real distribution $\pi(\bx)$).
	\begin{block}{Extended probabilistic model}
		Introduce latent variable $\bz$ for each sample $\bx$
		\[
		p(\bx, \bz | \btheta) = p(\bx | \bz, \btheta) p(\bz); \quad 
		\log p(\bx, \bz | \btheta) = \log p(\bx | \bz, \btheta) + \log p(\bz).
		\]
		\[
		p(\bx | \btheta) = \int p(\bx, \bz | \btheta) d\bz = \int p(\bx | \bz, \btheta) p(\bz) d\bz.
		\]
	\end{block}
	\vspace{-0.3cm}
	\begin{block}{Motivation}
		The distributions $p(\bx | \bz, \btheta)$ and $p(\bz)$ could be quite simple.
	\end{block}
\end{frame}
%=======
\begin{frame}{Latent variable models (LVM)}
	\[
	\log p(\bx | \btheta) = \log \int p(\bx | \bz, \btheta) p(\bz) d\bz \rightarrow \max_{\btheta}
	\]
	\vspace{-0.6cm}
	\begin{block}{Examples}
		\begin{minipage}[t]{0.45\columnwidth}
			\textit{Mixture of gaussians} \\
			\vspace{-0.5cm}
			\begin{figure}
				\centering
				\includegraphics[width=0.75\linewidth]{figs/mixture_of_gaussians}
			\end{figure}
			\vspace{-0.5cm}
			\begin{itemize}
				\item $p(\bx | z, \btheta) = \cN(\bx | \bmu_z, \bSigma_z)$
				\item $p(z) = \text{Categorical}(\bpi)$
			\end{itemize}
		\end{minipage}%
		\begin{minipage}[t]{0.53\columnwidth}
			\textit{PCA model} \\
			\vspace{-0.5cm}
			\begin{figure}
				\centering
				\includegraphics[width=.7\linewidth]{figs/pca}
			\end{figure}
			\vspace{-0.3cm}
			\begin{itemize}
				\item $p(\bx | \bz, \btheta) = \cN(\bx | \bW \bz + \bmu, \sigma^2 \bI)$
				\item $p(\bz) = \cN(\bz | 0, \bI)$
			\end{itemize}
		\end{minipage}
	\end{block}
	\myfootnote{Bishop\,C. Pattern Recognition and Machine Learning, 2006}
\end{frame}
%=======
\begin{frame}{Latent variable models (LVM)}
	\[
	\log p(\bx | \btheta) = \log \int p(\bx | \bz, \btheta) p(\bz) d\bz \rightarrow \max_{\btheta}
	\]
	\textbf{PCA} projects original data $\bX$ onto a low dimensional latent space while maximizing the variance of the projected data. 
	\begin{figure}
		\centering
		\includegraphics[width=.7\linewidth]{figs/bayesian_pca}
	\end{figure}
	\vspace{-0.5cm}
	\begin{itemize}
		\item $p(\bx | \bz, \btheta) = \cN(\bx | \bW \bz + \bmu, \sigma^2 \bI)$
		\item $p(\bz) = \cN(\bz | 0, \bI)$
		\item $p(\bx) = \cN(\bx | \bmu, \bW \bW^T + \sigma^2 \bI)$
		\item $p(\bz | \bx) = \cN \bigl(\bM^{-1} \bW^T (\bx - \bmu), \sigma^2 \bM\bigr)$, where $\bM = \bW \bW^T + \sigma^2 \bI$
	\end{itemize}
\end{frame}
%=======
\begin{frame}{Maximum likelihood estimation for LVM}
	\begin{block}{MLE for extended problem}
		\vspace{-0.7cm}
		\begin{multline*}
			\vspace{-0.5cm}
			\btheta^* = \argmax_{\btheta} p(\bX, \bZ | \btheta) = \argmax_{\btheta} \prod_{i=1}^n p(\bx_i, \bz_i | \btheta) = \\ = \argmax_{\btheta} \sum_{i=1}^n \log p(\bx_i, \bz_i | \btheta).
		\end{multline*}
		\vspace{-0.5cm}
	\end{block}
	However, $\bZ$ is unknown.
	\begin{block}{MLE for original problem}
		\vspace{-0.7cm}
		\begin{multline*}
			\btheta^* = \argmax_{\btheta} \log p(\bX| \btheta) = \argmax_{\btheta} \sum_{i=1}^n \log p(\bx_i | \btheta) = \\ =  \argmax_{\btheta}  \sum_{i=1}^n \log \int p(\bx_i, \bz_i | \btheta) d \bz_i = \\ = \argmax_{\btheta} \log \sum_{i=1}^n \int p(\bx_i| \bz_i, \btheta) p(\bz_i) d\bz_i.
		\end{multline*}
	\end{block}
	
\end{frame}
%=======
\begin{frame}{Naive approach}
	\begin{figure}
		\includegraphics[width=.75\linewidth]{figs/lvm_diagram}
	\end{figure}
	\begin{block}{Monte-Carlo estimation}
		\vspace{-0.7cm}
		\[
			p(\bx | \btheta) = \int p(\bx | \bz, \btheta) p(\bz) d\bz = \bbE_{p(\bz)} p(\bx | \bz, \btheta) \approx \frac{1}{K} \sum_{k=1}^{K} p(\bx | \bz_k, \btheta),
		\]
		\vspace{-0.5cm} \\
		where $\bz_k \sim p(\bz)$. \\
		\textbf{Challenge:} to cover the space properly, the number of samples grows exponentially with respect to dimensionality of $\bz$. 
	\end{block}
	\myfootnotewithlink{https://jmtomczak.github.io/blog/4/4\_VAE.html}{image credit: https://jmtomczak.github.io/blog/4/4\_VAE.html}
\end{frame}
%=======
\begin{frame}{Variational lower bound (ELBO)}
	\begin{block}{Derivation 1}
		\vspace{-0.7cm}
		\begin{multline*}
			\log p(\bx| \btheta) 
			= \log \int p(\bx, \bz | \btheta) d\bz = \\ 
			= \log \int \frac{q(\bz)}{q(\bz)} p(\bx, \bz | \btheta) d\bz
			= \log \bbE_{q} \left[\frac{p(\bx, \bz| \btheta)}{q(\bz)} \right] \geq \\
			\geq \bbE_{q} \log \frac{p(\bx, \bz| \btheta)}{q(\bz)} = \int q(\bz) \log \frac{p(\bx, \bz| \btheta)}{q(\bz)} d \bz = \cL(q, \btheta)
		\end{multline*}
		\vspace{-0.7cm}
	\end{block}
	\begin{block}{Derivation 2}
		\vspace{-0.7cm}
		\begin{multline*}
			\log p(\bx| \btheta) 
			= \int q(\bz) \log p(\bx| \btheta) d\bz = \\ 
			= \int q(\bz) \log \frac{p(\bx, \bz| \btheta)}{p(\bz|\bx, \btheta)}d\bz
			= \int q(\bz) \log \frac{p(\bx, \bz| \btheta) q(\bz)}{p(\bz|\bx, \btheta) q(\bz)} d\bz = \\
			= \int q(\bz) \log \frac{p(\bx, \bz | \btheta)}{q(\bz)}d\bz + \int q(\bz) \log \frac{q(\bz)}{p(\bz|\bx, \btheta)}d\bz = \\ 
			= \mathcal{L} (q, \btheta) + KL(q(\bz) || p(\bz|\bx, \btheta)) \geq \mathcal{L} (q, \btheta).
		\end{multline*}
		\vspace{-0.7cm}
	\end{block}
\end{frame}
%=======
\begin{frame}{Variational lower bound}
	\vspace{-0.5cm}
	\begin{align*}
 	 \mathcal{L} (q, \btheta) &= \int q(\bz) \log \frac{p(\bx, \bz | \btheta)}{q(\bz)}d\bz = \\ 
	  &= \int q(\bz) \log p(\bx | \bz, \btheta) d\bz + \int q(\bz) \log \frac{p(\bz)}{q(\bz)}d\bz \\ 
	  &= \mathbb{E}_{q} \log p(\bx | \bz, \btheta) - KL (q(\bz) || p(\bz))
	\end{align*}
	\vspace{-0.5cm}
	\begin{block}{Log-likelihood decomposition}
		\vspace{-0.5cm}
		\[
			 \log p(\bx| \btheta) = \mathbb{E}_{q} \log p(\bx | \bz, \btheta) - KL (q(\bz) || p(\bz)) + KL(q(\bz) || p(\bz|\bx, \btheta)).
		\]
	\end{block}
	\begin{itemize}
	\item Instead of maximizing incomplete likelihood, maximize ELBO
   	\[
	    \max_{\btheta} p(\bx | \btheta) \quad \rightarrow \quad \max_{q, \btheta} \mathcal{L} (q, \btheta)
   	\]
   	\item Maximization of ELBO by variational distribution $q$ is equivalent to minimization of KL
  	\[
	    \max_q \mathcal{L} (q, \btheta) \equiv \min_q KL(q(\bz) || p(\bz|\bx, \btheta)).
  	\]
  	\end{itemize}
	   	    
\end{frame}
%=======
\begin{frame}{EM-algorithm}
	\[
		\mathcal{L} (q, \btheta)  = \int q(\bz) \log p(\bx | \bz, \btheta) d\bz + \int q(\bz) \log \frac{p(\bz)}{q(\bz)}d\bz.
	\]
	\begin{block}{Block-coordinate optimization}
	\begin{itemize}
		\item Initialize $\btheta^*$;
		\item E-step
		\begin{multline*}
			q^*(\bz) = \argmax_q \mathcal{L} (q, \btheta^*) = \\
			= \argmin_q KL(q(\bz) || p(\bz | \bx, \btheta^*)) = p(\bz| \bx, \btheta^*);
		\end{multline*}
		\item M-step
		\[
			\btheta^* = \argmax_{\btheta} \mathcal{L} (q^*, \btheta);
		\]
		\item Repeat E-step and M-step until convergence.
	\end{itemize}
	\end{block}
\end{frame}
%=======
\begin{frame}{EM illustration}
	
	\begin{minipage}[t]{0.45\columnwidth}
		\begin{figure}
			\includegraphics[width=0.9\linewidth]{figs/em_bishop1}
		\end{figure}
	\end{minipage}%
	\begin{minipage}[t]{0.55\columnwidth}
		\begin{figure}
			\includegraphics[width=0.85\linewidth]{figs/em_bishop2}
		\end{figure}
	\end{minipage}
	\begin{figure}
		\includegraphics[width=.55\linewidth]{figs/em_bishop3}
	\end{figure}

	\myfootnote{Bishop\,C. Pattern Recognition and Machine Learning, 2006}
\end{frame}
%=======
\begin{frame}{Amortized variational inference}
    \begin{block}{E-step}
    \vspace{-0.3cm}
    \[
		q(\bz) = \argmax_q \mathcal{L} (q, \btheta^*) = \argmin_q KL(q || p) =
		 p(\bz| \bx, \btheta^*).
	\]
	\begin{itemize}
		\item $p(\bz| \bx, \btheta^*)$ could be \textbf{intractable};
		\item $q(\bz)$ is different for each object $\bx$.
	\end{itemize}
    \end{block}
	\begin{block}{Idea}
	Restrict a family of all possible distributions $q(\bz)$ to a parametric class $q(\bz|\bx, \bphi)$ conditioned on samples $\bx$ with parameters $\bphi$.
	\end{block}
	
	\textbf{Variational Bayes}
	\begin{itemize}
		\item E-step
		\[
		\bphi_k = \bphi_{k-1} + \left.\eta \nabla_{\bphi} \mathcal{L}(\bphi, \btheta_{k-1})\right|_{\bphi=\bphi_{k-1}}
		\]
		\item M-step
		\[
		\btheta_k = \btheta_{k-1} + \left.\eta \nabla_{\btheta} \mathcal{L}(\bphi_k, \btheta)\right|_{\btheta=\btheta_{k-1}}
		\]
	\end{itemize}
\end{frame}
%=======
\begin{frame}{Variational EM-algorithm}
	\begin{block}{ELBO}
		\vspace{-0.5cm}
		\[
			\log p(\bx| \btheta) = \mathcal{L} (\bphi, \btheta) + KL(q(\bz | \bx, \bphi) || p(\bz|\bx, \btheta)) \geq \mathcal{L} (\bphi, \btheta).
		\]
		\vspace{-0.5cm}
	\end{block}
	\begin{itemize}
		\item E-step
		\[
		\bphi_k = \bphi_{k-1} + \left.\eta \nabla_{\bphi} \mathcal{L}(\bphi, \btheta_{k-1})\right|_{\bphi=\bphi_{k-1}},
		\]
		where $\bphi$~-- parameters of variational distribution $q(\bz | \bx, \bphi)$.
		\item M-step
		\[
		\btheta_k = \btheta_{k-1} + \left.\eta \nabla_{\btheta} \mathcal{L}(\bphi_k, \btheta)\right|_{\btheta=\btheta_{k-1}},
		\]
		where $\btheta$~-- parameters of the generative distribution $p(\bx | \bz, \btheta)$.
	\end{itemize}
	Now all we have to do is to obtain two gradients $\nabla_{\bphi} \mathcal{L}(\bphi, \btheta)$, $\nabla_{\btheta} \mathcal{L}(\bphi, \btheta)$.  \\
	\textbf{Challenge:} Number of samples $n$ could be huge (we heed to derive unbiased stochastic gradients).
\end{frame}
%=======
\begin{frame}{ELBO gradients}
	\[
		 \mathcal{L} (\bphi, \btheta)  = \mathbb{E}_{q} \left[\log p(\bx | \bz, \btheta) + \log \frac{p(\bz)}{q(\bz | \bx, \bphi)} \right] \rightarrow \max_{\bphi, \btheta}.
	\]	
	\begin{block}{M-step: $\nabla_{\btheta} \mathcal{L}(\bphi, \btheta)$}
		\vspace{-0.7cm}
		\begin{multline*}
			\nabla_{\btheta} \mathcal{L} (\bphi, \btheta)
			= \int q(\bz|\bx, \bphi) \nabla_{\btheta}\log p(\bx|\bz, \btheta) d \bz \approx  \\
			\approx \nabla_{\btheta}\log p(\bx|\bz^*, \btheta), \quad \bz^* \sim q(\bz|\bx, \bphi).
		\end{multline*}
		\vspace{-0.7cm}
	\end{block}
	\begin{block}{E-step: $\nabla_{\bphi} \mathcal{L}(\bphi, \btheta)$}
	Difference from M-step: density function $q(\bz| \bx, \bphi)$ depends on the parameters $\bphi$, it is impossible to use the Monte-Carlo estimation:
	\begin{align*}
		\nabla_{\bphi} \mathcal{L} (\bphi, \btheta) &= \nabla_{\bphi} \int q(\bz | \bx, \bphi) \left[\log p(\bx | \bz, \btheta) + \log \frac{p(\bz)}{q(\bz| \bx, \bphi)} \right] d \bz \\
		& \neq  \int q(\bz | \bx, \bphi) \nabla_{\bphi} \left[\log p(\bx | \bz, \btheta) + \log \frac{p(\bz)}{q(\bz| \bx, \bphi)} \right] d \bz \\
	\end{align*}
	\end{block}
\end{frame}
%=======
\begin{frame}{Summary}
	\begin{itemize}
		\item LVM introduces latent representation of observed samples to make model more interpretable.
		\vfill
		\item LVM maximizes variational evidence lower bound (ELBO) to find MLE of model parameters.
		\vfill
		\item The general variational EM algorithm maximizes ELBO objective.
		\vfill
		\item Amortized inference allows to efficiently compute stochastic gradients for ELBO using Monte-Carlo estimation.
	\end{itemize}
\end{frame}
%=======
\end{document} 