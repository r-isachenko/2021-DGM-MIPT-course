\documentclass{beamer}
\usepackage[utf8]{inputenc}
\usepackage{graphicx, epsfig}
\usepackage{amsmath,mathrsfs,amsfonts,amssymb}
%\usepackage{subfig}
\usepackage{floatflt}
\usepackage{epic,ecltree}
\usepackage{mathtext}
\usepackage{fancybox}
\usepackage{fancyhdr}
\usepackage{multirow}
\usepackage{enumerate}
\usepackage{epstopdf}
\usepackage{multicol}
\usepackage{algorithm}
\usepackage[noend]{algorithmic}
\usepackage{tikz}
\usepackage{blindtext}
\usetheme{default}%{Singapore}%{Warsaw}%{Warsaw}%{Darmstadt}
\usecolortheme{default}
\setbeamerfont{title}{size=\Huge}
\setbeamertemplate{footline}[page number]{}


\makeatletter
\newcommand\HUGE{\@setfontsize\Huge{35}{40}}
\makeatother    

\setbeamerfont{title}{size=\HUGE}
\beamertemplatenavigationsymbolsempty

\input{../utils/newcommands}
\input{../utils/title}

\newcommand\myfootnote[1]{%
  \tikz[remember picture,overlay]
  \draw (current page.south west) +(1in + \oddsidemargin,0.5em)
  node[anchor=south west,inner sep=0pt]{\parbox{\textwidth}{%
      \rlap{\rule{10em}{0.4pt}}\raggedright\scriptsize \textit{#1}}};}

\newcommand\myfootnotewithlink[2]{%
  \tikz[remember picture,overlay]
  \draw (current page.south west) +(1in + \oddsidemargin,0.5em)
  node[anchor=south west,inner sep=0pt]{\parbox{\textwidth}{%
      \rlap{\rule{10em}{0.4pt}}\raggedright\scriptsize\href{#1}{\textit{#2}}}};}
\createdgmtitle{3}
%--------------------------------------------------------------------------------
\begin{document}
%--------------------------------------------------------------------------------
\begin{frame}
%\thispagestyle{empty}
\titlepage
\end{frame}
%=======
\begin{frame}{Recap of previous lecture}
    \begin{block}{MLE problem}
    \vspace{-0.5cm}
    \[
        \btheta^* = \argmax_{\btheta} p(\bX | \btheta) = \argmax_{\btheta} \prod_{i=1}^n p(\bx_i | \btheta) = \argmax_{\btheta} \sum_{i=1}^n \log p(\bx_i | \btheta).
    \]
    \vspace{-0.5cm}
    \end{block}
    \begin{block}{Challenge}
    $p(\bx | \btheta)$ could be intractable.
    \end{block}
    \begin{block}{LVM}
    Introduce latent variable $\bz$ for each sample $\bx$
    \[
        p(\bx, \bz | \btheta) = p(\bx | \bz, \btheta) p(\bz); \quad 
        \log p(\bx, \bz | \btheta) = \log p(\bx | \bz, \btheta) + \log p(\bz).
    \]
    \[
        p(\bx | \btheta) = \int p(\bx, \bz | \btheta) d\bz = \int p(\bx | \bz, \btheta) p(\bz) d\bz.
    \]
    \end{block}
	\vspace{-0.3cm}
	\begin{block}{Motivation}
		The distributions $p(\bx | \bz, \btheta)$ and $p(\bz)$ could be quite simple.
	\end{block}
\end{frame}
%=======
\begin{frame}{Recap of previous lecture}
        \begin{block}{Incomplete likelihood maximization}
        \vspace{-0.7cm}
    	\[
    	        \btheta^* = \argmax_{\btheta} \log p(\bX| \btheta) = \argmax_{\btheta} \log \sum_{i=1}^n \int p(\bx_i| \bz_i, \btheta) p(\bz_i) d\bz_i.
    	\]
    	\vspace{-0.5cm}
	\end{block}
	\begin{block}{Variational lower bound}
		\[
		    \log p(\bx| \btheta) = \mathcal{L} (q, \btheta) + KL(q(\bz) || p(\bz|\bx, \btheta)) \geq \mathcal{L} (q, \btheta).
		\]
	\end{block}
	\begin{block}{Evidence Lower Bound (ELBO)}
	    \vspace{-0.3cm}
		\[
		    \mathcal{L} (q, \btheta) = \mathbb{E}_{q} \log p(\bx | \bz, \btheta) - KL (q(\bz) || p(\bz)).
		\]
	\end{block}
	Instead of maximizing incomplete likelihood, maximize ELBO (equivalently minimize KL)
	\[
	    \max_{\btheta} \log p(\bx | \btheta) \quad \rightarrow \quad \max_{q, \btheta} \mathcal{L} (q, \btheta) \equiv \min_{q, \btheta} KL(q(\bz) || p(\bz|\bx, \btheta)).
	\]
	    
\end{frame}
%=======
\begin{frame}{Recap of previous lecture}
	\begin{block}{EM algorithm (block-coordinate optimization)}
	\begin{itemize}
		\item Initialize $\btheta^*$;
		\item E-step
			\vspace{-0.3cm}
			\[
				q(\bz) = \argmax_q \mathcal{L} (q, \btheta^*) = \argmin_q KL(q || p) =
				 p(\bz| \bx, \btheta^*);
			\]
			\vspace{-0.5cm}
			\begin{itemize}
				\item $p(\bz| \bx, \btheta^*)$ could be \textbf{intractable};
				\item $q(\bz)$ is different for each object $\bx$.
			\end{itemize}
		\item M-step
			\[
				\btheta^* = \argmax_{\btheta} \mathcal{L} (q, \btheta);
			\]
		\item Repeat E-step and M-step until convergence.
	\end{itemize}
	\end{block}
	\begin{block}{Amortized variational inference}
	Restrict a family of all possible distributions $q(\bz)$ to a particular parametric class $q(\bz|\bx, \bphi)$ conditioned on samples $\bx$ with parameters $\bphi$.
	\end{block}
\end{frame}
%=======
\begin{frame}{Variational EM-algorithm}

	\begin{block}{ELBO}
		\vspace{-0.1cm}
		\[
		\log p(\bx| \btheta) = \mathcal{L} (\bphi, \btheta) + KL(q(\bz | \bx, \bphi) || p(\bz|\bx, \btheta)) \geq \mathcal{L} (\bphi, \btheta).
		\]
	\end{block}
	\begin{itemize}
		\item E-step
		\[
		\bphi_k = \bphi_{k-1} + \left.\eta \nabla_{\bphi} \mathcal{L}(\bphi, \btheta_{k-1})\right|_{\bphi=\bphi_{k-1}},
		\]
		where $\bphi$~-- parameters of variational distribution $q(\bz | \bx, \bphi)$.
		\item M-step
		\[
		\btheta_k = \btheta_{k-1} + \left.\eta \nabla_{\btheta} \mathcal{L}(\bphi_k, \btheta)\right|_{\btheta=\btheta_{k-1}},
		\]
		where $\btheta$~-- parameters of the generative distribution $p(\bx | \bz, \btheta)$.
	\end{itemize}
	Now all we have to do is to obtain two gradients $\nabla_{\bphi} \mathcal{L}(\bphi, \btheta)$, $\nabla_{\btheta} \mathcal{L}(\bphi, \btheta)$.  \\
	\begin{block}{Challenge}
		Number of samples $n$ could be huge (we heed to derive unbiased stochastic gradients).
	\end{block}
\end{frame}
%=======
\begin{frame}{ELBO interpretations}
	\[
		\log p(\bx | \btheta) = \cL (q, \btheta) + KL(q(\bz | \bx, \bphi) || p(\bz | \bx, \btheta)).
	\]
	\[
		\cL (q, \btheta) = \int q(\bz | \bx, \bphi) \log \frac{p(\bx, \bz | \btheta)}{q(\bz | \bx, \bphi)} d \bz.
	\]
	\begin{itemize}
	    \item Evidence minus posterior KL
	    \vspace{-0.1cm}
	    \[
	        \mathcal{L}(q, \btheta) = \log p(\bx | \btheta) - KL(q(\bz | \bx, \bphi) || p(\bz | \bx, \btheta)).
	    \]
	    \item Average negative energy plus entropy
	    \vspace{-0.1cm}
	    \begin{align*}
	        \mathcal{L}(q, \btheta) &= \mathbb{E}_{q(\bz | \bx, \bphi)} \left[\log p(\bx, \bz | \btheta) - \log q(\bz | \bx, \bphi)  \right] \\
	        &= \mathbb{E}_{q(\bz | \bx, \bphi)} \log p(\bx, \bz | \btheta) + \mathbb{H} \left[q(\bz | \bx, \bphi) \right].
	    \end{align*}
	    \item Average reconstruction minus KL to prior
	    \vspace{-0.1cm}
	    \begin{align*}
	        \mathcal{L}(q, \btheta) &= \mathbb{E}_{q(\bz | \bx, \bphi)} \left[ \log p(\bx | \bz, \btheta) + \log p(\bz) - \log q(\bz | \bx, \bphi) \right] \\
	        &= \mathbb{E}_{q(\bz | \bx, \bphi)} \log p(\bx | \bz, \btheta) - KL(q(\bz | \bx, \bphi) || p(\bz)).
	    \end{align*}
	\end{itemize}
\end{frame}
%=======
\begin{frame}{Monte-Carlo estimation}
		\vspace{-0.3cm}
	\[
		\sum_{i=1}^n \bbE_q f(\bz_i) \approx n \cdot \bbE_q f(\bz) = n \cdot \int q(\bz) f(\bz) d\bz \approx n \cdot f(\bz^*), \text{where } \bz^* \sim q(\bz).
	\]
	\begin{block}{ELBO gradients}
			\vspace{-0.5cm}
		\[
			\nabla_{\btheta} \sum_{i=1}^n \mathcal{L}_i (\bphi, \btheta) \approx n \cdot \nabla_{\btheta} \cL (\bphi, \btheta); 
			\quad \nabla_{\bphi} \sum_{i=1}^n \mathcal{L}_i (\bphi, \btheta) \approx n \cdot \nabla_{\bphi} \cL (\bphi, \btheta) 
		\]
		\vspace{-0.3cm}
	\end{block}
	\begin{block}{ELBO}
		\vspace{-0.3cm}
		\[
		\mathcal{L} (\bphi, \btheta)  = \mathbb{E}_{q} \left[\log p(\bx, \bz | \btheta) - \log q(\bz | \bx , \bphi) \right] \rightarrow \max_{\bphi, \btheta}.
		\]
		\vspace{-0.5cm}
	\end{block}
	
	\begin{block}{ELBO gradient (M-step, $\nabla_{\btheta} \mathcal{L}(\bphi, \btheta)$)}
		\vspace{-0.5cm}
		\begin{multline*}
			\nabla_{\btheta} \mathcal{L} (\bphi, \btheta)
			= \int q(\bz|\bx, \bphi) \nabla_{\btheta}\log p(\bx|\bz, \btheta) d \bz \approx  \\
			\approx \nabla_{\btheta}\log p(\bx|\bz^*, \btheta), \quad \bz^* \sim q(\bz|\bx, \bphi).
		\end{multline*}
	\end{block}
\end{frame}
%=======
\begin{frame}{ELBO gradient (E-step, $\nabla_{\bphi} \mathcal{L}(\bphi, \btheta)$)}
	\vspace{-0.3cm}
	\[
		 \mathcal{L} (\bphi, \btheta)  = \mathbb{E}_{q} \left[\log p(\bx, \bz | \btheta) - \log q(\bz| \bx, \bphi) \right] \rightarrow \max_{\bphi, \btheta}.
	\]	
	\vspace{-0.3cm}
	\begin{block}{Challenge}
	Difference from M-step: density function $q(\bz| \bx, \bphi)$ depends on the parameters $\bphi$, it is impossible to use the Monte-Carlo estimation:
	\begin{align*}
		\nabla_{\bphi} \mathcal{L} (\bphi, \btheta) &= \nabla_{\bphi} \int q(\bz | \bx, \bphi) \left[\log p(\bx, \bz | \btheta) - \log q(\bz| \bx, \bphi) \right] d \bz \\
		& \neq  \int q(\bz | \bx, \bphi) \nabla_{\bphi} \left[\log p(\bx, \bz | \btheta) - \log q(\bz| \bx, \bphi) \right] d \bz \\
	\end{align*}
	\end{block}
	\vspace{-0.8cm}
	\begin{block}{Solution}
		Reparametrization trick for $q(\bz | \bx, \bphi)$ allows the expectation to become independent of parameters~$\bphi$.
	\end{block}
\end{frame}
%======
\begin{frame}{Reparametrization trick}
		\vspace{-0.3cm}
		\[
		f(\xi) = \bbE_{q(\eta | \xi)} h(\eta) = \int q(\eta|\xi) h(\eta) d\eta
		\]
		Let $\eta = g(\xi, \epsilon)$, where $g$ is a deterministic function, $\epsilon$ is a random variable with a density function $r(\epsilon)$.
		{\small
		\[
			f(\xi) = \int q(\eta|\xi) h(\eta) d\eta = \int r(\epsilon) h(g(\xi, \epsilon)) d \epsilon \approx h(g(\xi, \epsilon^*)), \quad \epsilon^* \sim r(\epsilon).
		\]}
	\begin{block}{Examples} 
		\begin{itemize}
		\item $r(\epsilon) = \mathcal{N}(\epsilon|0, 1)$, $\eta = \sigma \cdot \epsilon + \mu$, $q(\eta|\xi) = \mathcal{N}(\eta| \mu, \sigma^2)$, $\xi = [\mu, \sigma]$.

		\item $\bepsilon^* \sim r(\bepsilon), \quad \bz = g(\bx, \bepsilon, \bphi), \quad \bz \sim q(\bz | \bx, \bphi)$
		\begin{multline*}
			\nabla_{\bphi}\int q(\bz|\bx, \bphi) f(\bz) d\bz = \nabla_{\bphi}\int r(\bepsilon)  f(\bz) d\bepsilon \\ = \int r(\bepsilon) \nabla_{\bphi} f(g(\bx, \bepsilon, \bphi)) d\bepsilon \approx \nabla_{\bphi} f(g(\bx, \bepsilon^*, \bphi))
		\end{multline*}
		\end{itemize}
	\end{block}
\end{frame}
%=======
\begin{frame}{ELBO gradient (E-step, $\nabla_{\bphi} \mathcal{L}(\bphi, \btheta)$)}
	\vspace{-0.7cm}
	\begin{multline*}
		\nabla_{\bphi} \mathcal{L} (\bphi, \btheta) = \nabla_{\bphi}\int q(\bz|\bx, \bphi) \left[\log p(\bx, \bz | \btheta)  - \log q(\bz | \bx, \bphi)\right] d\bz
		\\ = \int r(\bepsilon) \nabla_{\bphi} \bigl[\log p(\bx, g(\bx, \bepsilon, \bphi) | \btheta)  - \log q\bigl(g(\bx, \bepsilon, \bphi) | \bx, \bphi \bigr)\bigr]d\bepsilon
		\\ \approx \nabla_{\bphi} \bigl[\log p(\bx, g(\bx, \bepsilon^*, \bphi) | \btheta)  - \log q\bigl(g(\bx, \bepsilon^*, \bphi) | \bx, \bphi \bigr)\bigr]
	\end{multline*}
	\vspace{-0.7cm}
	\begin{block}{Variational assumption}
		\[
			r(\bepsilon) = \mathcal{N}(0, \bI); \quad  q(\bz| \bx, \bphi) = \mathcal{N} (\bmu_{\bphi}(\bx), \bsigma^2_{\bphi}(\bx)).
		\]
		\[
			\bz = g(\bx, \bepsilon, \bphi) = \bsigma_{\bphi}(\bx) \cdot \bepsilon + \bmu_{\bphi}(\bx).
		\]
		Here $\bmu_{\bphi}(\cdot), \bsigma_{\bphi}(\cdot)$ are parameterized functions (outputs of neural network).
	\end{block}
	If we could calculate $\log p(\bx, \bz | \btheta)$ and $\log q(\bz | \bx, \bphi)$, we are done. Could we?
\end{frame}
%=======
\begin{frame}{ELBO gradient (E-step, $\nabla_{\bphi} \mathcal{L}(\bphi, \btheta)$)}
	\vspace{-0.3cm}
	\[
		\nabla_{\bphi} \mathcal{L} (\bphi, \btheta) \approx \nabla_{\bphi} \bigl[\log p(\bx, g(\bx, \bepsilon^*, \bphi) | \btheta)  - \log q\bigl(g(\bx, \bepsilon^*, \bphi) | \bx, \bphi \bigr)\bigr]
	\]
	\vspace{-0.3cm}
	\begin{block}{First term}
		\vspace{-0.3cm}
		\[
			\log p(\bx, \bz | \btheta) = \log p(\bx | \bz, \btheta) + \log p(\bz).
		\]
		\vspace{-0.3cm}
		\begin{itemize}
			\item $p(\bz)$ -- prior distribution on latent variables $\bz$. We could specify any distribution that we want. Let say $p(\bz) = \cN (0, \bI)$.
			\item $p(\bx | \bz, \btheta)$ - generative distibution. Since it is a parameterized function let it be neural network with parameters $\btheta$.
		\end{itemize}
	\end{block}
	\begin{block}{Second term}
		Function $\bz = g(\bx, \bepsilon, \bphi)= \bsigma_{\bphi}(\bx) \cdot \bepsilon + \bmu_{\bphi}(\bx)$ is invertible.
		\[
			q(\bz | \bx, \bphi) = r(\bepsilon) \cdot \left| \frac{\partial \bepsilon}{\partial \bz}\right| \quad \Rightarrow \quad 
			\log q(\bz | \bx, \bphi) = \log r(\bepsilon) - \sum_{i=1}^d \log \left[\bsigma_{\bphi}(\bx) \right]_i
		\]
	\end{block}
	
\end{frame}
%=======
\begin{frame}{Variational autoencoder (VAE)}
	\begin{block}{Final algorithm}
		\begin{itemize}
			\item pick $i \sim U[1, n]$;
			\item compute a stochastic gradient w.r.t. $\bphi$
			\begin{multline*}
				\nabla_{\bphi} \mathcal{L} (\bphi, \btheta) \approx \nabla_{\bphi} \bigl[\log p(\bx, g(\bx, \bepsilon^*, \bphi) | \btheta)  - \\ - \log q\bigl(g(\bx, \bepsilon^*, \bphi) | \bx, \bphi \bigr)\bigr], \quad \bepsilon^* \sim r(\bepsilon);
			\end{multline*}
			\item compute a stochastic gradient w.r.t. $\btheta$
			\[
			\nabla_{\btheta} \mathcal{L} (\bphi, \btheta) \approx \nabla_{\btheta} \log p(\bx|\bz^*, \btheta), \quad \bz^* \sim q(\bz|\bx, \bphi);
			\]
			\item update $\btheta, \bphi$ according to the selected optimization method (SGD, Adam, RMSProp):
			\begin{align*}
				\bphi &:= \bphi + \eta \nabla_{\bphi} \mathcal{L}(\bphi, \btheta), \\
				\btheta &:= \btheta + \eta \nabla_{\btheta} \mathcal{L}(\bphi, \btheta).
			\end{align*}
		\end{itemize}
	\end{block}
\end{frame}
%=======
\begin{frame}{Variational autoencoder (VAE)}
	\begin{minipage}[t]{0.55\columnwidth}
	    \begin{itemize}
	    \item VAE learns stochastic mapping between $\bx$-space, from complicated distribution $\pi(\bx)$, and a latent $\bz$-space, with simple distribution. 
	    \item The generative model learns a joint distribution $p(\bx, \bz | \btheta) = p(\bz) p(\bx |\bz, \btheta)$, with a prior distribution $p(\bz)$, and a stochastic decoder $p(\bx|\bz, \btheta)$. 
	    \item The stochastic encoder $q(\bz|\bx, \bphi)$ (inference model), approximates the true but intractable posterior $p(\bz|\bx, \btheta)$ of the generative model.
	    \end{itemize}
	\end{minipage}%
	\begin{minipage}[t]{0.45\columnwidth}
		\begin{figure}[h]
			\centering
			\includegraphics[width=\linewidth]{figs/vae_scheme}
		\end{figure}
	\end{minipage}
	
	\myfootnotewithlink{https://arxiv.org/abs/1906.02691}{Kingma D. P., Welling M. An introduction to variational autoencoders, 2019}
\end{frame}
%=======
\begin{frame}{Variational Autoencoder}
	\begin{figure}[h]
		\centering
		\includegraphics[width=.7\linewidth]{figs/VAE.png}
	\end{figure}
	\myfootnotewithlink{http://ijdykeman.github.io/ml/2016/12/21/cvae.html}{image credit: http://ijdykeman.github.io/ml/2016/12/21/cvae.html}
\end{frame}
%=======
\begin{frame}{Variational autoencoder (VAE)}
	\begin{itemize}
		\item Encoder $q(\bz | \bx, \bphi) = \text{NN}_e(\bx, \bphi)$ outputs $\bmu_{\bphi}(\bx)$ and $\bsigma_{\bphi}(\bx)$.
		\item Decoder $p(\bx | \bz, \btheta) = \text{NN}_d(\bz, \btheta)$ outputs parameters of the sample distribution.
	\end{itemize}
	\begin{figure}[h]
		\centering
		\includegraphics[width=\linewidth]{figs/vae-gaussian.png}
	\end{figure}
	
	\myfootnotewithlink{https://lilianweng.github.io/lil-log/2018/08/12/from-autoencoder-to-beta-vae.html}{image credit: https://lilianweng.github.io/lil-log/2018/08/12/from-autoencoder-to-beta-vae.html}
\end{frame}
%=======
\begin{frame}{Variational Autoencoder}
	Generated images for latent objects $\bz$ sampled from prior $\mathcal{N}(0, \mathbf{I})$
	\begin{figure}[h]
		\centering
		\includegraphics[width=.5\linewidth]{figs/vae_0.png}
	\end{figure}
	\myfootnotewithlink{https://habr.com/ru/post/331664}{image credit: https://habr.com/ru/post/331664}
\end{frame}
%=======
\begin{frame}{Bayesian framework}
	\begin{block}{Posterior distribution}
		\vspace{-0.2cm}
		\[
			p(\btheta | \bX) = \frac{p(\bX | \btheta) p(\btheta)}{p(\bX)} = \frac{p(\bX | \btheta) p(\btheta)}{\int p(\bX | \btheta) p(\btheta) d \btheta} 
		\]
		\vspace{-0.2cm}
	\end{block}
	\begin{block}{Bayesian inference}
		\vspace{-0.3cm}
		\[
			p(\bx | \bX) = \int p(\bx | \btheta) p(\btheta | \bX) d \btheta
		\]
	\end{block}
    \begin{block}{Maximum a posteriori (MAP) estimation}
	    \vspace{-0.3cm}
	    \[
	        \btheta^* = \argmax_{\btheta} \log p(\btheta | \bX) = \argmax_{\btheta} \bigl(\log p(\bX | \btheta) + \log p(\btheta) \bigr)
	    \]
        \vspace{-0.3cm}
    \end{block}
	\begin{block}{MAP inference}
		\vspace{-0.4cm}
		\[
			p(\bx | \bX) = \int p(\bx| \btheta) p(\btheta | \bX ) d \btheta =  \int p(\bx| \btheta) \delta (\btheta - \btheta^* ) d \btheta \approx p(\bx | \btheta^*).
		\]
	\end{block}
\end{frame}
%=======
\begin{frame}{VAE as Bayesian model}
	\begin{block}{Posterior distribution}
		\vspace{-0.2cm}
		\[
			p(\btheta | \bX) = \frac{p(\bX | \btheta) p(\btheta)}{p(\bX)}
		\]
		\vspace{-0.5cm}
	\end{block}
	\begin{block}{ELBO}
		\vspace{-0.5cm}
		\begin{align*}
			 \log p(\btheta | \bX) &= \log p(\bX | \btheta) + \log p(\btheta) - \log p(\bX) \\
			 &= \cL(q, \btheta) + KL(q || p) + \log p(\btheta) - \log p(\bX) \\
			 &\geq \left[ \cL(q, \btheta) + \log p(\btheta) \right] - \log p(\bX) .
		\end{align*}
		\vspace{-0.7cm}
	\end{block}
	\begin{block}{EM-algorithm}
		\begin{itemize}
			\item E-step
				\vspace{-0.2cm}
				\[
					q(\bz) = \argmax_q \mathcal{L} (q, \btheta^*) = \argmin_q KL(q || p) =
				 p(\bz| \bx, \btheta^*);
				\]
				\vspace{-0.5cm}
			\item M-step
				\[
					\btheta^* = \argmax_{\btheta} \left[ \mathcal{L} (q, \btheta) + \log p(\btheta) \right].
				\]
		\end{itemize}
	\end{block}
\end{frame}
%=======
\begin{frame}{VAE limitations}
	\begin{itemize}
		\item Poor variational posterior distribution (inference model encoder)
		\[
		q(\bz | \bx, \bphi) = \mathcal{N}(\bz| \bmu_{\bphi}(\bx), \bsigma^2_{\bphi}(\bx)).
		\]
		\item Poor prior distribution
		\[
		p(\bz) = \mathcal{N}(0, \mathbf{I}).
		\]
		\item Poor probabilistic model (generative model, decoder)
		\[
		p(\bx | \bz, \btheta) = \mathcal{N}(\bx| \bmu_{\btheta}(\bz), \bsigma^2_{\btheta}(\bz)).
		\]
		\item Loose lower bound
		\[
		\log p(\bx | \btheta) - \mathcal{L}(q, \btheta) = (?).
		\]
	\end{itemize}
\end{frame}
%=======
\begin{frame}{Summary}
	\begin{itemize}
		\item Amortized inference allows to efficiently compute stochastic gradients for ELBO and to use deep neural networks for $q(\bz | \bx, \bphi)$ and $p(\bx | \bz, \btheta)$.
		\vfill
		\item ELBO gradients are computed using Monte-Carlo estimation.
		\vfill
		\item The reparametrization trick allows to get unbiased gradients w.r.t to a variational posterior distribution.
		\vfill
		\item The VAE model is an LVM with an encoder network for $q(\bz | \bx, \bphi)$ and a decoder network for $p(\bx | \bz, \btheta)$.
		\vfill
		\item VAE is not a "true" bayesian model since parameters $\btheta$ do not have a prior distribution.
		\vfill
		\item Standart VAE has several limitations that we will address later in the course.
	\end{itemize}
\end{frame}
%=======
\end{document} 