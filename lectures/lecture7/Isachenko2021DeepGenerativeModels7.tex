\documentclass{beamer}
\usepackage[utf8]{inputenc}
\usepackage{graphicx, epsfig}
\usepackage{amsmath,mathrsfs,amsfonts,amssymb}
%\usepackage{subfig}
\usepackage{floatflt}
\usepackage{epic,ecltree}
\usepackage{mathtext}
\usepackage{fancybox}
\usepackage{fancyhdr}
\usepackage{multirow}
\usepackage{enumerate}
\usepackage{epstopdf}
\usepackage{multicol}
\usepackage{algorithm}
\usepackage[noend]{algorithmic}
\usepackage{tikz}
\usepackage{blindtext}
\usetheme{default}%{Singapore}%{Warsaw}%{Warsaw}%{Darmstadt}
\usecolortheme{default}
\setbeamerfont{title}{size=\Huge}
\setbeamertemplate{footline}[page number]{}


\makeatletter
\newcommand\HUGE{\@setfontsize\Huge{35}{40}}
\makeatother    

\setbeamerfont{title}{size=\HUGE}
\beamertemplatenavigationsymbolsempty

\input{../utils/newcommands}
\input{../utils/title}

\newcommand\myfootnote[1]{%
  \tikz[remember picture,overlay]
  \draw (current page.south west) +(1in + \oddsidemargin,0.5em)
  node[anchor=south west,inner sep=0pt]{\parbox{\textwidth}{%
      \rlap{\rule{10em}{0.4pt}}\raggedright\scriptsize \textit{#1}}};}

\newcommand\myfootnotewithlink[2]{%
  \tikz[remember picture,overlay]
  \draw (current page.south west) +(1in + \oddsidemargin,0.5em)
  node[anchor=south west,inner sep=0pt]{\parbox{\textwidth}{%
      \rlap{\rule{10em}{0.4pt}}\raggedright\scriptsize\href{#1}{\textit{#2}}}};}
\createdgmtitle{7}
%--------------------------------------------------------------------------------
\begin{document}
%--------------------------------------------------------------------------------
\begin{frame}
%\thispagestyle{empty}
\titlepage
\end{frame}
%=======

\begin{frame}{Variational posterior}
\begin{block}{ELBO}
	\[
	\log p(\bx | \btheta) = \mathcal{L}(q, \btheta) + KL(q(\bz | \bx, \bphi) || p(\bz | \bx, \btheta)).
	\]
\end{block}
\begin{itemize}
	\item In E-step of EM-algorithm we wish $KL(q(\bz | \bx, \bphi) || p(\bz | \bx, \btheta)) = 0$. \\
	(In this case the lower bound is tight $\log p(\bx | \btheta) = \mathcal{L}(q, \btheta)$). \\
	\item Normal variational distribution $q(\bz | \bx, \bphi) = \mathcal{N}(\bz| \bmu_{\bphi}(\bx), \bsigma^2_{\bphi}(\bx))$ is poor (e.g. has only one mode). \\
	\item Flows models convert a simple base distribution to a complex one using invertible transformation with simple Jacobian. How to use flows in VAE?
\end{itemize}
\end{frame}
%=======
\begin{frame}{Flows in VAE posterior}
Apply a sequence of transformations to the random variable
\[
\bz_0 \sim q(\bz | \bx, \bphi) = \mathcal{N}(\bz| \bmu_{\bphi}(\bx), \bsigma^2_{\bphi}(\bx)).
\]
Here, $q(\bz | \bx, \bphi)$ (which is a VAE encoder) plays a role of a base distribution.
\[
\bz_0 \xrightarrow{g_1} \bz_1 \xrightarrow{g_2} \dots \xrightarrow{g_K} \bz_K, \quad \bz_K = g(\bz_0), \quad g = g_K \circ \dots \circ g_1.
\]
Each $g_k $ is a flow transformation (e.g. planar, coupling layer) parameterized by $\bphi_k$.
\begin{align*}
	\log q_K(\bz_K | \bx, \bphi, \{\bphi_k\}_{k=1}^K) &= \log q(\bz_0 | \bx, \bphi) \\ &- \sum_{k=1}^K \log \left| \det \left( \frac{\partial g_k(\bz_{k - 1}, \bphi_k)}{\partial \bz_{k-1}} \right) \right|.
\end{align*}

\myfootnotewithlink{https://arxiv.org/abs/1505.05770}{Rezende D. J., Mohamed S. Variational Inference with Normalizing Flows, 2015} 
\end{frame}
%=======
\begin{frame}{Flows in VAE posterior}
\begin{block}{ELBO}
	\vspace{-0.3cm}
	\[
	p(\bx | \btheta) \geq \mathcal{L} (\bphi, \btheta)  = \mathbb{E}_{q(\bz | \bx, \bphi)} \log \frac{p(\bx, \bz | \btheta)}{q(\bz| \bx, \bphi)} \rightarrow \max_{\bphi, \btheta}.
	\]
	\vspace{-0.5cm}
\end{block}
\begin{block}{Flow model in latent space}
	\vspace{-0.7cm}
	\[
	\log q_K(\bz_K | \bx, \bphi_*) = \log q(\bz_0 | \bx, \bphi) - \sum_{k=1}^K \log \left| \det \left( \frac{\partial g_k(\bz_{k - 1}, \bphi_k)}{\partial \bz_{k-1}} \right) \right|.
	\]
	\vspace{-0.5cm}
\end{block}
Let use $q_K(\bz_K | \bx, \bphi_*), \, \bphi_* = \{\bphi, \bphi_1, \dots, \bphi_K\}$ as a variational distribution. Here $\bphi$~-- encoder parameters, $\{\bphi_k\}_{k=1}^K$~-- flow parameters.

\begin{itemize}
	\item Encoder outputs base distribution $q(\bz_0 | \bx, \bphi)$.
	\item Flow model $\bz_K = g(\bz_0, \{\bphi_k\}_{k=1}^K)$ transforms the base distribution $q(\bz_0 | \bx, \bphi)$ to the distribution $q_K(\bz_K | \bx, \bphi_*)$.
	\item Distribution $q_K(\bz_K | \bx, \bphi_*)$ is used as a variational distribution for ELBO maximization.
\end{itemize}

\myfootnotewithlink{https://arxiv.org/abs/1505.05770}{Rezende D. J., Mohamed S. Variational Inference with Normalizing Flows, 2015} 
\end{frame}
%=======
\begin{frame}{Flows in VAE posterior}
\begin{block}{Flow model in latent space}
	\vspace{-0.7cm}
	\[
	\log q_K(\bz_K | \bx, \bphi_*) = \log q(\bz_0 | \bx, \bphi) - \sum_{k=1}^K \log \left| \det \left( \frac{\partial g_k(\bz_{k - 1}, \bphi_k)}{\partial \bz_{k-1}} \right) \right|.
	\]
	\vspace{-0.5cm}
\end{block}
\begin{block}{ELBO objective}
	\vspace{-0.5cm}
	\begin{align*}
		\mathcal{L} (\bphi, \btheta)  &= \mathbb{E}_{q_K(\bz_K | \bx, \bphi_*)} \log \frac{p(\bx, \bz_K | \btheta)}{q_K(\bz_K| \bx, \bphi_*)} \\
		&= \mathbb{E}_{q_K(\bz_K | \bx, \bphi_*)} \bigl[\log p(\bx, \bz_K | \btheta) - \log q_K(\bz_K| \bx, \bphi_*) \bigr] \\ 
		&=  \mathbb{E}_{q_K(\bz_K | \bx, \bphi_*)} \log p(\bx | \bz_K, \btheta) - KL (q_K(\bz_K| \bx, \bphi_*) || p(\bz_K)).
	\end{align*}
\end{block}
The second term in ELBO is reverse KL divergence. Planar flows was originally proposed for variational inference in VAE.
\myfootnotewithlink{https://arxiv.org/abs/1505.05770}{Rezende D. J., Mohamed S. Variational Inference with Normalizing Flows, 2015} 
\end{frame}
%=======
\begin{frame}{Flows in VAE posterior}
\begin{block}{Variational distribution}
	\vspace{-0.5cm}
	\[
	\log q_K(\bz_K | \bx, \bphi_*) = \log q(\bz_0 | \bx, \bphi) - \sum_{k=1}^K \log \left| \det \left( \frac{\partial g_k(\bz_{k - 1}, \bphi_k)}{\partial \bz_{k-1}} \right) \right|.
	\]
	\vspace{-0.5cm}
\end{block}
\begin{block}{ELBO objective}
	\vspace{-0.5cm}
	\begin{align*}
		\mathcal{L} (\bphi, \btheta) 
		&= \mathbb{E}_{q_K(\bz_K | \bx, \bphi_*)} \bigl[\log p(\bx, \bz_K | \btheta) - \log q_K(\bz_K| \bx, \bphi_*) \bigr] \\
		&= \mathbb{E}_{q(\bz_0 | \bx, \bphi)} \left. \bigl[\log p(\bx, \bz_K | \btheta) - \log q_K(\bz_K| \bx, \bphi_*) \bigr]\right|_{\bz_K = g(\bz_0, \{\bphi_k\}_{k=1}^K)} \\
		&= \mathbb{E}_{q(\bz_0 | \bx, \bphi)} \bigg[\log p(\bx, \bz_K | \btheta) -  \log q(\bz_0 | \bx, \bphi ) + \\ & \quad  + \sum_{k=1}^K \log \left| \det \left( \frac{\partial g_k(\bz_{k - 1}, \bphi_k)}{\partial \bz_{k-1}} \right) \right| \bigg].
	\end{align*}
\end{block}
\myfootnotewithlink{https://arxiv.org/abs/1505.05770}{Rezende D. J., Mohamed S. Variational Inference with Normalizing Flows, 2015} 
\end{frame}
%=======
\begin{frame}{Flows in VAE posterior}
\begin{block}{Variational distribution}
	\vspace{-0.6cm}
	\[
	\log q_K(\bz_K | \bx, \bphi_*) = \log q(\bz_0 | \bx, \bphi) - \sum_{k=1}^K \log \left| \det \left( \frac{\partial g_k(\bz_{k - 1}, \bphi_k)}{\partial \bz_{k-1}} \right) \right|.
	\]
	\vspace{-0.6cm}
\end{block}
\begin{block}{ELBO objective}
	\vspace{-0.7cm}
	\begin{align*}
		\mathcal{L} (\bphi, \btheta) 
		&= \mathbb{E}_{q(\bz_0 | \bx, \bphi)} \bigg[\log p(\bx, \bz_K | \btheta) -  \log q(\bz_0 | \bx, \bphi ) + \\ & \quad  + \sum_{k=1}^K \log \left| \det \left( \frac{\partial g_k(\bz_{k - 1}, \bphi_k)}{\partial \bz_{k-1}} \right) \right| \bigg].
	\end{align*}
	\vspace{-0.5cm}
\end{block}
\begin{itemize}
	\item Obtain samples $\bz_0$ from the encoder.
	\item Apply flow model $\bz_K = g(\bz_0, \{\bphi_k\}_{k = 1}^K)$.
	\item Compute likelihood for $\bz_K$ using the decoder, base distribution for $\bz_0$ and the Jacobian.
	\item We do not need inverse flow function, if we use flows in variational inference.
\end{itemize}
\myfootnotewithlink{https://arxiv.org/abs/1505.05770}{Rezende D. J., Mohamed S. Variational Inference with Normalizing Flows, 2015} 
\end{frame}
%=======
\begin{frame}{Recap of previous lecture}
	\begin{block}{ELBO}
		\vspace{-0.3cm}
		\[
			p(\bx | \btheta) \geq \mathcal{L} (\bphi, \btheta)  = \mathbb{E}_{q(\bz | \bx, \bphi)} \log \frac{p(\bx, \bz | \btheta)}{q(\bz| \bx, \bphi)} \rightarrow \max_{\bphi, \btheta}.
		\]
		\vspace{-0.5cm}
	\end{block}
		\begin{itemize}
			\item Normal variational distribution $q(\bz | \bx, \bphi) = \mathcal{N}(\bz| \bmu_{\bphi}(\bx), \bsigma^2_{\bphi}(\bx))$ is poor (e.g. has only one mode). \\
			\item Flows models convert a simple base distribution to a compex one using an invertible transformation with simple Jacobian. 
		\end{itemize}
	\begin{block}{Flow model in latent space}
		\vspace{-0.7cm}
		\[
			\log q_K(\bz_K | \bx, \bphi_*) = \log q(\bz_0 | \bx, \bphi) - \sum_{k=1}^K \log \left| \det \left( \frac{\partial g_k(\bz_{k - 1}, \bphi_k)}{\partial \bz_{k-1}} \right) \right|.
		\]
		\vspace{-0.5cm}
	\end{block}
	Let's use $q_K(\bz_K | \bx, \bphi_*), \, \bphi_* = \{\bphi, \bphi_1, \dots, \bphi_K\}$ as a variational distribution. Here, $\bphi$~-- encoder parameters, $\{\bphi_k\}_{k=1}^K$~-- flow parameters.
	
	\myfootnotewithlink{https://arxiv.org/abs/1505.05770}{Rezende D. J., Mohamed S. Variational Inference with Normalizing Flows, 2015} 
\end{frame}
%=======
\begin{frame}{Recap of previous lecture}
	\begin{block}{Variational distribution}
		\vspace{-0.6cm}
		\[
			\log q_K(\bz_K | \bx, \bphi_*) = \log q(\bz_0 | \bx, \bphi) - \sum_{k=1}^K \log \left| \det \left( \frac{\partial g_k(\bz_{k - 1}, \bphi_k)}{\partial \bz_{k-1}} \right) \right|.
		\]
		\vspace{-0.6cm}
	\end{block}
	\begin{block}{ELBO objective}
		\vspace{-0.7cm}
		\begin{align*}
			\mathcal{L} (\bphi, \btheta) 
			&= \mathbb{E}_{q(\bz_0 | \bx, \bphi)} \bigg[\log p(\bx, \bz_K | \btheta) -  \log q(\bz_0 | \bx, \bphi ) + \\ & \quad  + \sum_{k=1}^K \log \left| \det \left( \frac{\partial g_k(\bz_{k - 1}, \bphi_k)}{\partial \bz_{k-1}} \right) \right| \bigg].
		\end{align*}
		\vspace{-0.5cm}
	\end{block}
	\begin{itemize}
		\item Obtain samples $\bz_0$ from the encoder.
		\item Apply flow model $\bz_K = g(\bz_0, \{\bphi_k\}_{k = 1}^K)$.
		\item Compute likelihood for $\bz_K$ using the decoder, base distribution for $\bz_0$ and the Jacobian.
		\item We do not need an inverse flow function if we use flows in variational inference.
	\end{itemize}
	\myfootnotewithlink{https://arxiv.org/abs/1505.05770}{Rezende D. J., Mohamed S. Variational Inference with Normalizing Flows, 2015} 
\end{frame}
%=======
\begin{frame}{Inverse autoregressive flow (IAF)}
	\vspace{-0.3cm}
	\begin{align*}
		\bx &= g(\bz, \btheta) \quad \Rightarrow \quad x_i = \tilde{\sigma}_i (\bz_{1:i-1}) \cdot z_i + \tilde{\mu}_i(\bz_{1:i-1}). \\
		\bz &= f(\bx, \btheta) \quad \Rightarrow \quad z_i = \left(x_i - \tilde{\mu}_i(\bz_{1:i-1}) \right) \cdot \frac{1}{\tilde{\sigma}_i (\bz_{1:i-1})}.
	\end{align*}
	\vspace{-0.5cm}
	\begin{block}{Reverse KL for flow model}
  		\vspace{-0.5cm}
		\[
			KL(p || \pi)  = \bbE_{p(\bz)} \left[  \log p(\bz) - \log \left|\det \left( \frac{\partial g(\bz, \btheta)}{\partial \bz} \right) \right| - \log \pi(g(\bz, \btheta)) \right]
		\]
		\vspace{-0.3cm}
	\end{block}
	\begin{itemize}
	\item We don’t need to think about computing the function $f(\bx, \btheta)$.
	\item Inverse autoregressive flow is a natural choice for using flows in VAE:
	\end{itemize}
	\vspace{-0.3cm}
	\begin{align*}
		\bz_0 &= \bsigma(\bx) \odot \bepsilon + \bmu(\bx), \quad \bepsilon \sim \mathcal{N}(0, 1); \quad  \sim q(\bz_0 | \bx, \bphi). \\
		\bz_k &= \tilde{\bsigma}_k(\bz_{k - 1}) \odot \bz_{k-1} + \tilde{\bmu}_k(\bz_{k - 1}), \quad k \geq 1; \quad  \sim q_k(\bz_k | \bx, \bphi, \{\bphi_j\}_{j=1}^k).
	\end{align*}
	\myfootnotewithlink{https://arxiv.org/abs/1606.04934}{Kingma D. P. et al. Improving Variational Inference with Inverse Autoregressive Flow, 2016} 
\end{frame}
%=======
\begin{frame}{Inverse autoregressive flow (IAF)}
	\begin{figure}
		\includegraphics[width=\linewidth]{figs/iaf2.png}
	\end{figure}
	\begin{figure}
		\includegraphics[width=\linewidth]{figs/iaf1.png}
	\end{figure}

	\myfootnotewithlink{https://arxiv.org/abs/1606.04934}{Kingma D. P. et al. Improving Variational Inference with Inverse Autoregressive Flow, 2016} 
\end{frame}
%=======
\begin{frame}{Dequantization}
	\begin{itemize}
		\item Images are discrete data, pixels lie in the [0, 255] integer domain (the model is $P(\bx | \btheta) = \text{Categorical}(\bpi(\btheta))$).
		\item Flow is a continuous model (it works with continuous data $\bx$).
	\end{itemize}
	By fitting a continuous density model to discrete data, one can produce a degenerate solution with all probability mass on discrete values. \\
	How to convert a discrete data distribution to a continuous one?
	
	\begin{minipage}[t]{0.5\columnwidth}
		\begin{block}{Uniform dequantization}
		\vspace{-0.5cm}
			\begin{align*}
				\bx &\sim \text{Categorical}(\bpi) \\
				 \bu &\sim U[0, 1]
			\end{align*}
			\[
			\by = \bx + \bu \sim \text{Continuous} 
			\]
		\end{block}
	\end{minipage}%
	\begin{minipage}[t]{0.5\columnwidth}
		\begin{figure}
			\centering
			\includegraphics[width=1.0\linewidth]{figs/uniform_dequantization.png}
		\end{figure}
	\end{minipage}
	\myfootnotewithlink{https://arxiv.org/abs/1511.01844}{Theis L., Oord A., Bethge M. A note on the evaluation of generative models. 2015}
\end{frame}
%=======
\begin{frame}{Uniform dequantization}
	\begin{block}{Statement}
		Fitting continuous model $p(\by | \btheta)$ on uniformly dequantized data $\by = \bx + \bu, \, \bu \sim U[0, 1]$ is equivalent to maximization of a lower bound on log-likelihood for a discrete model:
		\[
		P(\bx | \btheta) = \int_{U[0, 1]} p(\bx + \bu | \btheta) d \bu
		\]
		\vspace{-0.2cm} \\
		Thus, the maximisation of continuous model log-likelihood on $\by$ can't lead to the a collapse onto the discrete data (the objective is bounded above by the discrete model log-likelihood).
	\end{block}
	\begin{block}{Proof}
		\vspace{-1cm}
		\begin{multline*}
			 \log P(\bx | \btheta) = \log \int_{U[0, 1]} p(\bx + \bu | \btheta) d \bu \geq \\ \geq \int_{U[0, 1]} \log p(\bx + \bu | \btheta) d \bu = \log p(\by | \btheta).
		\end{multline*}
	\end{block}
\end{frame}
%=======
\begin{frame}{Dequantization}
	\begin{itemize}
		\item Images are discrete data, pixels lie in the \{0, 255\} integer domain (the model is $P(\bx | \btheta) = \text{Categorical}(\bpi(\btheta))$).
		\item Flow is a continuous model (it works with continuous data $\bx$).
	\end{itemize}
	By fitting a continuous density model to discrete data, one can produce a degenerate solution with all probability mass on discrete values. \\
	How to convert a discrete data distribution to a continuous one?
	
	\begin{minipage}[t]{0.5\columnwidth}
		\begin{block}{Uniform dequantization}
		\vspace{-0.5cm}
			\begin{align*}
				\bx &\sim \text{Categorical}(\bpi) \\
				 \bu &\sim U[0, 1]
			\end{align*}
			\[
			\by = \bx + \bu \sim \text{Continuous} 
			\]
		\end{block}
	\end{minipage}%
	\begin{minipage}[t]{0.5\columnwidth}
		\begin{figure}
			\centering
			\includegraphics[width=1.0\linewidth]{figs/uniform_dequantization.png}
		\end{figure}
	\end{minipage}
	\myfootnotewithlink{https://arxiv.org/abs/1511.01844}{Theis L., Oord A., Bethge M. A note on the evaluation of generative models. 2015}
\end{frame}
%=======
\begin{frame}{Uniform dequantization}
	\begin{block}{Statement}
		Fitting continuous model $p(\by | \btheta)$ on uniformly dequantized data $\by = \bx + \bu, \, \bu \sim U[0, 1]$ is equivalent to maximization of a lower bound on log-likelihood for a discrete model:
		\vspace{-0.2cm}
		\[
		P(\bx | \btheta) = \int_{U[0, 1]} p(\bx + \bu | \btheta) d \bu
		\]
		\vspace{-0.5cm} 
	\end{block}
	\begin{block}{Proof}
		\vspace{-0.8cm}
		\begin{multline*}
			\bbE_{\pi} \log p(\by | \btheta) = \int \pi(\by) \log p(\by | \btheta) d \by = \\ 
			= \sum \pi(\bx) \int_{U[0,1]} \log p(\bx + \bu | \btheta) d \bu \leq \\
			 \leq \sum \pi(\bx) \log \int_{U[0,1]}  p(\bx + \bu | \btheta) d \bu = \\
			 = \sum \pi(\bx) \log P(\bx | \btheta) = \bbE_{\pi} \log P(\bx | \btheta).
		\end{multline*}
	\end{block}
	\myfootnotewithlink{https://arxiv.org/abs/1511.01844}{Theis L., Oord A., Bethge M. A note on the evaluation of generative models. 2015}
\end{frame}
%=======
\begin{frame}{Variational dequantization}
	\begin{minipage}[t]{0.5\columnwidth}
			\begin{figure}
				\centering
				\includegraphics[width=1.0\linewidth]{figs/uniform_dequantization.png}
			\end{figure}
	\end{minipage}%
	\begin{minipage}[t]{0.5\columnwidth}
		\begin{figure}
			\centering
			\includegraphics[width=1.0\linewidth]{figs/variational_dequantization.png}
		\end{figure}
	\end{minipage}
	\begin{itemize}
		\item $p(\by | \btheta)$ assign unifrom density to unit hypercubes $\bx + U[0, 1]$ (left fig).
		\item Neural network density models are smooth function approximators (right fig).
		\item Smooth dequantization is more natural.
	\end{itemize}
	How to perform the smooth dequantization? \\
\end{frame}
%=======
\begin{frame}{Flow++}
	\begin{block}{Variational dequantization}
		Introduce variational dequantization noise distribution $q(\bu | \bx)$ and treat it as an approximate posterior. 
	\end{block}
	\begin{block}{Variational lower bound}
		\vspace{-0.7cm}
		\begin{multline*}
		 \log P(\bx | \btheta) = \left[ \log \int q(\bu | \bx) \frac{p(\bx + \bu | \btheta)}{q(\bu | \bx)} d \bu \right] \geq \\ 
			\geq  \int q(\bu | \bx) \log \frac{p(\bx + \bu | \btheta)}{q(\bu | \bx)} d \bu = \mathcal{L}(q, \btheta).
		\end{multline*}
	\end{block}
	\vspace{-0.5cm}
	\begin{block}{Uniform dequantization bound}
		\vspace{-0.4cm}
		\[
		 \log P(\bx | \btheta) = \log \int_{U[0, 1]} p(\bx + \bu | \btheta) d \bu \geq \int_{U[0, 1]} \log p(\bx + \bu | \btheta) d \bu.
		\]
	\end{block}
	\myfootnotewithlink{https://arxiv.org/abs/1902.00275}{Ho J. et al. Flow++: Improving Flow-Based Generative Models with Variational Dequantization and Architecture Design, 2019}
\end{frame}
%=======
\begin{frame}{Flow++}
	\begin{block}{Variational lower bound}
		\[
		\mathcal{L}(q, \btheta) = \int q(\bu | \bx) \log \frac{p(\bx + \bu | \btheta)}{q(\bu | \bx)} d \bu.
		\]
	\end{block}
	Let $\bu = h(\bepsilon, \bphi)$ is a flow model with base distribution $\bepsilon \sim p(\bepsilon) = \mathcal{N}(0, \mathbf{I})$:
	\vspace{-0.3cm}
	\[
		q(\bu | \bx) = p(h^{-1}(\bu, \bphi)) \cdot \left| \det \frac{\partial h^{-1}(\bu, \bphi)}{\partial \bu}\right|.
	\]
	\vspace{-0.3cm}
	Then
	\[
		\log P(\bx | \btheta) \geq \cL(\bphi, \btheta) = \int p(\bepsilon) \log \left( \frac{p(\bx + h(\bepsilon, \bphi) | \btheta)}{p(\bepsilon) \cdot \left| \det \frac{\partial h(\bepsilon, \bphi)}{\partial \bepsilon}\right|^{-1}} \right) d\bepsilon.
	\]
	\myfootnotewithlink{https://arxiv.org/abs/1902.00275}{Ho J. et al. Flow++: Improving Flow-Based Generative Models with Variational Dequantization and Architecture Design, 2019}
\end{frame}
%=======
\begin{frame}{Flow++}
	\begin{block}{Variational lower}
	\vspace{-0.3cm}
	\[
		\log P(\bx | \btheta) \geq \int p(\bepsilon)\log \left( \frac{p(\bx + h(\bepsilon, \bphi))}{p(\bepsilon) \cdot \left| \det \frac{\partial h(\bepsilon, \bphi)}{\partial \bepsilon}\right|^{-1}} \right) d\bepsilon.
	\]
	\end{block}
	\begin{itemize}
	\item If $p(\bx + \bu | \btheta)$ is also a flow model, it is straightforward to calculate stochastic gradient of this ELBO.
	
	\item Uniform dequantization is a special case of variational dequantization ($q(\bu | \bx) = U[0, 1]$).
	The gap between $\log P(\bx | \btheta)$ and the derived ELBO is 
	$ KL(q(\bu | \bx) || p(\bu | \bx))$.
	\item In the case of uniform dequantization the model unnaturally places uniform density over each hypercube $\bx + U[0, 1]$ due to inexpressive distribution $q$.
	\end{itemize}
	\myfootnotewithlink{https://arxiv.org/abs/1902.00275}{Ho J. et al. Flow++: Improving Flow-Based Generative Models with Variational Dequantization and Architecture Design, 2019}
\end{frame}
%=======
\begin{frame}{Flow++}
	\begin{figure}
		\centering
		\includegraphics[width=0.7\linewidth]{figs/flow++1.png}
	\end{figure}
	\vspace{-0.1cm}
	\begin{figure}
		\centering
		\includegraphics[width=0.8\linewidth]{figs/flow++2.png}
	\end{figure}
	\myfootnotewithlink{https://arxiv.org/abs/1902.00275}{Ho J. et al. Flow++: Improving Flow-Based Generative Models with Variational Dequantization and Architecture Design, 2019}
\end{frame}
%=======
\begin{frame}{VAE limitations}
	\begin{itemize}
		\item Poor variational posterior distribution (encoder)
		\[
			q(\bz | \bx, \bphi) = \mathcal{N}(\bz| \bmu_{\bphi}(\bx), \bsigma^2_{\bphi}(\bx)).
		\]
		\item Poor prior distribution
		\[
			p(\bz) = \mathcal{N}(0, \mathbf{I}).
		\]
		\item Poor probabilistic model (decoder)
		\[
			p(\bx | \bz, \btheta) = \mathcal{N}(\bx| \bmu_{\btheta}(\bz), \bsigma^2_{\btheta}(\bz)).
		\]
		\item Loose lower bound
		\[
			\log p(\bx | \btheta) - \mathcal{L}(q, \btheta) = (?).
		\]
	\end{itemize}
\end{frame}
%=======
\begin{frame}{ELBO interpretations}
	\[
		\log p(\bx | \btheta) = \cL (\bphi, \btheta) + KL(q(\bz | \bx, \bphi) || p(\bz | \bx, \btheta)) \geq  \cL (\bphi, \btheta).
	\]
	\[
		\cL (\bphi, \btheta) = \int q(\bz | \bx, \bphi) \log \frac{p(\bx, \bz | \btheta)}{q(\bz | \bx, \bphi)} d \bz.
	\]
	\begin{itemize}
	    \item Evidence minus posterior KL
	    \vspace{-0.1cm}
	    \[
	        \mathcal{L}(q, \btheta) = \log p(\bx | \btheta) - KL(q(\bz | \bx, \bphi) || p(\bz | \bx, \btheta)).
	    \]
	    \item Average negative energy plus entropy
	    \vspace{-0.1cm}
	    \begin{align*}
	        \mathcal{L}(q, \btheta) &= \mathbb{E}_{q(\bz | \bx, \bphi)} \left[\log p(\bx, \bz | \btheta) - \log q(\bz | \bx, \bphi)  \right] \\
	        &= \mathbb{E}_{q(\bz | \bx, \bphi)} \log p(\bx, \bz | \btheta) + \mathbb{H} \left[q(\bz | \bx, \bphi) \right].
	    \end{align*}
	    \item Average reconstruction minus KL to prior
	    \vspace{-0.1cm}
	    \begin{align*}
	        \mathcal{L}(q, \btheta) &= \mathbb{E}_{q(\bz | \bx, \bphi)} \left[ \log p(\bx | \bz, \btheta) + \log p(\bz) - \log q(\bz | \bx, \bphi) \right] \\
	        &= \mathbb{E}_{q(\bz | \bx, \bphi)} \log p(\bx | \bz, \btheta) - KL(q(\bz | \bx, \bphi) || p(\bz)).
	    \end{align*}
	\end{itemize}
\end{frame}
%=======
\begin{frame}{ELBO surgery}
	\vspace{-0.3cm}
	\[
	    \frac{1}{n} \sum_{i=1}^n \mathcal{L}_i(q, \btheta) = \frac{1}{n} \sum_{i=1}^n \left[ \mathbb{E}_{q(\bz | \bx_i)} \log p(\bx_i | \bz, \btheta) - KL(q(\bz | \bx_i) || p(\bz)) \right].
	\]
	\vspace{-0.3cm}
	\begin{block}{Theorem}
		\[
		    \frac{1}{n} \sum_{i=1}^n KL(q(\bz | \bx_i) || p(\bz)) = KL(q(\bz) || p(\bz)) + \bbI_{q} [\bx, \bz],
		\]
		\begin{itemize}
			\item $q(\bz) = \frac{1}{n} \sum_{i=1}^n q(\bz | \bx_i)$ -- \textbf{aggregated} posterior distribution.
			\item $\bbI_{q} [\bx, \bz]$ -- mutual information between $\bx$ and $\bz$ under empirical data distribution and distribution $q(\bz | \bx)$.
			\item First term pushes $q(\bz)$ towards the prior $p(\bz)$.
			\item Second term reduces the amount of	information about $\bx$ stored in $\bz$. 
		\end{itemize}
	\end{block}
	\myfootnotewithlink{http://approximateinference.org/accepted/HoffmanJohnson2016.pdf}{Hoffman M. D., Johnson M. J. ELBO surgery: yet another way to carve up the variational evidence lower bound, 2016}
\end{frame}
%=======
\begin{frame}{ELBO surgery}
	\begin{block}{Theorem}
	\[
	    \frac{1}{n} \sum_{i=1}^n KL(q(\bz | \bx_i) || p(\bz)) = KL(q(\bz) || p(\bz)) + \bbI_q [\bx, \bz].
	\]
	\end{block}
	\begin{block}{Proof}
	\vspace{-0.3cm}
	{\footnotesize
	\begin{multline*}
	    \frac{1}{n} \sum_{i=1}^n KL(q(\bz | \bx_i) || p(\bz)) = \frac{1}{n} \sum_{i=1}^n \int q(\bz | \bx_i) \log \frac{q(\bz | \bx_i)}{p(\bz)} d \bz = \\
	    = \frac{1}{n} \sum_{i=1}^n \int q(\bz | \bx_i) \log \frac{q(\bz) q(\bz | \bx_i)}{p(\bz)q(\bz)} d \bz 
	    = \int \frac{1}{n} \sum_{i=1}^n  q(\bz | \bx_i) \log \frac{q(\bz)}{p(\bz)} d \bz + \\ 
	    + \frac{1}{n}\sum_{i=1}^n \int q(\bz | \bx_i) \log \frac{q(\bz | \bx_i)}{q(\bz)} d \bz = 
	     KL (q(\bz) || p(\bz)) + \frac{1}{n}\sum_{i=1}^n KL(q(\bz | \bx_i) || q (\bz))
	\end{multline*}
	}
	\end{block}

	\myfootnotewithlink{http://approximateinference.org/accepted/HoffmanJohnson2016.pdf}{Hoffman M. D., Johnson M. J. ELBO surgery: yet another way to carve up the variational evidence lower bound, 2016}
\end{frame}
%=======
\begin{frame}{ELBO surgery}
	\begin{block}{Theorem}
			\[
			    \frac{1}{n} \sum_{i=1}^n KL(q(\bz | \bx_i) || p(\bz)) = KL(q(\bz) || p(\bz)) + \bbI_{q} [\bx, \bz],
			\]
	\end{block}
	\begin{block}{Proof (continued)}
	\vspace{-0.5cm}
		\[
		    \frac{1}{n} \sum_{i=1}^n KL(q(\bz | \bx_i) || p(\bz_i)) = KL (q(\bz) || p(\bz)) + \frac{1}{n}\sum_{i=1}^n KL(q(\bz | \bx_i) || q (\bz))
		\]
		It could be shown (exercise):
		\[
			\bbI_{q} [\bx, \bz] = \frac{1}{n}\sum_{i=1}^n KL(q(\bz | \bx_i) || q (\bz)) \in [0, \log n].
		\]
	\end{block}

	\myfootnotewithlink{http://approximateinference.org/accepted/HoffmanJohnson2016.pdf}{Hoffman M. D., Johnson M. J. ELBO surgery: yet another way to carve up the variational evidence lower bound, 2016}
\end{frame}
%=======
\begin{frame}{Summary}
	\begin{itemize}
		\item Uniform dequantization is the simplest form of dequantization. 
		\vfill
		\item Variational dequantization is a more natural type that was proposed in Flow++ model.
		\vfill
		\item The IWAE could get the tighter lower bound to the likelihood, but the training of such model becomes more difficult.
		\vfill
		\item The ELBO surgery reveals insights about a prior distribution in VAE. The optimal prior is the aggregated posterior.
	\end{itemize}
\end{frame}

\end{document} 