\documentclass{beamer}
\usepackage[utf8]{inputenc}
\usepackage{graphicx, epsfig}
\usepackage{amsmath,mathrsfs,amsfonts,amssymb}
%\usepackage{subfig}
\usepackage{floatflt}
\usepackage{epic,ecltree}
\usepackage{mathtext}
\usepackage{fancybox}
\usepackage{fancyhdr}
\usepackage{multirow}
\usepackage{enumerate}
\usepackage{epstopdf}
\usepackage{multicol}
\usepackage{algorithm}
\usepackage[noend]{algorithmic}
\usepackage{tikz}
\usepackage{blindtext}
\usetheme{default}%{Singapore}%{Warsaw}%{Warsaw}%{Darmstadt}
\usecolortheme{default}
\setbeamerfont{title}{size=\Huge}
\setbeamertemplate{footline}[page number]{}


\makeatletter
\newcommand\HUGE{\@setfontsize\Huge{35}{40}}
\makeatother    

\setbeamerfont{title}{size=\HUGE}
\beamertemplatenavigationsymbolsempty

\input{../utils/newcommands}
\input{../utils/title}

\newcommand\myfootnote[1]{%
  \tikz[remember picture,overlay]
  \draw (current page.south west) +(1in + \oddsidemargin,0.5em)
  node[anchor=south west,inner sep=0pt]{\parbox{\textwidth}{%
      \rlap{\rule{10em}{0.4pt}}\raggedright\scriptsize \textit{#1}}};}

\newcommand\myfootnotewithlink[2]{%
  \tikz[remember picture,overlay]
  \draw (current page.south west) +(1in + \oddsidemargin,0.5em)
  node[anchor=south west,inner sep=0pt]{\parbox{\textwidth}{%
      \rlap{\rule{10em}{0.4pt}}\raggedright\scriptsize\href{#1}{\textit{#2}}}};}
\createdgmtitle{7}
%--------------------------------------------------------------------------------
\begin{document}
%--------------------------------------------------------------------------------
\begin{frame}[noframenumbering,plain]
%\thispagestyle{empty}
\titlepage
\end{frame}
%=======
\begin{frame}{Recap of previous lecture}
	\begin{block}{LVM}
		\vspace{-0.3cm}
		\[
		    p(\bx | \btheta) = \int p(\bx, \bz | \btheta) d \bz = \int p(\bx | \bz, \btheta) p(\bz) d \bz 
		\]
		\vspace{-0.3cm}
	\end{block}
	\begin{itemize}
		\item More powerful $p(\bx | \bz, \btheta)$ leads to more powerful generative model $p(\bx | \btheta)$.
		\item Too powerful $p(\bx | \bz, \btheta)$ could lead to posterior collapse: $q(\bz | \bx)$ will not carry any information about $\bx$ and close to prior $p(\bz)$.
	\end{itemize}
	\begin{block}{Autoregressive decoder}
		\vspace{-0.2cm}
		\[
		    p(\bx | \bz , \btheta) = \prod_{i=1}^n p(x_i | \bx_{1:i - 1}, \bz , \btheta)
		\]
	\end{block}
	\begin{itemize}
		\item Global structure is captured by latent variables.
		\item Local statistics are captured by limited receptive field autoregressive model.
	\end{itemize}
	
	\myfootnotewithlink{https://arxiv.org/abs/1611.05013}{Gulrajani I. et al. PixelVAE: A Latent Variable Model for Natural Images, 2016}
\end{frame}
%=======
\begin{frame}{Recap of previous lecture}
	\begin{block}{Decoder weakening}
		\begin{itemize}
			\item Powerful decoder $p(\bx | \bz, \btheta)$ makes the model expressive, but posterior collapse is possible.
			\item PixelVAE model uses the autoregressive PixelCNN model with small number of layers to limit receptive field.
		\end{itemize}
	\end{block}
	
	\begin{block}{KL annealing}
		\vspace{-0.2cm}
		\[
		    \mathcal{L}(q, \btheta, \beta) = \mathbb{E}_{q(\bz | \bx)} \log p(\bx | \bz, \btheta) - \beta \cdot KL (q(\bz | \bx) || p(\bz))
		\]
		Start training with $\beta = 0$, increase it until $\beta = 1$ during training.
	\end{block}
	
	\begin{block}{Free bits}
		Ensure the use of less than $\lambda$ bits of information:
		\[
		    \mathcal{L}(q, \btheta, \lambda) = \mathbb{E}_{q(\bz | \bx)} \log p(\bx | \bz, \btheta) - \max(\lambda, KL (q(\bz | \bx) || p(\bz))).
		\]
		This results in $KL (q(\bz | \bx) || p(\bz)) \geq \lambda$.
	\end{block}
\end{frame}
%=======
\begin{frame}{Recap of previous lecture}
	\begin{block}{VAE objective}
		\vspace{-0.2cm}
		\[
		\log p(\bx | \btheta) \geq \mathcal{L} (q, \btheta)  = \mathbb{E}_{q(\bz | \bx)} \log \frac{p(\bx, \bz | \btheta)}{q(\bz| \bx)} \rightarrow \max_{q, \btheta}
		\]
		\vspace{-0.2cm}
	\end{block}
	\begin{block}{IWAE objective}
		\vspace{-0.4cm}
		\[
		\mathcal{L}_K (q, \btheta)  = \mathbb{E}_{\bz_1, \dots, \bz_K \sim q(\bz | \bx)} \log \left( \frac{1}{K}\sum_{k=1}^K\frac{p(\bx, \bz_k | \btheta)}{q(\bz_k| \bx)} \right) \rightarrow \max_{q, \btheta}.
		\]
		\vspace{-0.4cm}
	\end{block}
	\begin{block}{Theorem}
		\begin{enumerate}
			\item $\log p(\bx | \btheta) \geq \mathcal{L}_K (q, \btheta) \geq \mathcal{L}_M (q, \btheta) \geq \mathcal{L} (q, \btheta), \quad \text{for } K \geq M$;
			\item $\log p(\bx | \btheta) = \lim_{K \rightarrow \infty} \mathcal{L}_K (q, \btheta)$ if $\frac{p(\bx, \bz | \btheta)}{q(\bz | \bx)}$ is bounded.
		\end{enumerate}
	\end{block}
	
	\myfootnotewithlink{https://arxiv.org/abs/1509.00519}{Burda Y., Grosse R., Salakhutdinov R. Importance Weighted Autoencoders, 2015}
\end{frame}
%=======

\begin{frame}{IWAE}
	\begin{block}{Theorem}
		\begin{enumerate}
			\item $\log p(\bx | \btheta) \geq \mathcal{L}_K (q, \btheta) \geq \mathcal{L}_M (q, \btheta), \quad \text{for } K \geq M$;
			\item $\log p(\bx | \btheta) = \lim_{K \rightarrow \infty} \mathcal{L}_K (q, \btheta)$ if $\frac{p(\bx, \bz | \btheta)}{q(\bz | \bx)}$ is bounded.
		\end{enumerate}
		\vspace{-0.2cm}
	\end{block}
	\begin{block}{Proof of 1.}
		{ \footnotesize
			\vspace{-0.6cm}
			\begin{align*}
				\mathcal{L}_K (q, \btheta) &= \mathbb{E}_{\bz_1, \dots, \bz_K} \log \left( \frac{1}{K}\sum_{k=1}^K\frac{p(\bx, \bz_k | \btheta)}{q(\bz_k| \bx)} \right) = \\
				&= \mathbb{E}_{\bz_1, \dots, \bz_K} \log \mathbb{E}_{k_1, \dots, k_M} \left( \frac{1}{M}\sum_{m=1}^M\frac{p(\bx, \bz_{k_M} | \btheta)}{q(\bz_{k_m}| \bx)} \right) \geq \\
				&\geq \mathbb{E}_{\bz_1, \dots, \bz_K} \mathbb{E}_{k_1, \dots, k_m} \log \left( \frac{1}{M}\sum_{m=1}^M\frac{p(\bx, \bz_{k_m} | \btheta)}{q(\bz_{k_m}| \bx)} \right) = \\
				&= \mathbb{E}_{\bz_1, \dots, \bz_M} \log \left( \frac{1}{M}\sum_{m=1}^M\frac{p(\bx, \bz_m | \btheta)}{q(\bz_m| \bx)} \right) = \mathcal{L}_M (q, \btheta)
			\end{align*}
			\[
			\frac{a_1 + \dots + a_K}{K} = \mathbb{E}_{k_1, \dots, k_M} \frac{a_{k_1} + \dots + a_{k_M}}{M}, \quad k_1, \dots, k_M \sim U[1, K]
			\]
		}
	\end{block}
	
	\myfootnotewithlink{https://arxiv.org/abs/1509.00519}{Burda Y., Grosse R., Salakhutdinov R. Importance Weighted Autoencoders, 2015}
\end{frame}
%=======
\begin{frame}{IWAE}
	\begin{block}{Theorem}
		\begin{enumerate}
			\item $\log p(\bx | \btheta) \geq \mathcal{L}_K (q, \btheta) \geq \mathcal{L}_M (q, \btheta), \quad \text{for} K \geq M$;
			\item $\log p(\bx | \btheta) = \lim_{K \rightarrow \infty} \mathcal{L}_K (q, \btheta)$ if $\frac{p(\bx, \bz | \btheta)}{q(\bz | \bx)}$ is bounded.
		\end{enumerate}
		\vspace{-0.2cm}
	\end{block}
	\begin{block}{Proof of 2.}
		\vspace{0.2cm}
		Consider r.v. $\xi_K = \frac{1}{K}\sum_{k=1}^K \frac{p(\bx, \bz_k | \btheta)}{q(\bz_k | \bx)}$. \\
		\vspace{0.2cm}
		If summands are bounded, then (from the strong law of large numbers)
		\[
		\xi_K \xrightarrow[K \rightarrow \infty]{a.s.} \mathbb{E}_{q(\bz | \bx)} \frac{p(\bx, \bz | \btheta)}{q(\bz | \bx)} = p(\bx | \btheta).
		\]
		Hence $\mathcal{L}_K (q, \btheta) = \mathbb{E} \log \xi_K$ converges to $\log p(\bx | \btheta)$ as $K \rightarrow \infty$.
	\end{block}

	\myfootnotewithlink{https://arxiv.org/abs/1509.00519}{Burda Y., Grosse R., Salakhutdinov R. Importance Weighted Autoencoders, 2015}
\end{frame}
%=======
\begin{frame}{IWAE}
	\[
	\log p(\bx | \btheta) \geq \mathcal{L}_K(q, \btheta) \geq \mathcal{L}(q, \btheta)
	\]
	If $K > 1$ the bound could be tighter.
	\begin{align*}
		\mathcal{L} (q, \btheta) &= \mathbb{E}_{q(\bz | \bx)} \log \frac{p(\bx, \bz | \btheta)}{q(\bz| \bx)}; \\
		\mathcal{L}_K (q, \btheta) &= \mathbb{E}_{\bz_1, \dots, \bz_K \sim q(\bz | \bx)} \log \left( \frac{1}{K}\sum_{k=1}^K\frac{p(\bx, \bz_k | \btheta)}{q(\bz_k| \bx)} \right).
	\end{align*}
	\vspace{-0.2cm}
	\begin{itemize}
		\item $\mathcal{L}_1(q, \btheta) = \mathcal{L}(q, \btheta)$;
		\item $\mathcal{L}_{\infty}(q, \btheta) = \log p(\bx | \btheta)$.
		\item Which $q^*(\bz | \bx)$ gives $\mathcal{L}(q^*, \btheta) = \log p(\bx | \btheta)$? 
		\item Which $q^*(\bz | \bx)$ gives $\mathcal{L}(q^*, \btheta) = \mathcal{L}_K(q, \btheta)$?
	\end{itemize}

	\myfootnotewithlink{https://arxiv.org/abs/1509.00519}{Burda Y., Grosse R., Salakhutdinov R. Importance Weighted Autoencoders, 2015}
\end{frame}
%=======
\begin{frame}{IWAE}
	\begin{block}{Theorem}
		$\mathcal{L}(q^*, \btheta) = \mathcal{L}_K(q, \btheta)$
		for the following variational distribution
		\[
		q^*(\bz | \bx) = \mathbb{E}_{\bz_2, \dots, \bz_K \sim q(\bz | \bx)} q_{IW}(\bz | \bx, \bz_{2:K}),
		\]
		where
		\vspace{-0.4cm}
		\[
			q_{IW}(\bz | \bx, \bz_{2:K}) = \frac{\frac{p(\bx, \bz)}{q(\bz | \bx)}}{\frac{1}{K} \sum_{k=1}^K \frac{p(\bx, \bz_k)}{q(\bz_k | \bx)}} q(\bz | \bx) = \frac{p(\bx, \bz)}{\frac{1}{K}\left( \frac{p(\bx, \bz)}{q(\bz | \bx)} + \sum_{k=2}^K \frac{p(\bx, \bz_k)}{q(\bz_k | \bx)}\right)}.
		\]
	\end{block}
	\vspace{-0.5cm}
	\begin{block}{IWAE posterior}
		\vspace{-0.3cm}
		\begin{figure}
			\centering
			\includegraphics[width=\linewidth]{figs/IWAE_1.png}
		\end{figure}
	\end{block}

	\myfootnotewithlink{https://arxiv.org/abs/1704.02916}{Cremer C., Morris Q., Duvenaud D. Reinterpreting Importance-Weighted Autoencoders, 2017}
\end{frame}
%=======
\begin{frame}{IWAE}
	\begin{block}{Objective}
		\vspace{-0.5cm}
		\[
		\mathcal{L}_K (q, \btheta)  = \mathbb{E}_{\bz_1, \dots, \bz_K \sim q(\bz | \bx, \bphi)} \log \left( \frac{1}{K}\sum_{k=1}^K\frac{p(\bx, \bz_k | \btheta)}{q(\bz_k| \bx, \bphi)} \right) \rightarrow \max_{\bphi, \btheta}.
		\]
		\vspace{-0.3cm}
	\end{block}
	\begin{block}{Gradient}
		\vspace{-0.3cm}
		\[
		\Delta_K = \nabla_{\btheta, \bphi} \log \left( \frac{1}{K}\sum_{k=1}^K\frac{p(\bx, \bz_k | \btheta)}{q(\bz_k| \bx, \bphi)} \right), \quad \bz_k \sim q(\bz | \bx, \bphi).
		\]
		\vspace{-0.3cm}
	\end{block}
	\begin{block}{Theorem}
		\vspace{-0.4cm}
		\[
			\text{SNR}_K = \frac{\bbE [\Delta_K]}{\sigma(\Delta_K)}; \quad
			\text{SNR}_K(\btheta) = O(\sqrt{K}); \quad 
			\text{SNR}_K(\bphi) = O\left(\sqrt{\frac{1}{K}}\right).
		\]
		Hence, increasing $K$ vanishes gradient signal of inference network $q(\bz | \bx, \bphi)$.
	\end{block}

	\myfootnotewithlink{https://arxiv.org/abs/1802.04537}{Rainforth T. et al. Tighter variational bounds are not necessarily better, 2018}
\end{frame}
%=======
\begin{frame}{IWAE}
	\begin{block}{Theorem}
		\vspace{-0.5cm}
		\[
			\text{SNR}_K = \frac{\bbE [\Delta_K]}{\sigma(\Delta_K)}; \quad
			\text{SNR}_K(\btheta) = O(\sqrt{K}); \quad 
			\text{SNR}_K(\bphi) = O\left(\sqrt{\frac{1}{K}}\right).
		\]
		\vspace{-0.8cm}
	\end{block}
		\begin{minipage}[t]{0.5\columnwidth}
			\begin{figure}[h]
				\centering
				\includegraphics[width=1.\linewidth]{figs/IWAE_SNR_1.png}
			\end{figure}
		\end{minipage}%
		\begin{minipage}[t]{0.5\columnwidth}
			\begin{figure}[h]
				\centering
				\includegraphics[width=1.\linewidth]{figs/IWAE_SNR_2.png}
			\end{figure}
		\end{minipage}
	\begin{itemize}
		\item IWAE makes the variational bound tighter and extends the class of variational distributions.
		\item Gradient signal becomes really small, training is complicated.
		\item IWAE is very popular technique as a quality measure for VAE models.
	\end{itemize}
	\myfootnotewithlink{https://arxiv.org/abs/1802.04537}{Rainforth T. et al. Tighter variational bounds are not necessarily better, 2018}
\end{frame}
%=======
\begin{frame}{VAE limitations}
	\begin{itemize}
		\item Poor variational posterior distribution (encoder)
		\[
			q(\bz | \bx, \bphi) = \mathcal{N}(\bz| \bmu_{\bphi}(\bx), \bsigma^2_{\bphi}(\bx)).
		\]
		\item Poor prior distribution
		\[
			p(\bz) = \mathcal{N}(0, \mathbf{I}).
		\]
		\item Poor probabilistic model (decoder)
		\[
			p(\bx | \bz, \btheta) = \mathcal{N}(\bx| \bmu_{\btheta}(\bz), \bsigma^2_{\btheta}(\bz)).
		\]
		\item Loose lower bound
		\[
			\log p(\bx | \btheta) - \mathcal{L}(q, \btheta) = (?).
		\]
	\end{itemize}
\end{frame}
%=======
\begin{frame}{ELBO interpretations}
	\[
		\log p(\bx | \btheta) = \cL (\bphi, \btheta) + KL(q(\bz | \bx, \bphi) || p(\bz | \bx, \btheta)) \geq  \cL (\bphi, \btheta).
	\]
	\[
		\cL (\bphi, \btheta) = \int q(\bz | \bx, \bphi) \log \frac{p(\bx, \bz | \btheta)}{q(\bz | \bx, \bphi)} d \bz.
	\]
	\begin{itemize}
	    \item Evidence minus posterior KL
	    \[
	        \mathcal{L}(\bphi, \btheta) = \log p(\bx | \btheta) - KL(q(\bz | \bx, \bphi) || p(\bz | \bx, \btheta)).
	    \]
	    \item Average reconstruction loss with regularizer (prior KL)
	    \begin{align*}
	        \mathcal{L}(\bphi, \btheta) &= \mathbb{E}_{q(\bz | \bx, \bphi)} \left[ \log p(\bx | \bz, \btheta) + \log p(\bz) - \log q(\bz | \bx, \bphi) \right] \\
	        &= \mathbb{E}_{q(\bz | \bx, \bphi)} \log p(\bx | \bz, \btheta) - KL(q(\bz | \bx, \bphi) || p(\bz)).
	    \end{align*}
	\end{itemize}
\end{frame}
%=======
\begin{frame}{ELBO surgery}
	\vspace{-0.3cm}
	\[
	    \frac{1}{n} \sum_{i=1}^n \mathcal{L}_i(q, \btheta) = \frac{1}{n} \sum_{i=1}^n \left[ \mathbb{E}_{q(\bz | \bx_i)} \log p(\bx_i | \bz, \btheta) - KL(q(\bz | \bx_i) || p(\bz)) \right].
	\]
	\vspace{-0.3cm}
	\begin{block}{Theorem}
		\[
		    \frac{1}{n} \sum_{i=1}^n KL(q(\bz | \bx_i) || p(\bz)) = KL(q(\bz) || p(\bz)) + \bbI_{q} [\bx, \bz],
		\]
		\begin{itemize}
			\item $q(\bz) = \frac{1}{n} \sum_{i=1}^n q(\bz | \bx_i)$ -- \textbf{aggregated} posterior distribution.
			\item $\bbI_{q} [\bx, \bz]$ -- mutual information between $\bx$ and $\bz$ under empirical data distribution and distribution $q(\bz | \bx)$.
			\item First term pushes $q(\bz)$ towards the prior $p(\bz)$.
			\item Second term reduces the amount of	information about $\bx$ stored in $\bz$. 
		\end{itemize}
	\end{block}
	\myfootnotewithlink{http://approximateinference.org/accepted/HoffmanJohnson2016.pdf}{Hoffman M. D., Johnson M. J. ELBO surgery: yet another way to carve up the variational evidence lower bound, 2016}
\end{frame}
%=======
\begin{frame}{ELBO surgery}
	\begin{block}{Theorem}
		\vspace{-0.3cm}
		\[
		    \frac{1}{n} \sum_{i=1}^n KL(q(\bz | \bx_i) || p(\bz)) = KL(q(\bz) || p(\bz)) + \bbI_q [\bx, \bz].
		\]
		\vspace{-0.4cm}
	\end{block}
	\begin{block}{Proof}
		\vspace{-0.5cm}
		{\footnotesize
		\begin{multline*}
		    \frac{1}{n} \sum_{i=1}^n KL(q(\bz | \bx_i) || p(\bz)) = \frac{1}{n} \sum_{i=1}^n \int q(\bz | \bx_i) \log \frac{q(\bz | \bx_i)}{p(\bz)} d \bz = \\
		    = \frac{1}{n} \sum_{i=1}^n \int q(\bz | \bx_i) \log \frac{q(\bz) q(\bz | \bx_i)}{p(\bz)q(\bz)} d \bz 
		    = \int \frac{1}{n} \sum_{i=1}^n  q(\bz | \bx_i) \log \frac{q(\bz)}{p(\bz)} d \bz + \\ 
		    + \frac{1}{n}\sum_{i=1}^n \int q(\bz | \bx_i) \log \frac{q(\bz | \bx_i)}{q(\bz)} d \bz = 
		     KL (q(\bz) || p(\bz)) + \frac{1}{n}\sum_{i=1}^n KL(q(\bz | \bx_i) || q (\bz))
		\end{multline*}
		}
		Without proof:
		\vspace{-0.4cm}
		\[
			\bbI_{q} [\bx, \bz] = \frac{1}{n}\sum_{i=1}^n KL(q(\bz | \bx_i) || q (\bz)) \in [0, \log n].
		\]
	\end{block}

	\myfootnotewithlink{http://approximateinference.org/accepted/HoffmanJohnson2016.pdf}{Hoffman M. D., Johnson M. J. ELBO surgery: yet another way to carve up the variational evidence lower bound, 2016}
\end{frame}
%=======
\begin{frame}{ELBO surgery}
	\begin{block}{ELBO revisiting}
		\vspace{-0.7cm}
		\begin{multline*}
		    \frac{1}{n}\sum_{i=1}^n \cL_i(q, \btheta) = \frac{1}{n} \sum_{i=1}^n \left[ \mathbb{E}_{q(\bz | \bx_i)} \log p(\bx_i | \bz, \btheta) - KL(q(\bz | \bx_i) || p(\bz)) \right] = \\
		    = \underbrace{\frac{1}{n} \sum_{i=1}^n \mathbb{E}_{q(\bz | \bx_i)} \log p(\bx_i | \bz, \btheta)}_{\text{Reconstruction loss}} - \underbrace{\vphantom{ \sum_{i=1}^n} \bbI_q [\bx, \bz]}_{\text{MI}} - \underbrace{\vphantom{ \sum_{i=1}^n} KL(q(\bz) || p(\bz))}_{\text{Marginal KL}}
		\end{multline*}
		\vspace{-0.3cm}
	\end{block}
	Prior distribution $p(\bz)$ is only in the last term.
	\begin{block}{Optimal VAE prior}
		\vspace{-0.3cm}
		\[
	  		KL(q(\bz) || p(\bz)) = 0 \quad \Leftrightarrow \quad p (\bz) = q(\bz) = \frac{1}{n} \sum_{i=1}^n q(\bz | \bx_i).
		\]
		The optimal prior $p(\bz)$ is the aggregated posterior $q(\bz)$.
	\end{block}
	
	\myfootnotewithlink{http://approximateinference.org/accepted/HoffmanJohnson2016.pdf}{Hoffman M. D., Johnson M. J. ELBO surgery: yet another way to carve up the variational evidence lower bound, 2016}
\end{frame}
%=======
\begin{frame}{Optimal VAE prior}
	How to choose the optimal $p(\bz)$?
	\begin{itemize}
		\item Standard Gaussian $p(\bz) = \mathcal{N}(0, I)$ $\Rightarrow$ over-regularization;
		\item $p(\bz) = q(\bz) = \frac{1}{n}\sum_{i=1}^n q(\bz | \bx_i)$ $\Rightarrow$ overfitting and highly expensive.
	\end{itemize}
	\vspace{-0.3cm}
	\begin{minipage}[t]{0.5\columnwidth}
		\begin{block}{Non learnable prior $p(\bz)$}
			\begin{figure}[h]
				\centering
				\includegraphics[width=0.8\linewidth]{figs/non_learnable_prior}
			\end{figure}
		\end{block}
	\end{minipage}%
	\begin{minipage}[t]{0.5\columnwidth}
		\begin{block}{Learnable prior $p(\bz | \blambda)$}
			\begin{figure}[h]
				\centering
				\includegraphics[width=0.8\linewidth]{figs/learnable_prior}
			\end{figure}
		\end{block}
	\end{minipage}
	\myfootnotewithlink{https://jmtomczak.github.io/blog/7/7\_priors.html}{image credit: https://jmtomczak.github.io/blog/7/7\_priors.html}
\end{frame}
%=======
\begin{frame}{Learnable VAE prior}
	\begin{block}{Optimal prior}
		\vspace{-0.4cm}
		\[
			KL(q(\bz) || p(\bz)) = 0 \quad \Leftrightarrow \quad p (\bz) = q(\bz) = \frac{1}{n} \sum_{i=1}^n q(\bz | \bx_i).
		\]
		\vspace{-0.4cm}
	\end{block}
	\begin{block}{Mixture of Gaussians}
		\vspace{-0.3cm}
		\[
			p(\bz | \blambda) = \sum_{k=1}^K w_k \cN(\bz | \bmu_k, \bsigma_k^2), \quad \blambda = \{w_k, \bmu_k, \bsigma_k\}_{k=1}^K.
		\]
		\vspace{-0.5cm}
	\end{block}
	\begin{block}{Variational Mixture of posteriors (VampPrior)}
		\vspace{-0.3cm}
		\[
		p(\bz | \blambda) = \frac{1}{K} \sum_{k=1}^K q(\bz | \bu_k),
		\]
		where $\blambda = \{\bu_1, \dots, \bu_K\}$ are trainable pseudo-inputs.
	\end{block}
	\begin{itemize}
		\item Multimodal $\Rightarrow$ prevents over-regularization;.
		\item $K \ll n$ $\Rightarrow$ prevents from potential overfitting + less expensive to train.
	\end{itemize}
	\myfootnotewithlink{https://arxiv.org/abs/1705.07120}{Tomczak J. M., Welling M. VAE with a VampPrior, 2017}
\end{frame}
%=======
\begin{frame}{VampPrior}
	\begin{itemize}
	\item Do we really need the multimodal prior?
	\item Is it beneficial to couple the prior with the variational posterior or the MoG prior is enough?
	\end{itemize}
	\begin{block}{Results}
		\vspace{-0.3cm}
		\begin{minipage}[t]{0.55\columnwidth}
			\begin{figure}[h]
				\centering
				\includegraphics[width=1.0\linewidth]{figs/VampPrior_1.png}
			\end{figure}
		\end{minipage}%
		\begin{minipage}[t]{0.45\columnwidth}
			\begin{figure}[h]
				\centering
				\includegraphics[width=1.0\linewidth]{figs/VampPrior_2.png}
			\end{figure}
		\end{minipage}
	\end{block}
	\textbf{Top row:} generated images by PixelHVAE + VampPrior for chosen pseudo-input in the left top corner. \\
	\vspace{0.1cm}
	\textbf{Bottom row:} pseudo-inputs for different datasets.
	\myfootnotewithlink{https://arxiv.org/abs/1705.07120}{Tomczak J. M., Welling M. VAE with a VampPrior, 2017}
\end{frame}
%=======
\begin{frame}{Flows-based VAE prior}
	\begin{block}{Flow model in latent space}
		\vspace{-0.5cm}
		\[
			\log p(\bz | \blambda) = \log p(\bepsilon) + \log \det \left | \frac{d \bepsilon}{d\bz}\right| = \log p(\bepsilon) + \log \det \left | \frac{\partial f(\bz, \blambda)}{\partial \bz}\right| 
		\]
		\[
			\bz = g(\bepsilon, \blambda) = f^{-1}(\bepsilon, \blambda)
		\]
	\end{block}
	\vspace{-0.3cm}
	\begin{itemize}
		\item RealNVP flow.
		\item Autoregressive flow (MAF).
	\end{itemize}
	Why it is not a good idea to use IAF for VAE prior?
	\begin{block}{ELBO with flow-based VAE prior}
		\vspace{-0.5cm}
		{\footnotesize
		\begin{multline*}
			\mathcal{L}(\bphi, \btheta) = \mathbb{E}_{q(\bz | \bx, \bphi)} \left[ \log p(\bx | \bz, \btheta) +  \log p(\bz | \blambda) - \log q(\bz | \bx, \bphi) \right] \\
				= \mathbb{E}_{q(\bz | \bx, \bphi)} \left[ \log p(\bx | \bz, \btheta) + \underbrace{ \Bigl( \log p(f(\bz, \blambda)) + \log \left| \det \frac{\partial f(\bz, \blambda)}{\partial \bz} \right| \Bigr) }_{\text{flow-based prior}} - \log q(\bz | \bx, \bphi) \right] 
		\end{multline*}
		}
	\end{block}
	\myfootnotewithlink{https://arxiv.org/abs/1611.02731}{Chen X. et al. Variational Lossy Autoencoder, 2016}
\end{frame}
%=======
\begin{frame}{VAE limitations}
	\begin{itemize}
		\item Poor variational posterior distribution (encoder)
		\[
			q(\bz | \bx, \bphi) = \mathcal{N}(\bz| \bmu_{\bphi}(\bx), \bsigma^2_{\bphi}(\bx)).
		\]
		\item Poor prior distribution
		\[
			p(\bz) = \mathcal{N}(0, \mathbf{I}).
		\]
		\item Poor probabilistic model (decoder)
		\[
			p(\bx | \bz, \btheta) = \mathcal{N}(\bx| \bmu_{\btheta}(\bz), \bsigma^2_{\btheta}(\bz)).
		\]
		\item Loose lower bound
		\[
			\log p(\bx | \btheta) - \mathcal{L}(q, \btheta) = (?).
		\]
	\end{itemize}
\end{frame}
%=======
\begin{frame}{Variational posterior}
	\begin{block}{ELBO}
		\[
		\log p(\bx | \btheta) = \mathcal{L}(q, \btheta) + KL(q(\bz | \bx, \bphi) || p(\bz | \bx, \btheta)).
		\]
	\end{block}
	\begin{itemize}
		\item In E-step of EM-algorithm we wish $KL(q(\bz | \bx, \bphi) || p(\bz | \bx, \btheta)) = 0$. \\
		(In this case the lower bound is tight $\log p(\bx | \btheta) = \mathcal{L}(q, \btheta)$). \\
		\item Normal variational distribution $q(\bz | \bx, \bphi) = \mathcal{N}(\bz| \bmu_{\bphi}(\bx), \bsigma^2_{\bphi}(\bx))$ is poor (e.g. has only one mode). \\
		\item Flows models convert a simple base distribution to a complex one using invertible transformation with simple Jacobian. How to use flows in VAE posterior?
	\end{itemize}
\end{frame}
%=======
\begin{frame}{Summary}
	\begin{itemize}
		\item The IWAE could get the tighter lower bound to the likelihood, but the training of such model becomes more difficult.
		\vfill
		\item The ELBO surgery reveals insights about a prior distribution in VAE. The optimal prior is the aggregated posterior.
		\vfill
		\item VampPrior proposes to use a variational mixture of posteriors as the prior to approximate the aggregated posterior.
		\vfill
		\item We could use flow-based prior in VAE (moreover, autoregressive) as well as flow-based posterior (next lecture).
	\end{itemize}
\end{frame}

\end{document} 