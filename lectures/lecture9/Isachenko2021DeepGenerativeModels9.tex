\documentclass{beamer}
\usepackage[utf8]{inputenc}
\usepackage{graphicx, epsfig}
\usepackage{amsmath,mathrsfs,amsfonts,amssymb}
%\usepackage{subfig}
\usepackage{floatflt}
\usepackage{epic,ecltree}
\usepackage{mathtext}
\usepackage{fancybox}
\usepackage{fancyhdr}
\usepackage{multirow}
\usepackage{enumerate}
\usepackage{epstopdf}
\usepackage{multicol}
\usepackage{algorithm}
\usepackage[noend]{algorithmic}
\usepackage{tikz}
\usepackage{blindtext}
\usetheme{default}%{Singapore}%{Warsaw}%{Warsaw}%{Darmstadt}
\usecolortheme{default}
\setbeamerfont{title}{size=\Huge}
\setbeamertemplate{footline}[page number]{}


\makeatletter
\newcommand\HUGE{\@setfontsize\Huge{35}{40}}
\makeatother    

\setbeamerfont{title}{size=\HUGE}
\beamertemplatenavigationsymbolsempty

\input{../utils/newcommands}
\input{../utils/title}

\newcommand\myfootnote[1]{%
  \tikz[remember picture,overlay]
  \draw (current page.south west) +(1in + \oddsidemargin,0.5em)
  node[anchor=south west,inner sep=0pt]{\parbox{\textwidth}{%
      \rlap{\rule{10em}{0.4pt}}\raggedright\scriptsize \textit{#1}}};}

\newcommand\myfootnotewithlink[2]{%
  \tikz[remember picture,overlay]
  \draw (current page.south west) +(1in + \oddsidemargin,0.5em)
  node[anchor=south west,inner sep=0pt]{\parbox{\textwidth}{%
      \rlap{\rule{10em}{0.4pt}}\raggedright\scriptsize\href{#1}{\textit{#2}}}};}
\createdgmtitle{9}
%--------------------------------------------------------------------------------
\begin{document}
%--------------------------------------------------------------------------------
\begin{frame}[noframenumbering,plain]
%\thispagestyle{empty}
\titlepage
\end{frame}
%=======
\begin{frame}{$\beta$-VAE}
	\begin{figure}
	    \centering
	    \includegraphics[width=0.9\linewidth]{figs/betaVAE_1.png}
	\end{figure}

	\myfootnotewithlink{https://openreview.net/references/pdf?id=Sy2fzU9gl}{Higgins I. et al. beta-VAE: Learning Basic Visual Concepts with a Constrained Variational Framework, 2017}
\end{frame}
%=======
\begin{frame}{$\beta$-VAE}
		\begin{figure}[h]
			\centering
			\includegraphics[width=.95\linewidth]{figs/betaVAE_5.png}
		\end{figure}

	\myfootnotewithlink{https://openreview.net/references/pdf?id=Sy2fzU9gl}{Higgins I. et al. beta-VAE: Learning Basic Visual Concepts with a Constrained Variational Framework, 2017}
\end{frame}
%=======
\begin{frame}{$\beta$-VAE}
	\begin{block}{ELBO}
		\vspace{-0.2cm}
		\[
		\mathcal{L}(q, \btheta, \beta) = \mathbb{E}_{q(\bz | \bx)} \log p(\bx | \bz, \btheta) - \beta \cdot KL(q(\bz | \bx) || p(\bz)).
		\]
		\vspace{-0.4cm}
	\end{block}
	\begin{block}{ELBO surgery}
		\vspace{-0.3cm}
		{\footnotesize
			\[
			\frac{1}{n} \sum_{i=1}^n \cL_i(q, \btheta, \beta) = \underbrace{\frac{1}{n} \sum_{i=1}^n \mathbb{E}_{q(\bz | \bx_i)} \log p(\bx_i | \bz, \btheta)}_{\text{Reconstruction loss}} - \beta \cdot \underbrace{\bbI_{q} [\bx, \bz]\vphantom{\sum_{i=1}}}_{\text{MI}} - \beta \cdot \underbrace{KL(q(\bz) || p(\bz))\vphantom{\sum_{i=1}}}_{\text{Marginal KL}}
			\]}
	\end{block}
	\begin{block}{Minimization of MI}
	\begin{itemize}
		\item It is not necessary and not desirable for disentanglement. 
		\item It hurts reconstruction.
	\end{itemize}
	\end{block}
	\myfootnotewithlink{https://arxiv.org/abs/1804.03599}{Burgess C. P. et al. Understanding disentangling in $\beta$-VAE, 2018}
\end{frame}
%=======
\begin{frame}{DIP-VAE}
	\begin{block}{Disentangled aggregated variational posterior}
		\vspace{-0.3cm}
		\[
		q(\bz) = \frac{1}{n} \sum_{i=1}^n q(\bz | \bx) = \prod_{j=1}^d q(z_j)
		\]
		\vspace{-0.3cm}
	\end{block}
	\begin{block}{DIP-VAE Objective}
		\vspace{-0.3cm}
		{\footnotesize
			\begin{multline*}
			\cL_{\text{DIP}}(q, \btheta) = \frac{1}{n} \sum_{i=1}^n \cL_i(q, \btheta) -\lambda \cdot KL(q(\bz) || p(\bz)) = \\
			= \frac{1}{n} \sum_{i=1}^n\left[ \bbE_{q(\bz | \bx_i)} \log p(\bx_i | \bz, \btheta) - KL(q(\bz | \bx_i) || p(\bz)) \right] -\lambda \cdot KL(q(\bz) || p(\bz)) = \\
			= \underbrace{ \frac{1}{n} \sum_{i=1}^n \left[\bbE_{q(\bz | \bx_i)} \log p(\bx_i | \bz, \btheta)\right]}_{\text{Reconstruction loss}} - \underbrace{\vphantom{\sum_{i=1}^n} \bbI_{q} [\bx, \bz]}_{\text{MI}} - (1 + \lambda) \cdot \underbrace{\vphantom{\sum_{i=1}^n} KL(q(\bz) || p(\bz))}_{\text{Marginal KL}}
			\end{multline*}
		}
		\vspace{-0.3cm}
	\end{block}

	\myfootnotewithlink{https://arxiv.org/abs/1711.00848}{Kumar A., Sattigeri P., Balakrishnan A. Variational Inference of Disentangled Latent Concepts from Unlabeled Observations, 2017}
\end{frame}
%=======
\begin{frame}{DIP-VAE}
		\vspace{-0.2cm}
		\[
			\cL_{\text{DIP}}(q, \btheta) = \frac{1}{n} \sum_{i=1}^n \cL_i(q, \btheta) -\lambda \cdot \underbrace{KL(q(\bz) || p(\bz))}_{\text{intractable}}
		\]
	Let match the moments of $q(\bz)$ and $p(\bz)$:
	\[
	\text{cov}_{q(\bz)}(\bz) = \bbE_{q(\bz)} \left[ (\bz - \bbE_{q(\bz)}(\bz)) (\bz - \bbE_{q(\bz)}(\bz))^T \right]
	\]
	DIP-VAE regularizes $\text{cov}_{q(\bz)}(\bz) $ to be close to the identity matrix. 
	\begin{block}{Objective}
		\vspace{-0.5cm}
		\begin{multline*}
		\max_{q, \btheta} \Bigl[\frac{1}{n} \sum_{i=1}^n \cL_i(q, \btheta) - \\ - \lambda_1 \sum_{i \neq j} \left[\text{cov}_{q(\bz)} (\bz) \right]^2_{ij} - \lambda_2 \sum_{i} \left( \left[ \text{cov}_{q(\bz)} (\bz) \right]_{ii} - 1 \right)^2 \Bigr]
		\end{multline*}
		\vspace{-0.5cm}
	\end{block}

	\myfootnotewithlink{https://arxiv.org/abs/1711.00848}{Kumar A., Sattigeri P., Balakrishnan A. Variational Inference of Disentangled Latent Concepts from Unlabeled Observations, 2017}
\end{frame}
%=======
\begin{frame}{DIP-VAE}
	Reconstructions become better.
	\begin{figure}
		\centering
		\includegraphics[width=\linewidth]{figs/dip-vae_1}
	\end{figure}

	\myfootnotewithlink{https://arxiv.org/abs/1711.00848}{Kumar A., Sattigeri P., Balakrishnan A. Variational Inference of Disentangled Latent Concepts from Unlabeled Observations, 2017}
\end{frame}
%=======
\begin{frame}{Challenging Disentanglement Assumptions}

\begin{block}{Theorem}
	Let $\bz \sim P$ with a density $p(\bz) = \prod^d_{i=1} p(z_i)$. Then, there exists an \textbf{infinite} family of bijective functions $f : \text{supp}(\bz) \rightarrow \text{supp}(\bz)$:
	\begin{itemize}
		\item $\frac{\partial f_i(\bz)}{\partial z_j} \neq 0$ for all $i$ and $j$ ($\bz$ and $f(\bz)$ are completely entangled);
		\item $P(\bz \leq \bu) = P(f(\bz) \leq \bu)$ for all $\bu \in \text{supp}(\bz)$.
	\end{itemize}  
\end{block}
Consider a generative model with disentangled representation $\bz$.
\begin{itemize}
	\item $\exists$ $\hat{\bz} = f(\bz)$ where $\hat{\bz}$ is completely entangled
	with respect to $\bz$.
	\item The disentanglement method cannot distinguish between the two equivalent generative models:
	\vspace{-0.3cm}
	\[
		p(\bx) = \int p(\bx | \bz) p(\bz) d\bz = \int p(\bx | \hat{\bz})p(\hat{\bz}) d \hat{\bz}.
	\]
\end{itemize}
Theorem claims that unsupervised disentanglement learning is impossible for arbitrary generative models with a factorized prior.

\myfootnotewithlink{https://arxiv.org/abs/1811.12359}{Locatello F. et al. Challenging Common Assumptions in the Unsupervised Learning of Disentangled Representations, 2018}
\end{frame}
%=======
\begin{frame}{Challenging Disentanglement Assumptions}
	\begin{figure}
		\centering
		\includegraphics[width=0.85\linewidth]{figs/challenge_dis_2}
	\end{figure}
	\vspace{-0.3cm}
	\begin{figure}
		\centering
		\includegraphics[width=0.65\linewidth]{figs/challenge_dis_3}
	\end{figure}

\myfootnotewithlink{https://arxiv.org/abs/1811.12359}{Locatello F. et al. Challenging Common Assumptions in the Unsupervised Learning of Disentangled Representations, 2018}
\end{frame}
%=======
\begin{frame}{Recap of previous lecture}
	\begin{block}{Autoregressive flow prior}
		\vspace{-0.5cm}
		\[
			\log p(\bz | \blambda) = \log p(\bepsilon) + \log \det \left | \frac{d \bepsilon}{d\bz}\right|; \quad 
			\bz = g(\bepsilon, \blambda) = f^{-1}(\bepsilon, \blambda)
		\]
	\end{block}
	\vspace{-0.4cm}
	\begin{block}{Theorem}
	VAE with the AF prior for latent code $\bz$ is equivalent to using the IAF posterior for latent code $\bepsilon$.
	\end{block}
	{\footnotesize
	\begin{align*}
		\mathcal{L}(q, \btheta) &= \mathbb{E}_{\bz \sim q(\bz | \bx)} \Bigl[ \log p(\bx | \bz, \btheta) + \underbrace{ \Bigl( \log p(f(\bz, \blambda)) + \log \left| \det \frac{\partial f(\bz, \blambda)}{\partial \bz} \right| \Bigr) }_{\text{AF prior}} - \log q(\bz | \bx) \Bigr] \\
		&= \mathbb{E}_{\bz \sim q(\bz | \bx)} \Bigl[ \log p(\bx | \bz, \btheta) +  \log p(f(\bz, \blambda)) - \underbrace{ \Bigl( \log q(\bz | \bx) - \log \left| \det \frac{\partial f(\bz, \blambda)}{\partial \bz} \right| \Bigr) }_{\text{IAF posterior}} \Bigr]
	\end{align*}
	}
	\myfootnotewithlink{https://arxiv.org/abs/1611.02731}{Chen X. et al. Variational Lossy Autoencoder, 2016}
\end{frame}
%=======
\begin{frame}{Recap of previous lecture}
	\begin{block}{ELBO}
		\vspace{-0.3cm}
		\[
			p(\bx | \btheta) \geq \mathcal{L} (\bphi, \btheta)  = \mathbb{E}_{q(\bz | \bx, \bphi)} \log \frac{p(\bx, \bz | \btheta)}{q(\bz| \bx, \bphi)} \rightarrow \max_{\bphi, \btheta}.
		\]
		\vspace{-0.5cm}
	\end{block}
		\begin{itemize}
			\item Normal variational distribution $q(\bz | \bx, \bphi) = \mathcal{N}(\bz| \bmu_{\bphi}(\bx), \bsigma^2_{\bphi}(\bx))$ is poor (e.g. has only one mode). \\
			\item Flows models convert a simple base distribution to a compex one using an invertible transformation with simple Jacobian. 
		\end{itemize}
	\begin{block}{Flow model in latent space}
		\vspace{-0.7cm}
		\[
			\log q_K(\bz_K | \bx, \bphi_*) = \log q(\bz_0 | \bx, \bphi) - \sum_{k=1}^K \log \left| \det \left( \frac{\partial g_k(\bz_{k - 1}, \bphi_k)}{\partial \bz_{k-1}} \right) \right|.
		\]
		\vspace{-0.5cm}
	\end{block}
	Let's use $q_K(\bz_K | \bx, \bphi_*), \, \bphi_* = \{\bphi, \bphi_1, \dots, \bphi_K\}$ as a variational distribution. Here, $\bphi$~-- encoder parameters, $\{\bphi_k\}_{k=1}^K$~-- flow parameters.
	
	\myfootnotewithlink{https://arxiv.org/abs/1505.05770}{Rezende D. J., Mohamed S. Variational Inference with Normalizing Flows, 2015} 
\end{frame}
%=======
\begin{frame}{Recap of previous lecture}
	\begin{block}{Variational distribution}
		\vspace{-0.6cm}
		\[
			\log q_K(\bz_K | \bx, \bphi_*) = \log q(\bz_0 | \bx, \bphi) - \sum_{k=1}^K \log \left| \det \left( \frac{\partial g_k(\bz_{k - 1}, \bphi_k)}{\partial \bz_{k-1}} \right) \right|.
		\]
		\vspace{-0.6cm}
	\end{block}
	\begin{block}{ELBO objective}
		\vspace{-0.7cm}
		\begin{align*}
			\mathcal{L} (\bphi, \btheta) 
			&= \mathbb{E}_{q(\bz_0 | \bx, \bphi)} \bigg[\log p(\bx, \bz_K | \btheta) -  \log q(\bz_0 | \bx, \bphi ) + \\ & \quad  + \sum_{k=1}^K \log \left| \det \left( \frac{\partial g_k(\bz_{k - 1}, \bphi_k)}{\partial \bz_{k-1}} \right) \right| \bigg].
		\end{align*}
		\vspace{-0.5cm}
	\end{block}
	\begin{itemize}
		\item Obtain samples $\bz_0$ from the encoder.
		\item Apply flow model $\bz_K = g(\bz_0, \{\bphi_k\}_{k = 1}^K)$.
		\item Compute likelihood for $\bz_K$ using the decoder, base distribution for $\bz_0$ and the Jacobian.
		\item We do not need an inverse flow function if we use flows in variational inference.
	\end{itemize}
	\myfootnotewithlink{https://arxiv.org/abs/1505.05770}{Rezende D. J., Mohamed S. Variational Inference with Normalizing Flows, 2015} 
\end{frame}
%=======
\begin{frame}{Recap of previous lecture}
	Images are discrete data flow is a continuous model.
	We need to convert a discrete data distribution to a continuous one.
	
	\begin{block}{Uniform dequantization bound}
		\vspace{-0.5cm}
		\[
			\bx \sim \text{Categorical}(\bpi), \quad 
			\bu \sim U[0, 1], \quad 
			\by = \bx + \bu \sim \text{Continuous} 
		\]
		\vspace{-0.4cm}
		\[
			\log P(\bx | \btheta) \geq \int_{U[0, 1]} \log p(\bx + \bu | \btheta) d \bu.
		\]
	\end{block}
	\vspace{-0.2cm}
	\begin{block}{Variational dequantization bound}
		Introduce variational dequantization noise distribution $q(\bu | \bx)$ and treat it as an approximate posterior. 
		\vspace{-0.2cm}
		\[
			\log P(\bx | \btheta) \geq  \int q(\bu | \bx) \log \frac{p(\bx + \bu | \btheta)}{q(\bu | \bx)} d \bu = \mathcal{L}(q, \btheta).
		\]
	\end{block}
	\myfootnotewithlink{https://arxiv.org/abs/1902.00275}{Ho J. et al. Flow++: Improving Flow-Based Generative Models with Variational Dequantization and Architecture Design, 2019}
\end{frame}
%=======
\begin{frame}{Recap of previous lecture}
	
	\begin{minipage}[t]{0.5\columnwidth}
		\begin{figure}
			\centering
			\includegraphics[width=1.0\linewidth]{figs/uniform_dequantization.png}
		\end{figure}
	\end{minipage}%
	\begin{minipage}[t]{0.5\columnwidth}
		\begin{figure}
			\centering
			\includegraphics[width=1.0\linewidth]{figs/variational_dequantization.png}
		\end{figure}
	\end{minipage}
	\begin{block}{Flow model for dequantization}
	\vspace{-0.3cm}
	\[
	q(\bu | \bx) = p(h^{-1}(\bu, \bphi)) \cdot \left| \det \frac{\partial h^{-1}(\bu, \bphi)}{\partial \bu}\right|.
	\]
	\vspace{-0.3cm}
	\end{block}
	\begin{block}{Variational dequantization bound}
		\[
		\mathcal{L}(q, \btheta) = \int q(\bu | \bx) \log \frac{p(\bx + \bu | \btheta)}{q(\bu | \bx)} d \bu.
		\]
	\end{block}
	\myfootnotewithlink{https://arxiv.org/abs/1902.00275}{Ho J. et al. Flow++: Improving Flow-Based Generative Models with Variational Dequantization and Architecture Design, 2019}
\end{frame}
%=======
\begin{frame}{Recap of previous lecture}
	\begin{block}{Disentanglement learning}
	A disentangled representation is a one where single latent units are sensitive to changes in single generative factors, while being invariant to changes in other factors. 
	\end{block}
	\begin{block}{$\beta$-VAE}
	\vspace{-0.2cm}
	\[
	    \mathcal{L}(q, \btheta, \beta) = \mathbb{E}_{q(\bz | \bx)} \log p(\bx | \bz, \btheta) - \beta \cdot KL (q(\bz | \bx) || p(\bz)).
	\]
	Representations becomes disentangled by setting a stronger constraint with $\beta > 1$. However, it leads to poorer reconstructions and a loss of high frequency details. 
	\end{block}
	
	\begin{block}{ELBO surgery}
		\vspace{-0.3cm}
		{\footnotesize
			\[
			\frac{1}{n} \sum_{i=1}^n \cL_i(q, \btheta, \beta) = \underbrace{\frac{1}{n} \sum_{i=1}^n \mathbb{E}_{q(\bz | \bx_i)} \log p(\bx_i | \bz, \btheta)}_{\text{Reconstruction loss}} - \beta \cdot \underbrace{\bbI_{q} [\bx, \bz]\vphantom{\sum_{i=1}}}_{\text{MI}} - \beta \cdot \underbrace{KL(q(\bz) || p(\bz))\vphantom{\sum_{i=1}}}_{\text{Marginal KL}}
			\]}
	\end{block}
	\myfootnotewithlink{https://arxiv.org/abs/1804.03599}{Burgess C. P. et al. Understanding disentangling in $\beta$-VAE, 2018}
\end{frame}
%=======
\begin{frame}{Generative models zoo}
	\begin{figure}
		\centering
		\includegraphics[width=1.0\linewidth]{figs/generative_models_zoo.pdf}
	\end{figure}
\end{frame}
%=======
\begin{frame}{Likelihood based models}
	Is likelihood a good measure of model quality?
	\begin{minipage}[t]{0.48\columnwidth}
		\begin{block}{Poor likelihood \\ Great samples}
			\vspace{-0.3cm}
			\[
				p_1(\bx) = \frac{1}{n} \sum_{i=1}^n \cN(\bx | \bx_i, \epsilon \bI)
			\]
			For small $\epsilon$ this model will generate samples with great quality, but likelihood will be very poor.
		\end{block}
	\end{minipage}%
	\begin{minipage}[t]{0.52\columnwidth}
		\begin{block}{Great likelihood \\ Poor samples}
			\vspace{-0.3cm}
			\[
				p_2(\bx) = 0.01p(\bx) + 0.99p_{\text{noise}}(\bx)
			\]
			\begin{multline*}
				\log \left[ 0.01p(\bx) + 0.99p_{\text{noise}}(\bx) \right] \geq  \\ \geq \log \left[ 0.01p(\bx) \right]  = \log p(\bx) - \log 100
			\end{multline*}
		Noisy irrelevant samples, but for high dimensions $\log p(\bx)$ becomes proportional to $m$.
		\end{block}
	\end{minipage}
	\myfootnotewithlink{https://arxiv.org/abs/1511.01844}{Theis L., Oord A., Bethge M. A note on the evaluation of generative models, 2015}
\end{frame}
%=======
\begin{frame}{Likelihood-free learning}
	\begin{itemize}
		\item Likelihood is not a perfect measure quality measure for generative model.
		\item Likelihood could be intractable.
	\end{itemize}
	\begin{block}{Where did we start}
	 We would like to approximate true data distribution $\pi(\bx)$.
		Instead of searching true $\pi(\bx)$ over all probability distributions, learn function approximation $p(\bx | \btheta) \approx \pi(\bx)$.
	\end{block}
	Imagine we have two sets of samples 
	\begin{itemize}
		\item $\cS_1 = \{\bx_i\}_{i=1}^{n_1} \sim \pi(\bx)$ -- real samples;
		\item $\cS_2 = \{\bx_i\}_{i=1}^{n_2} \sim p(\bx | \btheta)$ -- generated (or fake) samples.
	\end{itemize}
	\begin{block}{Two sample test}
		\vspace{-0.3cm}
		\[
			H_0: \pi(\bx) = p(\bx | \btheta), \quad H_1: \pi(\bx) \neq p(\bx | \btheta)
		\]
	\end{block}
	Define test statistic $T(\cS_1, \cS_2)$. The test statistic is likelihood free.
	If $T(\cS_1, \cS_2) < \alpha$, then accept $H_0$, else reject it.
\end{frame}
%=======
\begin{frame}{Likelihood-free learning}
	\begin{block}{Two sample test}
		\vspace{-0.3cm}
		\[
			H_0: \pi(\bx) = p(\bx | \btheta), \quad H_1: \pi(\bx) \neq p(\bx | \btheta)
		\]
		\vspace{-0.6cm}
	\end{block}
	\begin{block}{Desired behaviour}
		\begin{itemize}
			\item $p(\bx | \btheta)$ minimizes the value of test statistic~$T(\cS_1, \cS_2)$.
			\item It is hard to find an appropriate test statistic in high dimensions. $T(\cS_1, \cS_2)$ could be learnable.
		\end{itemize}
	\end{block}
	\begin{block}{GAN objective}
		\vspace{-0.5cm}
		\[
			\min_{G} \max_D V(G, D) = \min_{G} \max_D \left[ \bbE_{\pi(\bx)} \log D(\bx) + \bbE_{p(\bz)} \log (1 - D(G(\bz))) \right]
		\]
		\vspace{-0.4cm}
	\end{block}
	\begin{itemize}
		\item \textbf{Generator:} generative model $\bx = G(\bz)$, which makes generated sample more realistic.
		\item \textbf{Discriminator:} a classifier $D(\bx) \in [0, 1]$, which distinguishes real samples from generated samples.
	\end{itemize}
	 \myfootnotewithlink{https://arxiv.org/abs/1406.2661}{Goodfellow I. J. et al. Generative Adversarial Networks, 2014}
\end{frame}
%=======
\begin{frame}{Vanilla GAN optimality}
	\begin{block}{Theorem}
	The minimax game 
		\vspace{-0.1cm}
		\[
			\min_{G} \max_D V(G, D) = \min_{G} \max_D \left[ \bbE_{\pi(\bx)} \log D(\bx) + \bbE_{p(\bz)} \log (1 - D(G(\bz))) \right]
		\]
	has the global optimum $\pi(\bx) = p(\bx | \btheta)$, in this case $D^*(\bx) = 0.5$.
	\end{block}
	\begin{block}{Proof (fixed $G$)}
		\vspace{-0.5cm}
		\begin{align*}
			V(G, D) &= \bbE_{\pi(\bx)} \log D(\bx) + \bbE_{p(\bx | \btheta)} \log (1 - D(\bx)) \\
			&= \int \underbrace{\left[ \pi(\bx) \log D(\bx) + p(\bx | \btheta)\log (1 - D(\bx) \right]}_{y(D)} d \bx
		\end{align*}
		\vspace{-0.2cm}
		\[
			\frac{d y(D)}{d D} = \frac{\pi(\bx)}{D(\bx)} - \frac{p(\bx | \btheta)}{1 - D(\bx)} = 0 \quad \Rightarrow \quad D^*(\bx) = \frac{\pi(\bx)}{\pi(\bx) + p(\bx | \btheta)}
		\]
	\end{block}
	\myfootnotewithlink{https://arxiv.org/abs/1406.2661}{Goodfellow I. J. et al. Generative Adversarial Networks, 2014}
\end{frame}
%=======
\begin{frame}{Vanilla GAN optimality}
	\begin{block}{Proof confitnued (fixed $D = D^*$)}
		\vspace{-0.5cm}
		\begin{multline*}
			V(G, D^*) = \bbE_{\pi(\bx)} \log \frac{\pi(\bx)}{\pi(\bx) + p(\bx | \btheta)} + \bbE_{p(\bx | \btheta)} \log \frac{p(\bx | \btheta)}{\pi(\bx) + p(\bx | \btheta)} \\
		 = KL \left(\pi(\bx) || \frac{\pi(\bx) + p(\bx | \btheta)}{2}\right) + KL \left(p(\bx | \btheta) || \frac{\pi(\bx) + p(\bx | \btheta)}{2}\right) - 2\log 2 \\
		 = 2JSD(\pi(\bx) || p(\bx | \btheta)) - 2\log 2.
		\end{multline*}
	\end{block}
	\vspace{-0.3cm}
	\begin{block}{Jensen-Shannon divergence (symmetric KL divergence)}
		\vspace{-0.2cm}
		\footnotesize
		\[
			JSD(\pi(\bx) || p(\bx | \btheta)) = \frac{1}{2} \left[KL \left(\pi(\bx) || \frac{\pi(\bx) + p(\bx | \btheta)}{2}\right) + KL \left(p(\bx | \btheta) || \frac{\pi(\bx) + p(\bx | \btheta)}{2}\right) \right]
		\]
	\end{block}
	Could be used as a distance measure!
	\[
		V(G^*, D^*) = -2\log 2, \quad \pi(\bx) = p(\bx | \btheta).
	\]
	
	 \myfootnotewithlink{https://arxiv.org/abs/1406.2661}{Goodfellow I. J. et al. Generative Adversarial Networks, 2014}
\end{frame}
%=======
\begin{frame}{Vanilla GAN optimality}
	\begin{block}{Theorem}
		The minimax game 
		\vspace{-0.1cm}
		\[
			\min_{G} \max_D  V(G, D) = \min_{G} \max_D \left[ \bbE_{\pi(\bx)} \log D(\bx) + \bbE_{p(\bz)} \log (1 - D(G(\bz))) \right]
		\]
		has the global optimum $\pi(\bx) = p(\bx | \btheta)$, in this case $D^*(\bx) = 0.5$.
	\end{block}
	\vspace{-0.2cm}
	\begin{block}{Proof}
	for fixed $G$:
	\[
		D^*(\bx) = \frac{\pi(\bx)}{\pi(\bx) + p(\bx | \btheta)}
	\]
	\vspace{-0.5cm} \\
	for fixed $D = D^*$:
	\[
		\min_{G} V(G, D^*) = \min_{G} \left[ 2 JSD(\pi || p) - \log 4 \right] = -\log 4, \quad \pi(\bx) = p(\bx | \btheta).
	\]
	\vspace{-0.6cm}
	\end{block}
	If the generator could be any function and the discriminator is optimal at every step, then the generator is guaranteed to converge to the data distribution.
	 \myfootnotewithlink{https://arxiv.org/abs/1406.2661}{Goodfellow I. J. et al. Generative Adversarial Networks, 2014}
\end{frame}
%=======
\begin{frame}{Vanilla GAN}
	\begin{block}{Objective}
		\vspace{-0.4cm}
		\[
		\min_{G} \max_D V(G, D) = \min_{G} \max_D \left[ \bbE_{\pi(\bx)} \log D(\bx) + \bbE_{p(\bz)} \log (1 - D(G(\bz))) \right]
		\]
		\vspace{-0.4cm}
	\end{block}

	\begin{figure}
		\centering
		\includegraphics[width=1.0\linewidth]{figs/gan_1}
	\end{figure}
	\begin{itemize}
		\item Generator updates are made in parameter space.
		\item Discriminator is not optimal at every step.
		\item Generator and discriminator loss keeps oscillating during GAN training.
	\end{itemize}

	 \myfootnotewithlink{https://arxiv.org/abs/1406.2661}{Goodfellow I. J. et al. Generative Adversarial Networks, 2014}
\end{frame}
%=======
\begin{frame}{Summary}
	\begin{itemize}
		\item $\beta$-VAE makes the latent components more independent, but the reconstructions get poorer.
		\vfill
		\item DIP-VAE does not make the reconstructions worse using ELBO surgery theorem.
		\vfill
		\item Majority of disentanglement learning models use heuristic objective or regularizers to achieve the goal, but the task itself could not be solved without good inductive bias.
		\vfill
		\item Likelihood is not a perfect criteria to measure quality of generative model.
		\vfill
		\item Adversarial learning suggest to solve minimax problem to match the distributions.
		\vfill
		\item Vanilla GAN tries to optimize Jensen-Shannon divergence (in theory).
	\end{itemize}
\end{frame}
%=======
\end{document} 